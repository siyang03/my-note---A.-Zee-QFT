\chapter{the S-matrix and time-ordered products}
\section{the LSZ reduction formula}
\begin{itemize}
	\item S-matrix element 为
	\begin{equation}\label{the LSZ reduction formula.1.1}
		\begin{dcases}
			\ket{i} = \sqrt{(2 \pi)^3 2 \omega_{p_1}} \sqrt{(2 \pi)^3 2 \omega_{p_2}} a^\dag_{\vec{p}_1}(- \infty) a^\dag_{\vec{p}_2}(- \infty) \ket{\Omega} \\
			\ket{f} = \sqrt{(2 \pi)^3 2 \omega_{p_3}} \cdots \sqrt{(2 \pi)^3 2 \omega_{p_n}} a^\dag_{\vec{p}_3}(+ \infty) \cdots a^\dag_{\vec{p}_n}(+ \infty) \ket{\Omega} \\
			\braket{f | S | i} = (2 \pi)^{3 n / 2} \sqrt{2 \omega_{p_1} \cdots 2 \omega_{p_n}} \braket{\Omega | a_{\vec{p}_3}(+ \infty) \cdots a_{\vec{p}_n}(+ \infty) a^\dag_{\vec{p}_1}(- \infty) a^\dag_{\vec{p}_2}(- \infty) | \Omega}
		\end{dcases}.
	\end{equation}
	
	\begin{tcolorbox}[title=remark:]
		注意到
		\begin{equation}
			\begin{dcases}
				a_{\vec{p}}(t) = U^\dag(t) a_{\vec{p}}(0) U(t) \\
				\ket{\vec{p}(t)} = U(t) \ket{\vec{p}(0)}
			\end{dcases} \Longrightarrow \ket{\vec{p}(t)} = a^\dag_{\vec{p}}(- t) \ket{\Omega},
		\end{equation}
		那么, 对于初末态, 有
		\begin{align}
			& \ket{i(- \infty)} = \sqrt{(2 \pi)^3 2 \omega_{p_1}} \sqrt{(2 \pi)^3 2 \omega_{p_2}} a^\dag_{\vec{p}_1}(0) a^\dag_{\vec{p}_2}(0) \ket{\Omega} \notag \\
			\Longrightarrow & \ket{i(0)} = \cdots,
		\end{align}
		可见 \eqref{the LSZ reduction formula.1.1} 中的 $\ket{i}, \ket{f}$ 都是 $t = 0$ 即 Heisenberg picture 中的形式.
	\end{tcolorbox}
	
	\item LSZ 需要用到以下公式 (其中 $p$ is on-shell),
	\begin{equation} \label{the S-matrix and time-ordered products.1.4}
		i \int d^4 x \, e^{i p \cdot x} (\partial^2 + m^2) \phi(x) = \sqrt{(2 \pi)^3 2 \omega_p} (e^{i \omega_p t} a_{\vec{p}}(t)) \Big|^\infty_{- \infty},
	\end{equation}
	注意到, 对于 free field, 等式两边等于零.
	
	\begin{tcolorbox}[title=proof:]
		注意到
		\begin{align}
			\int d^3 x \, e^{i p \cdot x} (\partial^2 + m^2) \phi(x) &= \int d^3 x \, e^{i p \cdot x} (\partial_t^2 - \nabla^2 + m^2) \phi(x) \notag \\
			&= \int d^3 x \, (e^{i p \cdot x} (\partial_t^2 + m^2) \phi(x) - \phi(x) \nabla^2 e^{i p \cdot x}) \notag \\
			&= \int d^3 x \, e^{i p \cdot x} (\partial_t^2 + \omega_p^2) \phi(x) \notag \\
			&= \int d^3 x \, \partial_t(e^{i p \cdot x} \dot{\phi} - i \omega_p e^{i p \cdot x} \phi) \\
			&= \frac{(2 \pi)^{3 / 2}}{\sqrt{2 \omega_p}} \partial_t \big( e^{i \omega_p t} (\partial_t - i \omega_p) (a_{\vec{p}}(t) + a^\dag_{- \vec{p}}(t)) \big)
		\end{align}
		其中注意到了
		\begin{equation}
			e^{i \omega_p t} (\partial_t^2 + \omega_p^2) O(t) = \partial_t(e^{i \omega_p t} (\partial_t - i \omega_p) O(t)),
		\end{equation}
		因此
		\begin{equation}
			\text{LHS} = i \frac{(2 \pi)^{3 / 2}}{\sqrt{2 \omega_p}} (e^{i \omega_p t} (\partial_t - i \omega_p) (a_{\vec{p}}(t) + a^\dag_{- \vec{p}}(t))) \Big|^\infty_{- \infty},
		\end{equation}
		注意到 $t = \pm \infty$ 时, fields are free, 所以
		\begin{equation} \label{the S-matrix and time-ordered products.1.9}
			\begin{dcases}
				\lim_{t \rightarrow \pm \infty} a_{\vec{p}}(t) \propto e^{- i \omega_p t} a_{\vec{p}} \\
				\lim_{t \rightarrow \pm \infty} a^\dag_{- \vec{p}}(t) \propto e^{i \omega_p t} a^\dag_{- \vec{p}}
			\end{dcases} \Longrightarrow \begin{dcases}
				\lim_{t \rightarrow \pm \infty} (\partial_t - i \omega_p) a_{\vec{p}}(t) = - 2 i \omega_p a_{\vec{p}}(t) \\
				\lim_{t \rightarrow \pm \infty} (\partial_t - i \omega_p) a^\dag_{- \vec{p}}(t) = 0
			\end{dcases},
		\end{equation}
		代入得到
		\begin{equation}
			\text{LHS} = \sqrt{(2 \pi)^3 2 \omega_p} (e^{i \omega_p t} a_{\vec{p}}(t)) \Big|^\infty_{- \infty}.
		\end{equation}
	\end{tcolorbox}
	
	\item 考虑 $p_1 \neq p_2 \neq \cdots \neq p_n$ 的情况.
	
	\item 考虑
	\begin{align}
		& \frac{d}{dt} T\{A_1(t_1) \cdots A_{n - 1}(t_{n - 1}) B(t)\} \notag \\
		=& T\{A_1(t_1) \cdots A_n(t_n) \frac{dB(t)}{dt}\} + \sum_{i = 1}^n \delta(t - t_i) T\{A_1(t_1) \cdots [B(t_i), A_i(t_i)] \cdots A_n(t_n)\},
	\end{align}
	那么
	\begin{align}
		& \partial_{t_i}^2 T\{\phi_1 \cdots \phi_n\} \notag \\
		=& \partial_{t_i} T\{\cdots \partial_{t_i} \phi_i \cdots\} + \sum_{j \neq i} \delta(\cdots) T\{\cdots \underbrace{[\phi(t_j, \vec{x}_i), \phi(t_j, \vec{x}_j)]}_{= 0} \cdots\} \notag \\
		=& T\{\cdots \partial_{t_i}^2 \phi_i \cdots\} + \sum_{j \neq i} \delta(t_j - t_i) T\{\cdots [\dot{\phi}(t_j, \vec{x}_i), \phi(t_j, \vec{x}_j)] \cdots\} \notag \\
		=& T\{\cdots \partial_{t_i}^2 \phi_i \cdots\} - i \sum_{j \neq i} \delta^{(4)}(x_j - x_i) T\{\cdots \phi_{j - 1} \phi_{j + 1} \cdots\},
	\end{align}
	所以
	\begin{align}
		& i \int d^4 x_i \, e^{i p_i \cdot x_i} (\partial_i^2 + m^2) T\{\phi_1 \cdots \phi_n\} \notag \\
		=& i \int d^4 x_i \, e^{i p_i \cdot x_i} \Big( T\{\cdots (\partial_i^2 + m^2) \phi_i \cdots\} - i \sum_{j \neq i} \delta^{(4)}(x_j - x_i) T\{\cdots \phi_{j - 1} \phi_{j + 1} \cdots\} \Big) \notag \\
		=& i \int dt_i \, T\{\cdots \int d^3 x_i \, \partial_{t_i}(e^{i p_i \cdot x_i} \dot{\phi}_i - i \omega_{p_i} e^{i p_i \cdot x_i} \phi_i) \cdots\} \notag \\
		& + \sum_{j \neq i} e^{i p_i \cdot x_j} T\{\cdots \phi_{j - 1} \phi_{j + 1} \cdots\} \notag \\
		=& i \int dt_i \, \Big( \partial_{t_i} T\{\cdots \int d^3 x_i \, (e^{i p_i \cdot x_i} \dot{\phi}_i - i \omega_{p_i} e^{i p_i \cdot x_i} \phi_i) \cdots\} \notag \\
		& - \sum_{j \neq i} \delta(t_i - t_j) T\{\cdots \underbrace{[\int d^3 x_i \, e^{i (\omega_{p_i} \cdot t_j - \vec{p}_i \cdot \vec{x}_i)} (\dot{\phi}(t_j, \vec{x}_i) - i \omega_{p_i} \phi(t_j, \vec{x}_i)), \phi_j]}_{= - i e^{i p_i \cdot x_j}} \cdots\} \Big) \notag \\
		& + \sum_{j \neq i} e^{i p_i \cdot x_j} T\{\cdots \phi_{j - 1} \phi_{j + 1} \cdots\} \label{the S-matrix and time-ordered products.1.13} \\
		=& i \int dt_i \, \partial_{t_i} T\{\cdots \int d^3 x_i \, (e^{i p_i \cdot x_i} \dot{\phi}_i - i \omega_{p_i} e^{i p_i \cdot x_i} \phi_i) \cdots\} \notag \\
		=& (2 \pi)^{3 / 2} \sqrt{2 \omega_{p_i}} \Big( e^{i \omega_{p_i} (+ \infty)} a_{\vec{p}_i}(+ \infty) T\{\cdots \phi_{i - 1} \phi_{i + 1} \cdots\} \notag \\
		& - T\{\cdots \phi_{i - 1} \phi_{i + 1} \cdots\} e^{i \omega_{p_i} (- \infty)} a_{\vec{p}_i}(- \infty) \Big),
	\end{align}
	其中, \eqref{the S-matrix and time-ordered products.1.13} 将对易子抵消掉是很重要的一步.
\end{itemize}

\section{LSZ for operators} \label{the S-matrix and time-ordered products.2}
\begin{itemize}
	\item $\ket{\Omega}$ 的具体形式见 subsection \ref{subsection Feynman rules.2.1}.
	
	\item 考虑
	\begin{equation}
		\phi(x) = e^{i P \cdot x} \phi(0) e^{- i P \cdot x},
	\end{equation}
	且 $P \ket{\Omega} = 0$, 因此
	\begin{equation}
		\begin{dcases}
			\braket{\Omega | \phi(x) | \Omega} = \braket{\Omega | \phi(0) | \Omega} \\
			\braket{p | \phi(x) | \Omega} = e^{i p \cdot x} \braket{p | \phi(0) | \Omega}
		\end{dcases}.
	\end{equation}
	
	\item LSZ reduction formula 的前提是场算符满足
	\begin{equation} \label{the S-matrix and time-ordered products.2.3}
		\begin{dcases}
			\braket{\Omega | \phi(x) | \Omega} = 0 \\
			\braket{p | \phi(x) | \Omega} = e^{i p \cdot x}
		\end{dcases}.
	\end{equation}
	
	\begin{tcolorbox}[title=remark:]
		前提是场算符具有以下形式,
		\begin{equation}
			\phi(x) = \int \frac{d^3 p}{(2 \pi)^{3 / 2} \sqrt{2 \omega_p}} (a_{\vec{p}}(t) e^{i \vec{p} \cdot \vec{x}} + a^\dag_{\vec{p}}(t) e^{- i \vec{p} \cdot \vec{x}}),
		\end{equation}
		和 $a^\dag_{\vec{p}}(\pm \infty) \ket{\Omega} \in \mathrm{span}(\ket{p})$ 是一个单粒子态, 且 $\bra{\Omega} a^\dag_{\vec{p}}(+ \infty) = a_{\vec{p}}(- \infty) \ket{\Omega} = 0$.
	\end{tcolorbox}
	
	\item 注意, 在 Heisenberg picture 中,
	\begin{equation}
		\phi(t = 0, \vec{x}) = \phi_0(\vec{x}) = \int \frac{d^3 p}{(2 \pi)^{3 / 2} \sqrt{2 \omega_p}} (a_{\vec{p}} e^{i \vec{p} \cdot \vec{x}} + a^\dag_{\vec{p}} e^{- i \vec{p} \cdot \vec{x}}).
	\end{equation}
	
	\item \eqref{the S-matrix and time-ordered products.1.9} 和 $H \ket{\Omega} = 0$ (要求 $H$ 不含时) 矛盾 \textcolor{red}{(?)}, 因为 LSZ 考虑的 Hamiltonian 是
	\begin{equation}
		H = H_0 + V \theta(t - T) \theta(t + T), T \gg 0,
	\end{equation}
	所以 \eqref{the S-matrix and time-ordered products.2.3} 只在 $- T < t < T$ 成立.
\end{itemize}
