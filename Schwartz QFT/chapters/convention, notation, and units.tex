\chapter*{convention, notation, and units}
\markboth{convention, notation, and units}{} % set the header mark
\addcontentsline{toc}{chapter}{convention, notation, and units}
\begin{itemize}
	\item 笔记中的\textbf{度规号差}约定为 $\boldsymbol{(+, -, -, -)}$.
	
	\item 使用 natural units, 此时 $\hbar, c, k_B = 1$, 因此 $1 \, \text{m} = \frac{1}{1.97 \times 10^{- 16} \, \text{GeV}}$ 且:
	
	\begin{center}
		\begin{tabularx}{\linewidth}{XX}
			\toprule 
			names/dimensions & expressions/values \\
			\midrule 
			Planck length ($L$) & $l_P = \sqrt{\frac{\hbar G}{c^3}} = 1.616 \times 10^{- 35} \, \text{m}$ \\
			Planck time ($T$) & $t_P = \frac{l_P}{c} = 5.391 \times 10^{- 44} \, \text{s}$ \\
			Planck mass ($M$) & $m_P = \sqrt{\frac{\hbar c}{G}} = 2.176 \times 10^{- 8} \, \text{kg} \simeq 10^{19} \, \text{GeV}$ \\
			Planck temperature ($\Theta$) & $T_P = \sqrt{\frac{\hbar c^5}{G k_B^2}} = 1.417 \times 10^{32} \, \text{K}$ \\
			\bottomrule
		\end{tabularx}
	\end{center}
	
	\item 时空维度用 $d = D + 1$ 表示.
\end{itemize}
