\chapter{Feynman rules}
\section{Lagrangian derivation}
\begin{itemize}
	\item 假设存在相互作用时, 场算符依然满足
	\begin{equation}
		\begin{dcases}
			[\phi(t, \vec{x}), \phi(t, \vec{y})] = 0 \\
			[\phi(t, \vec{x}), \partial_t \phi(t, \vec{y})] = i \delta^{(3)}(\vec{x} - \vec{y})
		\end{dcases},
	\end{equation}
	即 causality and canonical commutation relation.
	
	\item 一个重要的中间公式为
	\begin{equation} \label{Feynman rules.1.2}
		(\partial_x^2 + m^2) \braket{\Omega | T(\phi(x) \phi(y)) | \Omega} = \braket{\Omega | T((\partial_x^2 + m^2) \phi(x) \phi(y)) | \Omega} - i \delta^{(4)}(x - y).
	\end{equation}
	
	\begin{tcolorbox}[title=proof:]
		考虑
		\begin{align}
			\partial_t \braket{\Omega | T(\phi(x) \phi(y)) | \Omega} =& \partial_t (\braket{\Omega | \phi(x) \phi(y) | \Omega} \theta(t - t') + \braket{\Omega | \phi(y) \phi(x) | \Omega} \theta(t' - t)) \notag \\
			=& \braket{\Omega | \partial_t \phi(x) \phi(y) | \Omega} \theta(t - t') + \braket{\Omega | \phi(x) \phi(y) | \Omega} \delta(t - t') \notag \\
			& + \braket{\Omega | \phi(y) \partial_t \phi(x) | \Omega} \theta(t' - t) - \braket{\Omega | \phi(y) \phi(x) | \Omega} \delta(t' - t) \notag \\
			=& \braket{\Omega | T (\partial_t \phi(x) \phi(y)) | \Omega} + \braket{\Omega | [\phi(t, \vec{x}), \phi(t, \vec{y})] | \Omega} \delta(t - t') \notag \\
			=& \braket{\Omega | T (\partial_t \phi(x) \phi(y)) | \Omega},
		\end{align}
		那么
		\begin{align}
			\partial_t^2 \braket{\Omega | T(\phi(x) \phi(y)) | \Omega} =& \partial_t \braket{\Omega | T (\partial_t \phi(x) \phi(y)) | \Omega} \notag \\
			=& \partial_t (\braket{\Omega | \partial_t \phi(x) \phi(y) | \Omega} \theta(t - t') + \braket{\Omega | \phi(y) \partial_t \phi(x) | \Omega} \theta(t' - t)) \notag \\
			=& \braket{\Omega | \partial_t^2 \phi(x) \phi(y) | \Omega} \theta(t - t') + \braket{\Omega | \partial_t \phi(x) \phi(y) | \Omega} \delta(t - t') \notag \\
			& + \braket{\Omega | \phi(y) \partial_t^2 \phi(x) | \Omega} \theta(t' - t) - \braket{\Omega | \phi(y) \partial_t \phi(x) | \Omega} \delta(t' - t) \notag \\
			=& \braket{\Omega | T (\partial_t^2 \phi(x) \phi(y)) | \Omega} + \braket{\Omega | [\partial_t \phi(t, \vec{x}), \phi(t, \vec{y})] | \Omega} \delta(t - t') \notag \\
			=& \braket{\Omega | T (\partial_t^2 \phi(x) \phi(y)) | \Omega} - i \delta^{(4)}(x - y).
		\end{align}
	\end{tcolorbox}
	
	\item \eqref{Feynman rules.1.2} 可以推广为 (其中 $\phi_i$ 是 $\phi(x_i)$ 的简写)
	\begin{align}
		& (\partial_1^2 + m^2) \braket{\Omega | T(\phi(x_1) \cdots \phi(x_n)) | \Omega} \notag \\
		=& \braket{\Omega | T((\partial_1^2 + m^2) \phi(x_1) \cdots \phi(x_n)) | \Omega} - i \sum_{i = 2}^n \delta^{(4)}(x_i - x_1) \braket{\Omega | \phi_2 \cdots \phi_{i - 1} \phi_{i + 1} \cdots \phi_n | \Omega}. \label{Feynman rules.1.5}
	\end{align}
	
	\begin{tcolorbox}[title=proof:]
		首先
		\begin{equation}
			\partial_{t_1} \braket{\Omega | T(\phi(x_1) \cdots \phi(x_n)) | \Omega} = \braket{\Omega | T(\partial_{t_1} \phi(x_1) \cdots \phi(x_n)) | \Omega},
		\end{equation}
		那么
		\begin{equation}
			\partial_{t_1}^2 \braket{\Omega | T(\phi(x_1) \cdots \phi(x_n)) | \Omega} = \partial_{t_1} \braket{\Omega | T(\partial_{t_1} \phi(x_1) \cdots \phi(x_n)) | \Omega} = \cdots 
		\end{equation}
	\end{tcolorbox}
	
	\item canonical commutation relation 保证了 quantum field 满足 Euler--Lagrange equation, 因此
	\begin{equation} \label{Feynman rules.1.8}
		(\partial^2 + m^2) \phi = - \frac{\delta}{\delta \phi} V(\phi).
	\end{equation}
	
	\item 结合 \eqref{Feynman rules.1.5} 和 \eqref{Feynman rules.1.8} 得到 Schwinger--Dyson equations,
	\begin{align}
		& (\partial_1^2 + m^2) \braket{\Omega | T(\phi(x_1) \cdots \phi(x_n)) | \Omega} \notag \\
		=& \braket{\Omega | T(- \frac{\delta}{\delta \phi_1} V(\phi_1) \cdots \phi(x_n)) | \Omega} - i \sum_{i = 2}^n \delta^{(4)}(x_i - x_1) \braket{\Omega | \phi_2 \cdots \phi_{i - 1} \phi_{i + 1} \cdots \phi_n | \Omega}.
	\end{align}
\end{itemize}

\subsection{position-space Feynman rules}
\begin{itemize}
	\item Feynman propagator 为 ($\phi_0(x)$ 是 free field, 本 chapter 不做特别说明都是存在相互作用的 Heisenberg picture)
	\begin{equation}
		D_F(x - y) := \braket{0 | T(\phi_0(x) \phi_0(y)) | 0} = \int \frac{d^4 k}{(2 \pi)^4} \frac{i}{k^2 - m^2 + i \epsilon} e^{i k \cdot (x - y)},
	\end{equation}
	满足
	\begin{equation}
		(\partial^2 + m^2) D_F(x - y) = - i \delta^{(4)}(x - y).
	\end{equation}
	
	\item 2-point correlation function 可以重写作 (注意到 $\lim_{x \rightarrow \infty} D_{x y} = \lim_{x \rightarrow \infty} \partial_x D_{x y} = 0$)
	\begin{align}
		\braket{\Omega | T(\phi_1 \phi_2) | \Omega} &= i \int d^4 x \, ((\partial_x^2 + m^2) D_{x 1}) \braket{\Omega | T(\phi_x \phi_2) | \Omega} \notag \\
		&= i \int d^4 x \, D_{x 1} (\partial_x^2 + m^2) \braket{\Omega | T(\phi_x \phi_2) | \Omega} \notag \\
		&= D_{1 2} + i \int d^4 x \, D_{x 1} \braket{\Omega | T(- \frac{\delta}{\delta \phi_x} V(\phi_x) \phi_2) | \Omega},
	\end{align}
	类似地, 4-point function 可以写作
	\begin{align}
		\braket{\Omega | T(\phi_1 \phi_2 \phi_3 \phi_4) | \Omega} =& D_{1 2} D_{3 4} + D_{1 3} D_{2 4} + D_{1 4} D_{2 3} \notag \\
		& + i \int d^4 x \, D_{x 1} \braket{\Omega | T(- \frac{\delta}{\delta \phi_x} V(\phi_x) \phi_2 \phi_3 \phi_4) | \Omega} \notag \\
		& + D_{1 2} i \int d^4 y \, D_{3 y} \braket{\Omega | T(- \frac{\delta}{\delta \phi_y} V(\phi_y) \phi_4) | \Omega} + \cdots
	\end{align}
	
	\item 考虑 $V(\phi) = - \frac{g}{3!} \phi^3$, 那么 2-point function 为
	\begin{align}
		\braket{\Omega | T(\phi_1 \phi_2) | \Omega} &= D_{1 2} + i \int d^4 x \, D_{x 1} \braket{\Omega | T(\frac{g}{2!} \phi_x^2 \phi_2) | \Omega} \notag \\
		&= D_{1 2} + i \int d^4 x \, D_{x 1} \Big( i \int d^4 y \, D_{y 2} \braket{\frac{g}{2!} \phi_x^2 \frac{g}{2!} \phi_y^2} + 2! D_{x 2} \braket{\frac{g}{2!} \phi_x} \Big),
	\end{align}
	其中
	\begin{equation}
		\begin{dcases}
			\begin{aligned}
				\braket{\phi_x} &= i \int d^4 y \, D_{y x} (\partial_y^2 + m^2) \braket{\phi_y} = i \int d^4 y \, D_{y x} \braket{\frac{g}{2!} \phi_y^2} \\
				&= i \frac{g}{2!} \int d^4 y \, D_{y x} D_{y y} + O(g^2)
			\end{aligned} \\
			\begin{aligned}
				\braket{\phi_x^2 \phi_y^2} &= D_{x x} D_{y y} + 2 D_{x y} D_{x y} + O(g)
			\end{aligned}
		\end{dcases},
	\end{equation}
	因此
	\begin{align}
		\braket{\phi_1 \phi_2} &= \vcenter{\hbox{\includegraphics[scale=1]{figures/position space Feynman rules - D_{1 2}.pdf}}} + \vcenter{\hbox{\includegraphics[scale=1]{figures/position space Feynman rules - frac{(i g)^2}{4} D_{1 x} D_{x x} D_{y y} D_{y 2}.pdf}}} + \vcenter{\hbox{\includegraphics[scale=1]{figures/position space Feynman rules - frac{(i g)^2}{2} D_{1 x} D_{x y} D_{x y} D_{y 2}.pdf}}} + \vcenter{\hbox{\includegraphics[scale=1]{figures/position space Feynman rules - frac{(i g)^2}{2} D_{1 x} D_{x y} D_{y y} D_{x 2}.pdf}}} + O(g^3) \notag \\
		&= D_{1 2} + \frac{(i g)^2}{4} D_{1 x} D_{x x} D_{y y} D_{y 2} + \frac{(i g)^2}{2} D_{1 x} D_{x y} D_{x y} D_{y 2} + \frac{(i g)^2}{2} D_{1 x} D_{x y} D_{y y} D_{x 2} + O(g^3),
	\end{align}
	其中 $4, 2$ 是 symmetry factors.
	
	\item 下图的 symmetry factor 为 $1$,
	\begin{equation}
		\vcenter{\hbox{\includegraphics[scale=1]{figures/position space Feynman rules - (i g)^3 D_{1 x} D_{x y} D_{y z} D_{z x} D_{y 2} D_{z 3}.pdf}}} = (i g)^3 D_{1 x} D_{x y} D_{y z} D_{z x} D_{y 2} D_{z 3}.
	\end{equation}
\end{itemize}

\section{Hamiltonian derivation}
\begin{itemize}
	\item interaction picture...
\end{itemize}

\subsection{vacuum matrix elements} \label{subsection Feynman rules.2.1}
\begin{itemize}
	\item section \ref{the S-matrix and time ordered products.2} 中注意到了 $\ket{\Omega}$ 的定义 (in Heisenberg picture),
	\begin{equation}
		\begin{dcases}
			a_{\vec{p}}(- \infty) \ket{\Omega} = U^\dag(- \infty) a_{\vec{p}} U(- \infty) \ket{\Omega} = 0 \\
			a_{\vec{p}}(+ \infty) \ket{\Omega} = U^\dag(+ \infty) a_{\vec{p}} U(+ \infty) \ket{\Omega} = 0
		\end{dcases} \Longrightarrow U(\pm \infty) \ket{\Omega} \propto \ket{0} ,
	\end{equation}
	那么 Schrodinger picture 和 interaction picture 中分别有
	\begin{equation}
		\begin{dcases}
			\ket{\Omega(t)} = \mathcal{N}_i U(t, - \infty) \ket{0} = \mathcal{N}_f U(t, + \infty) \ket{0} & \text{Schrodinger picture} \\
			\ket{\Omega(t)}_I = \mathcal{N}_i U_I(t, - \infty) \ket{0} = \mathcal{N}_f U_I(t, + \infty) \ket{0} & \text{interaction picture}
		\end{dcases}.
	\end{equation}
	
	\begin{tcolorbox}[title=calculation:]
		令
		\begin{equation}
			\ket{\Omega} \equiv \ket{\Omega(0)} = \mathcal{N}_i U(0, - \infty) \ket{0} = \mathcal{N}_f U(0, + \infty) \ket{0},
		\end{equation}
		且
		\begin{equation}
			\begin{dcases}
				U_I(0, - \infty) U_0(0, - \infty) = U(0, - \infty) \\
				U_I(0, + \infty) U_0(0, + \infty) = U(0, + \infty)
			\end{dcases} \quad \text{and} \quad U_0(0, \pm \infty) \ket{0} = \ket{0},
		\end{equation}
		因此
		\begin{equation}
			\ket{\Omega}_I = \begin{dcases}
				\mathcal{N}_i U_I(t, 0) U(0, - \infty) \ket{0} = \mathcal{N}_i U_I(t, - \infty) \ket{0} \\
				\mathcal{N}_f U_I(t, 0) U(0, + \infty) \ket{0} = \mathcal{N}_f U_I(t, + \infty) \ket{0}
			\end{dcases}.
		\end{equation}
	\end{tcolorbox}
	
	\begin{itemize}
		\item 归一化要求 $\braket{\Omega | \Omega} = \mathcal{N}^*_f \mathcal{N}_i \braket{0 | U(- \infty, + \infty) | 0} = \mathcal{N}^*_f \mathcal{N}_i \braket{0 | U_I(- \infty, + \infty) | 0} = |\mathcal{N}_i|^2 = |\mathcal{N}_f|^2 = 1$.
		
		\item $H \ket{\Omega} = 0$ 因为 $[H, U(t_1, t_2)] = 0$ (前提是 Hamiltonian 不含时).
	\end{itemize}
	
	\item 因此
	\begin{equation}
		\braket{\Omega | T(\phi_1 \cdots \phi_n) | \Omega} = \frac{\braket{0 | T(\phi_{0, 1} \cdots \phi_{0, n} U_I(+ \infty, - \infty)) | 0}}{\braket{0 | U_I(+ \infty, - \infty) | 0}}.
	\end{equation}
	
	\begin{tcolorbox}[title=calculation:]
		\begin{align}
			& \braket{\Omega | T(\phi_1 \cdots \phi_n) | \Omega} \notag \\
			&= \mathcal{N}^*_f \mathcal{N}_i \braket{0 | U(+ \infty, 0) T((U(0, t_1) \phi_{S, 1} U(t_1, 0)) (U(0, t_2) \phi_{S, 2} U(t_2, 0)) \cdots) U(0, - \infty) | 0} \notag \\
			&= \mathcal{N}^*_f \mathcal{N}_i \braket{0 | U_I(+ \infty, 0) T((U_I(0, t_1) U_0(0, t_1) \phi_{S, 1} U_0(t_1, 0) U_I(t_1, 0)) \cdots) U_I(0, - \infty) | 0} \notag \\
			&= \mathcal{N}^*_f \mathcal{N}_i \braket{0 | U_I(+ \infty, 0) T((U_I(0, t_1) \phi_{0, 1} U_I(t_1, 0)) \cdots) U_I(0, - \infty) | 0} = \cdots
		\end{align}
		另外注意到
		\begin{equation}
			\mathcal{N}^*_f \mathcal{N}_i \braket{0 | U_I(- \infty, + \infty) | 0} = 1.
		\end{equation}
	\end{tcolorbox}
\end{itemize}
