\chapter{Maxwell's equations}
\section{Maxwell's equations}
\begin{itemize}
	\item 电磁场的 Lagrangian 为 (现实中 $\mu = 0$),
	\begin{equation} \label{11.1.1}
		\mathcal{L} = - \frac{1}{4} F^{\mu \nu} F_{\mu \nu} + \frac{1}{2} \mu^2 A^\mu A_\mu \quad \text{with} \quad F_{\mu \nu} = 2 \partial_{[\mu} A_{\nu]}
	\end{equation}
	其中 $A_\mu = (\phi, - \vec{A})$, 对作用量变分得到运动方程,
	\begin{equation} \label{11.1.2}
		\partial^\nu F_{\nu \mu} + \mu^2 A_\mu = 0
	\end{equation}
	\begin{itemize}
		\item 如果引入 Lorentz gauge condition,
		\begin{equation}
			\begin{dcases}
				\text{field eq. \eqref{11.1.2}} \\
				\partial^\mu A_\mu = 0
			\end{dcases} \Longrightarrow (\partial^2 + \mu^2) A_\mu = 0
		\end{equation}
	\end{itemize}
	
	\item 此外, $F_{\mu \nu}$ 满足 Bianchi identity,
	\begin{equation} \label{11.1.4}
		\nabla_\rho F_{\mu \nu} + \nabla_\nu F_{\rho \mu} + \nabla_\mu F_{\nu \rho} = 0
	\end{equation}
	
	\begin{tcolorbox}[title=calculation:]
		代入定义式,
		\begin{align}
			\nabla_\rho \nabla_{[\mu} A_{\nu]} + \cdots =& + \mathcolor{red}{\rho \mu \nu} - \mathcolor{blue}{\rho \nu \mu} \notag \\
			& + \mathcolor{blue}{\nu \rho \mu} - \mathcolor{orange}{\nu \mu \rho} \notag \\
			& + \mathcolor{orange}{\mu \nu \rho} - \mathcolor{red}{\mu \rho \nu} \notag \\
			=& (\underbrace{\tensor{R}{_{\rho \mu \nu}^\sigma} + \tensor{R}{_{\nu \rho \mu}^\sigma} + \tensor{R}{_{\mu \nu \rho}^\sigma}}_{= 0}) A_\sigma
		\end{align}
	\end{tcolorbox}
	
	\noindent\rule[0.5ex]{\linewidth}{0.5pt} % horizontal line
	
	\item 令,
	\begin{equation}
		F_{\mu \nu} = \begin{pmatrix}
			0 & E_1 & E_2 & E_3 \\
			- E_1 & 0 & - B_3 & B_2 \\
			- E_2 & B_3 & 0 & - B_1 \\
			- E_3 & - B_2 & B_1 & 0
		\end{pmatrix} \iff \begin{dcases}
			F_{0 i} = \vec{E} = - \vec{\nabla} \phi - \frac{\partial \vec{A}}{\partial t} \\
			- \frac{1}{2} \epsilon^{i j k} F_{j k} = \vec{B} = \vec{\nabla} \times \vec{A}
		\end{dcases}
	\end{equation}
	代入 \eqref{11.1.2},
	\begin{equation}
		\begin{dcases}
			\vec{\nabla} \cdot \vec{E} + \mu^2 \phi = 0 \\
			- \frac{\partial \vec{E}}{\partial t} + \vec{\nabla} \times \vec{B} + \mu^2 \vec{A} = 0
		\end{dcases}
	\end{equation}
	代入 \eqref{11.1.4},
	\begin{equation}
		\begin{dcases}
			\vec{\nabla} \cdot \vec{B} = 0 & \rho, \mu, \nu = 1, 2, 3 \\
			\vec{\nabla} \times \vec{E} - \frac{\partial \vec{B}}{\partial t} = 0 & \rho, \mu, \nu = 0, i, j
		\end{dcases}
	\end{equation}
	
	\item 最后, 电磁场的能动量张量见 subsection \ref{subsection D.4.1}.
\end{itemize}

\section{gauge symmetry}
\begin{itemize}
	\item 
\end{itemize}
