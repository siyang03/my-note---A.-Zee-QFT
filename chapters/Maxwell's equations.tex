\chapter{Maxwell's equations}
\section{the \texorpdfstring{$(1, 0) \oplus (0, 1)$}{(1, 0)+(0, 1)} representation of the Lorentz algebra}
\begin{itemize}
	\item 反对称张量 $F_{[\mu \nu]}$ 是 $(1, 0) \oplus (0, 1)$ rep. 中的向量, 详见笔记 \href{https://github.com/siyang03/my-note---Lie-Groups-and-Lie-Algebras}{Lie Groups and Lie Algebras}.
\end{itemize}

\section{Maxwell's equations}
\begin{itemize}
	\item 电磁场的 Lagrangian 为 (现实中 $\mu = 0$)
	\begin{equation} \label{11.2.1}
		\mathcal{L} = - \frac{1}{4} F^{\mu \nu} F_{\mu \nu} + \frac{1}{2} \mu^2 A^\mu A_\mu, \quad \text{with} \quad F_{\mu \nu} = 2 \partial_{[\mu} A_{\nu]},
	\end{equation}
	其中 $A_\mu = (\phi, - \vec{A})$, 对作用量变分得到运动方程,
	\begin{equation} \label{11.2.2}
		\partial^\nu F_{\nu \mu} + \mu^2 A_\mu = 0.
	\end{equation}
	
	\begin{tcolorbox}[title=calculation:]
		场方程还可以写成
		\begin{equation}
			\Big( - \partial^\mu \partial^\nu + \eta^{\mu \nu} (\partial^2 + \mu^2) \Big) A_\nu = 0 \iff (k^2 - \mu^2) \tilde{A}_\mu(k) = k_\mu k^\nu \tilde{A}_\nu(k)
		\end{equation}
		和
		\begin{equation}
			\begin{dcases}
				- \nabla^2 A_0 - \nabla \cdot \frac{\partial \vec{A}}{\partial t} + \mu^2 A_0 = 0 \\
				\frac{\partial}{\partial t} \Big( - \nabla A_0 - \frac{\partial \vec{A}}{\partial t} \Big) + \Big( \underbrace{\nabla^2 \vec{A} - \nabla (\nabla \cdot \vec{A})}_{= - \nabla \times (\nabla \times \vec{A})} \Big) - \mu^2 \vec{A} = 0
			\end{dcases}.
		\end{equation}
	\end{tcolorbox}
	
	\begin{itemize}
		\item 如果引入 Lorentz gauge condition (如 \eqref{2.1.12} 所示, 在 $\mu \neq 0$ 时必然成立),
		\begin{equation}
			\text{field eq. \eqref{11.2.2}} \overset{\mu \neq 0}{\iff} \begin{dcases}
				(\partial^2 + \mu^2) A_\mu = 0 \\
				\partial^\mu A_\mu = 0
			\end{dcases}.
		\end{equation}
	\end{itemize}
	
	\item 此外, $F_{\mu \nu}$ 满足 Bianchi identity,
	\begin{equation} \label{11.2.4}
		\nabla_\rho F_{\mu \nu} + \nabla_\nu F_{\rho \mu} + \nabla_\mu F_{\nu \rho} = 0.
	\end{equation}
	
	\begin{tcolorbox}[title=calculation:]
		代入定义式,
		\begin{align}
			\nabla_\rho \nabla_{[\mu} A_{\nu]} + \cdots =& + \mathcolor{red}{\rho \mu \nu} - \mathcolor{blue}{\rho \nu \mu} \notag \\
			& + \mathcolor{blue}{\nu \rho \mu} - \mathcolor{orange}{\nu \mu \rho} \notag \\
			& + \mathcolor{orange}{\mu \nu \rho} - \mathcolor{red}{\mu \rho \nu} \notag \\
			=& (\underbrace{\tensor{R}{_{\rho \mu \nu}^\sigma} + \tensor{R}{_{\nu \rho \mu}^\sigma} + \tensor{R}{_{\mu \nu \rho}^\sigma}}_{= 0}) A_\sigma.
		\end{align}
	\end{tcolorbox}
	
	\noindent\rule[0.5ex]{\linewidth}{0.5pt} % horizontal line
	
	\item 令
	\begin{equation}
		F_{\mu \nu} = \begin{pmatrix}
			0 & E_1 & E_2 & E_3 \\
			- E_1 & 0 & - B_3 & B_2 \\
			- E_2 & B_3 & 0 & - B_1 \\
			- E_3 & - B_2 & B_1 & 0
		\end{pmatrix} \iff \begin{dcases}
			F_{0 i} = \vec{E} = - \nabla \phi - \frac{\partial \vec{A}}{\partial t} \\
			- \frac{1}{2} \epsilon^{i j k} F_{j k} = \vec{B} = \nabla \times \vec{A}
		\end{dcases},
	\end{equation}
	代入场方程 \eqref{11.2.2},
	\begin{equation}
		\begin{dcases}
			\nabla \cdot \vec{E} + \mu^2 \phi = 0 \\
			- \frac{\partial \vec{E}}{\partial t} + \nabla \times \vec{B} + \mu^2 \vec{A} = 0
		\end{dcases},
	\end{equation}
	代入 Bianchi identity \eqref{11.2.4},
	\begin{equation}
		\begin{dcases}
			\nabla \cdot \vec{B} = 0 & \rho, \mu, \nu = 1, 2, 3 \\
			\nabla \times \vec{E} - \frac{\partial \vec{B}}{\partial t} = 0 & \rho, \mu, \nu = 0, i, j
		\end{dcases}.
	\end{equation}
	
	\item 最后, 电磁场的能动量张量见 subsection \ref{subsection D.4.1}.
\end{itemize}

\subsection{gauge symmetry}
\begin{itemize}
	\item $A_\mu$ 有 4 个分量, 但光子只有 2 个自由度 (偏振态).
	
	\item 首先, 考虑 $\tilde{A}_\mu$ 的场方程, 对于 $\mu = 0$ 有
	\begin{equation}
		\tilde{A}_0 = \frac{k^0}{\omega_k^2} \vec{k} \cdot \vec{\tilde{A}},
	\end{equation}
	这是一个约束条件, 将此式代入剩余的场方程, 得到
	\begin{equation} \label{11.2.12}
		(k^2 - \mu^2) \vec{\tilde{A}} = (k^2 - \mu^2) \frac{\vec{k} (\vec{k} \cdot \vec{\tilde{A}})}{\omega_k^2},
	\end{equation}
	因此
	\begin{equation}
		\begin{dcases}
			\text{when} \ \mu = 0: & \tilde{A}_0 = \frac{k^0}{|\vec{k}|} |\vec{\tilde{A}}|, \vec{\tilde{A}} = |\vec{\tilde{A}}| \hat{e}_k \\
			\text{when} \ \mu \neq 0: & (k^2 - \mu^2) (\vec{k} \cdot \vec{\tilde{A}}) = 0
		\end{dcases}.
	\end{equation}
	
	\begin{tcolorbox}[title=calculation:]
		对 \eqref{11.2.12} 两边同时内积 $\vec{k}$, 有
		\begin{equation}
			(k^2 - \mu^2) (\vec{k} \cdot \vec{\tilde{A}}) = 0 \quad \text{or} \quad \omega_k = |\vec{k}|,
		\end{equation}
		因此 $\mu = 0$ 时 \eqref{11.2.12} 自然成立, 而 massive 情况下 $\tilde{A}_0 = \vec{k} \cdot \vec{\tilde{A}} = 0$ 除非 on-shell.
	\end{tcolorbox}
	
	\item 除此之外还需要一个约束条件.
	
	\item 
\end{itemize}
