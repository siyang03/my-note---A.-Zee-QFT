\chapter{perturbation theory in QM}
\begin{itemize}
	\item this chapter is based on MIT OpenCourseWare \href{https://ocw.mit.edu/courses/8-06-quantum-physics-iii-spring-2018/resources/mit8_06s18ch1/}{Quantum Physics III Chapter 1: Perturbation Theory}.
	
	\noindent\rule[0.5ex]{\linewidth}{0.5pt} % horizontal line
	
	\item 研究的 Hamiltonian 与 well studied Hamiltonian 有微小差异时, 使用 perturbation theory,
	\begin{equation}
		H(\lambda) = H^{(0)} + \lambda \delta H,
	\end{equation}
	其中 $\lambda \in [0, 1]$.
	
	\item 考虑 $H^{(0)}$ 的本征态为
	\begin{equation}
		H^{(0)} \ket{k^{(0)}} = E_k^{(0)} \ket{k^{(0)}} \quad \text{and} \quad \begin{dcases}
			\braket{k^{(0)} | l^{(0)}} = \delta_{k l} \\
			E_0^{(0)} \leq E_1^{(0)} \leq E_2^{(0)} \leq \cdots
		\end{dcases}.
	\end{equation}
\end{itemize}

\section{non-degenerate perturbation theory}
\begin{itemize}
	\item 考虑 non-degenerate 能级 $k$, 有 $\cdots \leq E_{k - 1}^{(0)} < E_k^{(0)} < E_{k + 1}^{(0)} \leq \cdots$, 在 perturbation theory 适用的情况下,
	\begin{equation} \label{C.1.1}
		\begin{dcases}
			\ket{k}_\lambda = \ket{k^{(0)}} + \lambda \ket{k^{(1)}} + \lambda^2 \ket{k^{(2)}} + \cdots \\
			E_k(\lambda) = E_k^{(0)} + \lambda E_k^{(1)} + \lambda^2 E_k^{(2)} + \cdots
		\end{dcases}.
	\end{equation}
	\begin{itemize}
			\item 注意, 我们可以选取修正项满足
			\begin{equation} \label{C.1.2}
				\braket{k^{(0)} | k^{(n)}} = 0, n = 1, 2, \cdots.
			\end{equation}
			
			\begin{tcolorbox}[title=proof:]
				假设我们求解得到的修正项不满足 $\braket{k^{(0)} | k^{(n)}} = 0, n = 1, 2, \cdots$, 考虑
				\begin{equation}
					\ket{k^{(n)}}' = \ket{k^{(n)}} + a_n \ket{k^{(0)}} \quad \text{with} \quad \braket{k^{(0)} | k^{(n)}}' = 0,
				\end{equation}
				那么 (注意到态矢量可以乘一个常数, $\frac{1}{1 - a_1 \lambda - a_2 \lambda^2 - \cdots} = 1 + a_1 \lambda + (a_1^2 + a_2) \lambda^2 + \cdots$)
				\begin{align}
					\ket{k}_\lambda &= (1 - a_1 \lambda - a_2 \lambda^2 - \cdots) \ket{k^{(0)}} + \lambda \ket{k^{(1)}}' + \lambda^2 \ket{k^{(2)}}' + \cdots \notag \\
					\ket{k}'_\lambda &= \ket{k^{(0)}} + \frac{1}{1 - a_1 \lambda - a_2 \lambda^2 - \cdots} \Big( \lambda \ket{k^{(1)}}' + \lambda^2 \ket{k^{(2)}}' + \cdots \Big) \notag \\
					&= \ket{k^{(0)}} + \lambda \ket{k^{(1)}}' + \lambda^2 (a_1 \ket{k^{(1)}}' + \ket{k^{(2)}}') + \cdots,
				\end{align}
				可见修正项都与 $\ket{k^{(0)}}$ 正交.
			\end{tcolorbox}
			
			\item 注意, 不能要求 $\leftindex_\lambda {\braket{k | k}}_\lambda = 1$, 否则 $\ket{k^{(n)}}$ 将与 $\lambda$ 相关 (包括 $\ket{k^{(0)}}$),
			\begin{align}
				\leftindex_\lambda {\braket{k | k}}_\lambda =& \braket{k^{(0)} | k^{(0)}} \notag \\
				& + \lambda (\braket{k^{(1)} | k^{(0)}} + \braket{k^{(0)} | k^{(1)}}) \notag \\
				& + \lambda^2 (\braket{k^{(2)} | k^{(0)}} + \braket{k^{(1)} | k^{(1)}} + \braket{k^{(0)} | k^{(2)}}) \notag \\
				& \multicolumn{1}{c}{\vdots} \notag \\
				& + \lambda^n (\braket{k^{(n)} | k^{(0)}} + \braket{k^{(n - 1)} | k^{(1)}} + \cdots + \braket{k^{(0)} | k^{(n)}}).
			\end{align}
	\end{itemize}
	
	\item 将 \eqref{C.1.1} 代入 Schrödinger's eq., 得到:
	
	\begin{center}
		\newcolumntype{C}[1]{>{\centering\arraybackslash}p{#1}}
		\begin{tabularx}{\linewidth}{C{2cm}X}
			\toprule 
			$\lambda^0$ & $(H^{(0)} - E_k^{(0)}) \ket{k^{(0)}} = 0$ \\
			$\lambda^1$ & $(H^{(0)} - E_k^{(0)}) \ket{k^{(1)}} = (E_k^{(1)} - \delta H) \ket{k^{(0)}}$ \\
			$\lambda^2$ & $(H^{(0)} - E_k^{(0)}) \ket{k^{(2)}} = (E_k^{(1)} - \delta H) \ket{k^{(1)}} + E_k^{(2)} \ket{k^{(0)}}$ \\
			$\vdots$ & \multicolumn{1}{c}{$\vdots$} \\
			$\lambda^n$ & $(H^{(0)} - E_k^{(0)}) \ket{k^{(n)}} = (E_k^{(1)} - \delta H) \ket{k^{(n - 1)}} + E_k^{(2)} \ket{k^{(n - 2)}} +\cdots + E_k^{(n)} \ket{k^{(0)}}$ \\
			\bottomrule
		\end{tabularx}
	\end{center}
	
	\begin{tcolorbox}[title=calculation:]
		Schrödinger's eq. 为
		\begin{equation}
			(H^{(0)} + \lambda \delta H - E_k(\lambda)) \ket{k}_\lambda = 0,
		\end{equation}
		展开为
		\begin{align}
			\Big( (H^{(0)} - E_k^{(0)}) + \lambda (\delta H - E_k^{(1)}) - \lambda^2 E_k^{(2)} - \cdots \Big) (\ket{k^{(0)}} + \lambda \ket{k^{(1)}} + \lambda^2 \ket{k^{(2)}} + \cdots) = 0.
		\end{align}
	\end{tcolorbox}
	
	\item 现在来计算 $\braket{l^{(0)} | k^{(n)}}$, 有
	\begin{equation} \label{C.1.8}
		\begin{dcases}
			\begin{aligned}
				(E_l^{(0)} - E_k^{(0)}) \braket{l^{(0)} | k^{(1)}} =& E_k^{(1)} \delta_{l k} - \delta H_{l k} \\
				(E_l^{(0)} - E_k^{(0)}) \braket{l^{(0)} | k^{(2)}} =& E_k^{(1)} \braket{l^{(0)} | k^{(1)}} - \braket{l^{(0)} | \delta H | k^{(1)}} + E_k^{(2)} \delta_{l k} \\
				\multicolumn{1}{c}{\vdots} & \multicolumn{1}{c}{\vdots} \\
				(E_l^{(0)} - E_k^{(0)}) \braket{l^{(0)} | k^{(n)}} =& E_k^{(1)} \braket{l^{(0)} | k^{(n - 1)}} - \braket{l^{(0)} | \delta H | k^{(n - 1)}} \\
				& + E_k^{(2)} \braket{l^{(0)} | k^{(n - 2)}} + \cdots + E_k^{(n)} \delta_{l k}
			\end{aligned}
		\end{dcases},
	\end{equation}
	其中 $\delta H_{l k} = \braket{l^{(0)} | \delta H | k^{(0)}}$, 对于满足 \eqref{C.1.2} 的解, 有
	\begin{equation}
		E_k^{(n)} = \braket{k^{(0)} | \delta H | k^{(n - 1)}}, n = 1, 2, \cdots,
	\end{equation}
	并且
	\begin{equation} \label{C.1.10}
		\ket{k^{(1)}} = - \sum_{l \neq k} \frac{\delta H_{l k}}{E_l^{(0)} - E_k^{(0)}} \ket{l^{(0)}} \Longrightarrow E_k^{(2)} = - \sum_{l \neq k} \frac{|\delta H_{l k}|^2}{E_l^{(0)} - E_k^{(0)}}.
	\end{equation}
	
	\begin{tcolorbox}[title=calculation:]
		将 \eqref{C.1.10} 代入 \eqref{C.1.8}, 得到 ($l \neq k$)
		\begin{equation}
			(E_l^{(0)} - E_k^{(0)}) \braket{l^{(0)} | k^{(2)}} = - E_k^{(1)} \frac{\delta H_{l k}}{E_l^{(0)} - E_k^{(0)}} + \sum_{m \neq k} \frac{\delta H_{l m} \delta H_{m k}}{E_m^{(0)} - E_k^{(0)}},
		\end{equation}
		所以
		\begin{equation}
			\begin{dcases}
				\ket{k^{(2)}} = \sum_{l \neq k} \Big( - \frac{\delta H_{0 0} \delta H_{l k}}{(E_l^{(0)} - E_k^{(0)})^2} + \sum_{m \neq k} \frac{\delta H_{l m} \delta H_{m k}}{E_m^{(0)} - E_k^{(0)}} \Big) \ket{l^{(0)}} \\
				E_k^{(3)} = \sum_{l \neq k} \Big( - \frac{\delta H_{0 0} |\delta H_{l k}|^2}{(E_l^{(0)} - E_k^{(0)})^2} + \sum_{m \neq k} \frac{\delta H_{k l} \delta H_{l m} \delta H_{m k}}{E_m^{(0)} - E_k^{(0)}} \Big)
			\end{dcases}.
		\end{equation}
		
		\noindent\rule[0.5ex]{\linewidth}{0.5pt} % horizontal line
		
		计算归一化系数
		\begin{equation}
			\leftindex_\lambda {\braket{k | k}}_\lambda = 1 + \lambda^2 \sum_{l \neq k} \frac{|\delta H_{l k}|^2}{(E_l^{(0)} - E_k^{(0)})^2} + O(\lambda^3).
		\end{equation}
	\end{tcolorbox}
\end{itemize}

\subsection{level repulsion or the seesaw mechanism} \label{subsection C.1.1}
\begin{itemize}
	\item 能量的展开式为
	\begin{equation}
		E_k(\lambda) = E_k^{(0)} + \lambda \delta H_{k k} - \lambda^2 \sum_{l \neq k} \frac{|\delta H_{l k}|^2}{E_l^{(0)} - E_k^{(0)}} + O(\lambda^3),
	\end{equation}
	二阶项的效果是使能级间距增大, 对于基态能级, 二阶项使其能量减小.
\end{itemize}

\subsection{validity of the perturbation expansion}
\begin{itemize}
	\item 考虑两能级系统, 可以得出微扰展开收敛的条件, 即
	\begin{equation}
		|\lambda V| < \frac{1}{2} \Delta E^{(0)},
	\end{equation}
	因此, 对于能级简并的情况, $\Delta E^{(0)} = 0$, 情况会更复杂.
	
	\begin{tcolorbox}[title=calculation:]
		对于两能级系统
		\begin{equation}
			H(\lambda) = H^{(0)} + \lambda \hat{V} = \begin{pmatrix}
				E_1^{(0)} & \lambda V \\
				\lambda V^* & E_2^{(0)}
			\end{pmatrix},
		\end{equation}
		$H(\lambda)$ 的本征值可以直接计算,
		\begin{equation}
			E_{\pm}(\lambda) = \frac{1}{2} (E_1^{(0)} + E_2^{(0)}) \pm \frac{1}{2} (E_1^{(0)} - E_2^{(0)}) \sqrt{1 + \bigg( \frac{\lambda |V|}{\frac{1}{2} (E_1^{(0)} - E_2^{(0)})} \bigg)^2},
		\end{equation}
		考虑 $\sqrt{1 + z^2}$ 的 Taylor 展开,
		\begin{equation}
			\sqrt{1 + z^2} = 1 + \frac{z^2}{2} - \frac{z^4}{8} + \cdots + (- 1)^{n + 1} \frac{(2 n - 3)!!}{2^n n!} z^{2 n} + \cdots,
		\end{equation}
		注意到 $\sqrt{1 + z^2}$ 在 $z = \pm i$ 有 branch cut, 因此 $z = 0$ 附近的 Taylor expansion 只有在 $|z| < 1$ 内才收敛.
		
		\begin{figure}[H]
			\centering
			\includegraphics[scale=0.5]{figures/sqrt{1 + z^2}.pdf}
			\caption{graph of $\sqrt{1 + z^2}$.}
			\label{graph of sqrt{1 + z^2}.}
		\end{figure}
	\end{tcolorbox}
\end{itemize}

\section{degenerate perturbation theory}
\begin{itemize}
	\item 暂时先跳过.
\end{itemize}
