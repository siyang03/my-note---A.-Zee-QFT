\chapter{Coulomb and Newton: repulsive and attraction}
\section{massive spin-\texorpdfstring{$1$}{1} particle \& QED}
\begin{itemize}
	\item 构造有质量的光子的 Lagrangian density,
	\begin{equation} \label{3.1.1}
		\mathcal{L} = - \frac{1}{4} F_{\mu \nu} F^{\mu \nu} - \frac{1}{2} m^2 A_\mu A^\mu
	\end{equation}
	其中 $F_{\mu \nu} = 2 \partial_{[\mu} A_{\nu]}$.
	
	\item 做路径积分,
	\begin{equation}
		Z(J) = \int DA \, e^{i \int d^d x (\mathcal{L} + J_\mu A^\mu)} = \mathcal{C} e^{- \frac{i}{2} \int d^d x d^d y \, J_\mu D^{\mu \nu}(x - y) J_\nu(y)}
	\end{equation}
	
	\begin{tcolorbox}[title=calculation:]
		massive photon 的作用量为,
		\begin{align}
			S(A) &= \int d^d x \, \frac{1}{2} \Big( - (\partial_\mu A_\nu) (\partial^\mu A^\nu) + (\partial_\mu A_\nu) (\partial^\nu A^\mu) - m^2 A_\mu A^\mu \Big) \notag \\
			&= \int d^d x \, \frac{1}{2} \Big( A_\nu \partial^2 A^\nu - A_\nu \partial^\nu \partial_\mu A^\mu - m^2 A_\mu A^\mu \Big) + \text{total differential} \notag \\
			&= \int d^d x \, \frac{1}{2} A_\mu \Big( - \partial^\mu \partial^\nu + \eta^{\mu \nu} (\partial^2 - m^2) \Big) A_\nu + \text{total differential} \notag \\
			&= \int \frac{d^d k}{(2 \pi)^d} \tilde{A}_\mu(- k) \Big( k^\mu k^\nu + \eta^{\mu \nu} (- k^2 - m^2) \Big) \tilde{A}_\nu(k) + \text{boundary term}
		\end{align}
		那么, 需要有,
		\begin{align}
			& (- \partial^\mu \partial^\rho + \eta^{\mu \rho} (\partial^2 - m^2)) D_{\rho \nu}(x - y) = \delta^\mu_\nu \delta^{(d)}(x - y) \notag \\
			\Longrightarrow & \tilde{D}_{\mu \nu}(k) = \frac{k_\mu k_\nu / m^2 + \eta_{\mu \nu}}{- k^2 - m^2}
		\end{align}
	\end{tcolorbox}
	
	考虑到积分需要收敛, 作替换 $m^2 \mapsto m^2 - i \epsilon$, (为什么 $A_\mu$ 类空 \textcolor{red}{(?)}, 只知道 $\tilde{A}_\mu$ 类空, 见 subsection \ref{subsection 3.1.2}, 但路径积分中的 $A$ 显然不满足 field equation).
	
	\item 因此,
	\begin{align}
		W(J) &= - \frac{1}{2} \int d^d x d^d y \, J_\mu(x) D^{\mu \nu}(x - y) J_\nu(y) \\
		&= - \frac{1}{2} \int \frac{d^d k}{(2 \pi)^d} \tilde{J}_\mu(- k) \frac{k^\mu k^\nu / m^2 + \eta^{\mu \nu}}{- k^2 - m^2 + i \epsilon} \tilde{J}_\nu(k)
	\end{align}
	注意到 current conservation, 有 $\partial_\mu J^\mu = 0 \iff k^\mu \tilde{J}_\mu(k) = 0$, 所以,
	\begin{equation}
		W(J) = - \frac{1}{2} \int \frac{d^d k}{(2 \pi)^d} \tilde{J}^\mu(- k) \frac{1}{- k^2 - m^2 + i \epsilon} \tilde{J}_\mu(k)
	\end{equation}
	观察电荷分量, 可见同性相斥, 异性相吸.
\end{itemize}

\subsection{spin \& polarization vector} \label{subsection 3.1.1}
\begin{itemize}
	\item spin-$1$ particle 可以有 3 个极化方向, 即空间的 $x, y, z$ 方向, 在粒子静止系下, 极化矢量 $(\epsilon^i)_\mu = \delta^i_\mu, i = 1, 2, 3$, 而 $k_\mu = (- m, 0, 0, 0)$, 所以,
	\begin{equation}
		k^\mu (\epsilon^i)_\mu = 0
	\end{equation}
	\begin{itemize}
		\item 注意, 一个粒子的极化方向用 $e^i$ (这不是矢量) 表示, 极化矢量为 $\sum_{i = 1}^3 e^i (\epsilon^i)_\mu$.
	\end{itemize}
	
	\item 在粒子静止系下, 考虑,
	\begin{equation}
		\sum_{i = 1}^3 (\epsilon^i)_\mu (\epsilon^i)_\nu = \begin{pmatrix}
			0 & 0 \\
			0 & \delta_{i j}
		\end{pmatrix} = \frac{k_\mu k_\nu}{m^2} + \eta_{\mu \nu} := - G_{\mu \nu}
	\end{equation}
	可见,
	\begin{equation}
		\tilde{D}_{\mu \nu}(k) = \frac{\sum_{i = 1}^3 (\epsilon^i)_\mu (\epsilon^i)_\nu}{- k^2 - m^2 + i \epsilon}
	\end{equation}
\end{itemize}

\subsection{Maxwell Lagrangian} \label{subsection 3.1.2}
\begin{itemize}
	\item 根据 \eqref{3.1.1} 中的 Lagrangian density, 得到 field equation 如下,
	\begin{equation} \label{3.1.11}
		\Big( - \partial^\mu \partial^\nu + \eta^{\mu \nu} (\partial^2 - m^2) \Big) A_\nu
	\end{equation}
	\begin{itemize}
		\item spin-$1$ particle 有 3 个自旋自由度, 而 $A_\mu$ 有 4 个分量, 所以需要一个约束方程,
		\begin{equation}
			\partial^\mu A_\mu = 0 \iff k^\mu \tilde{A}_\mu(k) = 0
		\end{equation}
		实际上在 \eqref{3.1.11} 左右两边作用一个 $\partial_\mu$ 即可得到这个约束方程.
	\end{itemize}
\end{itemize}

\section{massive spin-\texorpdfstring{$2$}{2} particle \& gravity}
\begin{itemize}
	\item Lagrangian for spin-$2$ particle $=$ \colorbox{yellow}{linearized} Einstein Lagrangian.
	
	\item 受 subsection \ref{subsection 3.1.1} 启发, 对于 spin-$2$ particle, 其极化矢量有 5 个方向, 满足,
	\begin{equation} \label{3.2.1}
		\begin{dcases}
			k^\mu (\epsilon^a)_{(\mu \nu)} = 0 \\
			\eta^{\mu \nu} (\epsilon^a)_{(\mu \nu)} = 0
		\end{dcases}
	\end{equation}
	其中下指标 $\mu, \nu$ 对称, $a = 1, \cdots, 5$, (可以验证 $(\epsilon^a)_{\mu \nu}$ 确实有 5 个独立分量).
	\begin{itemize}
		\item 对 $(\epsilon^{a})_{\mu \nu}$ 的归一化条件可以定义为 $\sum_{a = 1}^5 (\epsilon^a)_{1 2} (\epsilon^a)_{1 2} = 1$.
		
		\item 与 subsection \ref{3.1.1} 中提示一样, 粒子的极化方向用 $e^a$ 表示.
	\end{itemize}
	
	\item 那么,
	\begin{equation}
		\sum_{a = 1}^5 (\epsilon^a)_{\mu \nu} (\epsilon^a)_{\rho \sigma} = (G_{\mu \rho} G_{\nu \sigma} + G_{\mu \sigma} G_{\nu \rho}) - \frac{2}{3} G_{\mu \nu} G_{\rho \sigma}
	\end{equation}
	
	\begin{tcolorbox}[title=calculation:]
		首先用 $k_\mu$ 和 $\eta_{\mu \nu}$ 构造最一般的关于 $\mu \leftrightarrow \nu, \rho \leftrightarrow \sigma, \mu \nu \leftrightarrow \rho \sigma$ 对称的 4 阶张量, (下式中把 $\frac{k_\mu}{m}$ 略写作 $k_\mu$),
		\begin{align}
			&A k_\mu k_\nu k_\rho k_\sigma + B (k_\mu k_\nu \eta_{\rho \sigma} + k_\rho k_\sigma \eta_{\mu \nu}) + C (k_\mu k_\rho \eta_{\nu \sigma} + k_\mu k_\sigma \eta_{\nu \rho} + k_\nu k_\rho \eta_{\mu \sigma} + k_\nu k_\sigma \eta_{\mu \rho}) \notag \\
			& + D \eta_{\mu \nu} \eta_{\rho \sigma} + E (\eta_{\mu \rho} \eta_{\nu \sigma} + \eta_{\mu \sigma} \eta_{\nu \rho})
		\end{align}
		代入 \eqref{3.2.1} 得,
		\begin{equation}
			\begin{dcases}
				0 = - A + B + 2 C = - B + D = - C + E \\
				0 = - A + 4 B + 4 C = - B + 4 D + 2 E
			\end{dcases} \Longrightarrow \frac{B = D, C = E}{A} = - \frac{1}{2}, \frac{3}{4}
		\end{equation}
		因此, 这个 4 阶张量最终确定为,
		\begin{equation}
			\frac{3}{4} A \Big( (G_{\mu \rho} G_{\nu \sigma} + G_{\mu \sigma} G_{\nu \rho}) - \frac{2}{3} G_{\mu \nu} G_{\rho \sigma} \Big)
		\end{equation}
	\end{tcolorbox}
	\item 所以,
	\begin{equation}
		\tilde{D}_{\mu \nu \rho \sigma}(k) = \frac{(G_{\mu \rho} G_{\nu \sigma} + G_{\mu \sigma} G_{\nu \rho}) - \frac{2}{3} G_{\mu \nu} G_{\rho \sigma}}{- k^2 - m^2 + i \epsilon}
	\end{equation}
	
	\noindent\rule[0.5ex]{\linewidth}{0.5pt} % horizontal line
	
	\item 计算路径积分中的 $W(T)$,
	\begin{equation}
		W(T) = - \frac{1}{2} \int \frac{d^4 k}{(2 \pi)^4} \tilde{T}_{\mu \nu}(- k) \frac{(G^{\mu \rho} G^{\nu \sigma} + G^{\mu \sigma} G^{\nu \rho}) - \frac{2}{3} G^{\mu \nu} G^{\rho \sigma}}{- k^2 - m^2 + i \epsilon} \tilde{T}_{\rho \sigma}(k)
	\end{equation}
	注意到 $\partial_\mu T^{\mu \nu}(x) = 0 \iff k_\mu \tilde{T}^{\mu \nu}(k) = 0$, 并考虑到 $T$ 是对称张量, 所以,
	\begin{equation}
		W(T) = - \frac{1}{2} \int \frac{d^4 k}{(2 \pi)^4} \tilde{T}_{\mu \nu}(- k) \frac{2 \eta^{\mu \rho} \eta^{\nu \sigma} - \frac{2}{3} \eta^{\mu \nu} \eta^{\rho \sigma}}{- k^2 - m^2 + i \epsilon} \tilde{T}_{\rho \sigma}(k)
	\end{equation}
	考虑能量项, 可见质量互相吸引.
\end{itemize}

\section{remarks}
\begin{itemize}
	\item 由于 seesaw mechanism (见 subsection \ref{C.1.1}), 引入扰动一般会降低基态能量, 因此大多数相互作用表现为吸引, 而 spin-$1$ 表现为同性相斥是因为 $\eta^{0 0} = - 1$.
	
	\item 本 chapter 中的计算都是 $m \neq 0$ 的粒子, 与真实世界有差异.
\end{itemize}
