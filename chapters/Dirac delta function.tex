\chapter{Dirac delta function \& Fourier transformation}
\section{Delta function}
\begin{itemize}
	\item 可以认为以下是定义式,
	\begin{equation}
		\delta(x) = \int \frac{dk}{2 \pi} e^{i k x} \iff \tilde{\delta}(k) = 1 = \int dx \, \delta(x) e^{- i k x}
	\end{equation}
	
	\noindent\rule[0.5ex]{\linewidth}{0.5pt} % horizontal line
	
	\item 第一个常用的公式,
	\begin{equation}
		\int_{- \infty}^{+ \infty} \delta(f(x)) g(x) dx = \sum_{\{i, f(x_i) = 0\}} \frac{g(x_i)}{|f'(x_i)|}
	\end{equation}
	
	\item 第二个常用的公式 (\href{https://en.wikipedia.org/wiki/Sokhotski%E2%80%93Plemelj_theorem}{Sokhotski-Plemelj theorem}),
	\begin{equation}
		\lim_{\epsilon \rightarrow 0^+} \frac{1}{x + i \epsilon} = \mathcal{P} \frac{1}{x} - i \pi \delta(x)
	\end{equation}
	其中 $\mathcal{P}$ 表示复函数的主值 (principal value).
	
	\begin{tcolorbox}[title=proof:]
		考虑,
		\begin{equation}
			\frac{1}{x + i \epsilon} = \frac{x - i \epsilon}{x^2 + \epsilon^2}
		\end{equation}
		且注意到,
		\begin{equation}
			\int \frac{\epsilon}{x^2 + \epsilon^2} dx = 2 \pi i \, \mathrm{Res}(f, i \epsilon) = \pi
		\end{equation}
		所以...
		
		\noindent\rule[0.5ex]{\linewidth}{0.5pt} % horizontal line
		
		取 $\epsilon = 0.1$ 时, 复变函数的实部, 虚部分别如下,
		
		\begin{figure}[H]
			\centering
			\includegraphics[scale=0.5]{figures/frac{1}{x + i epsilon}.pdf}
		\end{figure}
	\end{tcolorbox}
\end{itemize}

\section{Fourier transformation}
\begin{itemize}
	\item $d$-dim. Fourier transformation 如下,
	\begin{equation}
		\begin{dcases}
			\phi(x) = \int \frac{d^d k}{(2 \pi)^d} e^{i k \cdot x} \tilde{\phi}(k) \\
			\tilde{\phi}(k) = \int d^d x \, e^{- i k \cdot x} \phi(x)
		\end{dcases}
	\end{equation}
	
	\item 因此,
	\begin{equation}
		\partial_\mu \phi(x) \mapsto i k_\mu \tilde{\phi}(k)
	\end{equation}
\end{itemize}
