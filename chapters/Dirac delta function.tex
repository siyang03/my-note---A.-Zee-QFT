\chapter{Dirac delta function \& Fourier transformation}
\section{Delta function}
\begin{itemize}
	\item 可以认为以下是定义式,
	\begin{equation}
		\delta(x) = \int \frac{dk}{2 \pi} e^{i k x} \iff \tilde{\delta}(k) = 1 = \int dx \, \delta(x) e^{- i k x}.
	\end{equation}
	
	\noindent\rule[0.5ex]{\linewidth}{0.5pt} % horizontal line
	
	\item 第一个常用的公式,
	\begin{equation}
		\int_{- \infty}^{+ \infty} \delta(f(x)) g(x) dx = \sum_{\{i, f(x_i) = 0\}} \frac{g(x_i)}{|f'(x_i)|}.
	\end{equation}
	
	\item 第二个常用的公式 (\href{https://en.wikipedia.org/wiki/Sokhotski%E2%80%93Plemelj_theorem}{Sokhotski-Plemelj theorem}),
	\begin{equation}
		\lim_{\epsilon \rightarrow 0^+} \frac{1}{x + i \epsilon} = \mathcal{P} \frac{1}{x} - i \pi \delta(x),
	\end{equation}
	其中 $\mathcal{P}$ 表示复函数的主值 (principal value).
	
	\begin{tcolorbox}[title=proof:]
		考虑
		\begin{equation}
			\frac{1}{x + i \epsilon} = \frac{x - i \epsilon}{x^2 + \epsilon^2} \quad \text{and} \quad \int \frac{\epsilon}{x^2 + \epsilon^2} dx = 2 \pi i \, \mathrm{Res}(f, i \epsilon) = \pi,
		\end{equation}
		所以...
		
		\noindent\rule[0.5ex]{\linewidth}{0.5pt} % horizontal line
		
		取 $\epsilon = 0.1$ 时, 复变函数的实部, 虚部分别如下:
		
		\begin{figure}[H]
			\centering
			\includegraphics[scale=0.5]{figures/frac{1}{z + i epsilon}.pdf}
			\caption{graph of $\frac{1}{z + i \epsilon}$.}
		\end{figure}
	\end{tcolorbox}
	
	\item 另外, $\delta(x - a) \delta(x - b) = \delta(b - a) \delta(x - a)$.
\end{itemize}

\section{Fourier transformation}
\begin{itemize}
	\item $d$-dim. Fourier transformation 如下,
	\begin{equation} \label{A.2.1}
		\begin{dcases}
			\phi(x) = \int \frac{d^d k}{(2 \pi)^d} e^{i k \cdot x} \tilde{\phi}(k) \\
			\tilde{\phi}(k) = \int d^d x \, e^{- i k \cdot x} \phi(x)
		\end{dcases},
	\end{equation}
	
	\item 因此
	\begin{equation}
		\partial_\mu \phi(x) \mapsto i k_\mu \tilde{\phi}(k).
	\end{equation}
	
	\noindent\rule[0.5ex]{\linewidth}{0.5pt} % horizontal line
	
	\item 对于\textbf{实函数}, Fourier transformation 是正交变换, 其 Jacobi determinant 为
	\begin{equation} \label{A.2.3}
		\Big| \frac{\partial \phi(x) \cdots}{\partial \mathrm{Re} \tilde{\phi}(k) \cdots \partial \mathrm{Im} \tilde{\phi}(k) \cdots} \Big| = \Big( \frac{2}{V} \Big)^{(2 N + 1)^d} \det A = \Big( \frac{2 (2 N)^d}{V^2} \Big)^{\frac{(2 N + 1)^d}{2}}.
	\end{equation}
	
	\begin{tcolorbox}[title=proof:]
		position space 和 momentum space 的格点分别为
		\begin{equation}
			\begin{dcases}
				x_i^\mu = i^\mu \epsilon \in \{0, \pm \epsilon, \cdots, \frac{L}{2}\} \\
				k_n^\mu = n^\mu \frac{2 \pi}{L} \in \{0, \pm \frac{2 \pi}{L}, \cdots, \frac{\pi}{\epsilon}\}
			\end{dcases} \iff i^\mu, n^\mu \in \{0, \pm 1, \cdots, \pm N\},
		\end{equation}
		$x^\mu, k^\mu$ 分别有 $2 N + 1$ 个取值, 其中 $N \epsilon = \frac{L}{2}$, 时空总体积为 $V = L^d$, momentum space 的总体积为 $\tilde{V} = \frac{(4 \pi N)^d}{V}$.
		
		\noindent\rule[0.5ex]{\linewidth}{0.5pt} % horizontal line
		
		将 \eqref{A.2.1} 写成格点求和的形式,
		\begin{equation} \label{A.2.5}
			\begin{dcases}
				\begin{aligned}
					\phi(x_i) &= \frac{1}{(2 \pi)^d} \Big( \frac{2 \pi}{L} \Big)^d \sum_n e^{i k_n \cdot x_i} \tilde{\phi}(k_n) \\
					&= \frac{2}{V} \sum_{n^0 > 0} \Big( \cos(k_n \cdot x_i) \mathrm{Re} \tilde{\phi}(k_n) - \sin(k_n \cdot x_i) \mathrm{Im} \tilde{\phi}(k_n) \Big) \\
					\tilde{\phi}(k_n) &= \epsilon^d \sum_i e^{- i k_n \cdot x_i} \phi(x_i) \\
					&= \frac{V}{(2 N)^d} \sum_i \Big( \cos(k_n \cdot x_i) - i \sin(k_n \cdot x_i) \Big) \phi(x_i)
				\end{aligned}
			\end{dcases}.
		\end{equation}
		
		\noindent\hdashrule[0.5ex]{\linewidth}{0.5pt}{1mm} % horizontal dashed line
		
		\textbf{proof:}
		
		$\phi(x_i)$ 的变换需要做一些说明. 注意到 $\tilde{\phi}$ 的分量的数量是 $\phi$ 的两倍 (考虑到实部与虚部), 但在 $\phi \in \mathbb{R}^{(2 N + 1)^d}$ 时,
		\begin{equation}
			\tilde{\phi}^*(k) = \tilde{\phi}(- k),
		\end{equation}
		可见 $\tilde{\phi}$ 的分量并不独立, 取 $k^0 > 0$ 的部分为独立分量, 那么...
		
		\noindent\hdashrule[0.5ex]{\linewidth}{0.5pt}{1mm} % horizontal dashed line
		
		将 \eqref{A.2.5} 写成矩阵的形式,
		\begin{equation}
			\begin{dcases}
				\begin{pmatrix}
					\phi(x_0) \\
					\vdots \\
					\phi(x_{\max})
				\end{pmatrix} = \frac{2}{V} \overbrace{\begin{pmatrix}
					\cos k_0 \cdot x_0 & \cdots & \cos k_{\max} \cdot x_0 & - \sin k_0 \cdot x_0 & \cdots \\
					\vdots & & \ddots & &
				\end{pmatrix}}^{= A} \begin{pmatrix}
					\mathrm{Re} \tilde{\phi}(k_0) \\
					\vdots \\
					\mathrm{Im} \tilde{\phi}(k_0) \\
					\vdots
				\end{pmatrix} \\
				\begin{pmatrix}
					\mathrm{Re} \tilde{\phi}(k_0) \\
					\vdots \\
					\mathrm{Im} \tilde{\phi}(k_0) \\
					\vdots
				\end{pmatrix} = \frac{V}{(2 N)^d} \begin{pmatrix}
					\cos k_0 \cdot x_0 & \cdots & \cos k_0 \cdot x_{\max} \\
					\vdots & \ddots & \\
					- \sin k_0 \cdot x_0 & \cdots & - \sin k_0 \cdot x_{\max} \\
					\vdots & & \ddots
				\end{pmatrix} \begin{pmatrix}
					\phi(x_0) \\
					\vdots \\
					\phi(x_{\max})
				\end{pmatrix}
			\end{dcases},
		\end{equation}
		观察可见 $\tilde{\phi}$ 的变换中的矩阵是 $A^T$, 所以
		\begin{equation}
			\frac{2}{V} \frac{V}{(2 N)^d} A A^T = I \Longrightarrow \det A = \Big( \frac{(2 N)^d}{2} \Big)^{\frac{(2 N + 1)^d}{2}},
		\end{equation}
		因此...
	\end{tcolorbox}
	
	\begin{itemize}
		\item 顺便,
		\begin{equation}
			\int d^d x \, f(x) g(x) = \int \frac{d^d k}{(2 \pi)^d} \tilde{f}(- k) \tilde{g}(k).
		\end{equation}
	\end{itemize}
\end{itemize}

\subsection{an important example} \label{subsection A.2.1}
\begin{itemize}
	\item 考虑如下 PDE,
	\begin{equation}
		(\nabla^2 - \mu^2) \phi(\vec{x}) = f(\vec{x}),
	\end{equation}
	其 Green's function 为
	\begin{equation}
		G(\vec{x}) = - \frac{1}{4 \pi} \frac{e^{- \mu r}}{r},
	\end{equation}
	其中 $r = |\vec{x}|$.
	
	\begin{tcolorbox}[title=calculation:]
		Green's function 满足
		\begin{equation}
			(\nabla^2 - \mu^2) G(\vec{x}) = \delta^{(3)}(\vec{x}) \Longrightarrow \tilde{G}(\vec{k}) = - \frac{1}{|\vec{k}|^2 + \mu^2},
		\end{equation}
		因此
		\begin{align}
			G(\vec{x}) &= - \int \frac{d^3 k}{(2 \pi)^3} \frac{e^{i \vec{k} \cdot \vec{x}}}{|\vec{k}|^2 + \mu^2} \notag \\
			&= - \int \frac{k^2 \sin \theta d\theta d\phi dk}{(2 \pi)^3} \frac{e^{i \cos \theta k r}}{k^2 + \mu^2} = - \frac{1}{(2 \pi)^3} \int_0^\pi \sin \theta d\theta \int_0^{2 \pi} d\phi \int_0^\infty dk \, \frac{k^2 e^{i \cos \theta k r}}{k^2 + \mu^2} \notag \\
			&= - \frac{1}{(2 \pi)^2} \int_0^\infty \frac{k^2}{k^2 + \mu^2} \frac{2 \sin k r}{k r} dk,
		\end{align}
		注意到 (在复平面上考虑以下积分并使用 residue theorem)
		\begin{align}
			\int_0^\infty \frac{k \sin k r}{k^2 + \mu^2} dk &= \frac{1}{2} \int_{- \infty}^{+ \infty} \frac{k \sin k r}{k^2 + \mu^2} dk = \frac{1}{2 i} \int_{- \infty}^{+ \infty} \frac{k e^{i k r}}{k^2 + \mu^2} dk \notag \\
			&= \frac{1}{2 i} \int_{- \infty}^{+ \infty} \frac{k e^{i k r}}{(k + i \mu) (k - i \mu)} dk = \frac{1}{2 i} 2 \pi i \mathrm{Res}(f, k = i \mu) = \frac{\pi}{2} e^{- \mu r}.
		\end{align}
	\end{tcolorbox}
\end{itemize}
