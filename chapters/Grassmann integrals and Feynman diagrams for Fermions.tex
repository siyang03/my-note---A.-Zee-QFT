\chapter{Grassmann integrals and Feynman diagrams for Fermions}
\section{Grassmann number and Grassmann integrals}
\begin{itemize}
	\item 对于 Grassmann number $\theta_1, \theta_2$, 有反对易关系,
	\begin{equation}
		\theta_1 \theta_2 = - \theta_2 \theta_1
	\end{equation}
	因此 $\theta^2 = 0$, 且关于 Grassmann number 最一般的函数为,
	\begin{equation}
		f(\theta) = a \theta + b
	\end{equation}
	其中 $a, b \in \mathbb{C}$.
	
	\item 注意到 $(\theta_1 \theta_2) \theta_3 = \theta_3 (\theta_1 \theta_2)$, (但是 $(\theta_1 \theta_2)^2 = 0$, 所以 $\theta_1 \theta_2 \notin \mathbb{C}$), 且有,
	\begin{equation}
		(\theta_1 \theta_2) (\theta_3 \theta_4) = \theta_3 (\theta_1 \theta_2) \theta_4 = (\theta_3 \theta_4) (\theta_1 \theta_2)
	\end{equation}
	
	\item 定义 Grassmann integral (也称作 Berezin integral),
	\begin{equation}
		\int d\theta \, \theta = 1 \quad \int d\theta = 0
	\end{equation}
	并且具有 linearity.
	
	\begin{tcolorbox}[title=comment:]
		我们希望积分在 integration variable been shifted 之后 ($\theta \mapsto \theta + \eta$) 保持不变,
		\begin{equation}
			\int d\theta \, (a \theta + b) = \int d\theta \, (a \theta + a \eta + b)
		\end{equation}
		因此, 积分结果应该与常数无关, 只与斜率有关, 所以直接定义,
		\begin{equation}
			\int d\theta \, (a \theta + b) = a
		\end{equation}
	\end{tcolorbox}
	
	\begin{itemize}
		\item 另外, 对于 $f(\theta) = \eta \theta + b$, 有,
		\begin{equation}
			\int d\theta \, (\eta \theta + b) = \int d\theta \, (- \theta \eta + b) = - \eta
		\end{equation}
	\end{itemize}
\end{itemize}

\subsection{Gaussian-Berezin integrals}
\begin{itemize}
	\item 回顾 section \ref{1.4} 和 \eqref{8.4.2}, 我们希望 Gauss 积分中出现正号而不是符号, 即,
	\begin{equation}
		\int dx \, e^{- \frac{1}{2} a x^2} = \sqrt{2 \pi} e^{- \frac{1}{2} \ln a} \mapsto \propto e^{\mathcolor{red}{+} \frac{1}{2} \ln a}
	\end{equation}
	
	\item 对于两个独立的 Grassmann number $\theta, \bar{\theta}$, 有 Gauss 积分,
	\begin{equation}
		\int d\theta \int d\bar{\theta} \, e^{\bar{\theta} a \theta} = \int d\theta \int d\bar{\theta} \, (1 + \bar{\theta} a \theta) = a = e^{\mathcolor{red}{+} \ln a}
	\end{equation}
	
	\item 推广以上积分, 对于 $\theta = (\theta_1, \cdots, \theta_N) \in V, \bar{\theta} = (\bar{\theta}_1, \cdots, \bar{\theta}_N) \in V^*$, 有,
	\begin{equation}
		\int d\theta \int d\bar{\theta} \, e^{\bar{\theta} A \theta} = \det A
	\end{equation}
	其中 $A$ 是 $N \times N$ normal matrix.
	
	\begin{tcolorbox}[title=calculation:]
		对向量做幺正变换, $\eta = U \theta, \bar{\eta} = \bar{\theta} U^\dag$, 使得 $A$ 对角化 $D = U A U^\dag$, (注意对\textbf{积分顺序}的定义),
		\begin{align}
			I &= \int d\eta \int d\bar{\eta} \, e^{\bar{\eta} D \eta} \notag \\
			&= \sum_{n = 0}^\infty \int d\eta_N \cdots d\eta_1 \int d\bar{\eta}_1 \cdots d\bar{\eta}_N \, \frac{\big( \sum_{i = 1}^N \bar{\eta}_i D_i \eta_i \big)^n}{n!}
		\end{align}
		其中, 唯一不为零的项是 $\propto \prod_{i = 1}^N (\bar{\eta}_i D_i \eta_i)$, 并且注意到 $(\bar{\eta}_i D_i \eta_i)$ 互相对易, 所以,
		\begin{align}
			I &= \int d\eta_N \cdots d\eta_1 \int d\bar{\eta}_1 \cdots d\bar{\eta}_N \, \frac{n! \prod_{i = 1}^N (\bar{\eta}_i D_i \eta_i)}{n!} \notag \\
			&= \int d\eta_N \cdots d\eta_1 \int d\bar{\eta}_1 \cdots d\bar{\eta}_N \, (\bar{\eta}_N D_N \eta_N) \cdots (\bar{\eta}_1 D_1 \eta_1) \notag \\
			&= \int d\eta_N \cdots d\eta_1 \int d\bar{\eta}_1 \cdots d\bar{\eta}_{N - 1} \, \overbrace{(\bar{\eta}_{N - 1} D_{N - 1} \eta_{N - 1}) \cdots (\bar{\eta}_1 D_1 \eta_1)}^{\text{commutes with} \ \eta_N} D_N \eta_N \notag \\
			&= \cdots = \int d\eta_N \cdots d\eta_1 \, D_1 \eta_1 \cdots D_N \eta_N = \prod_{i = 1}^N D_i = \det A
		\end{align}
		注意到, 由于 $(\bar{\eta}_i D_i \eta_i)$ 互相对易, 所以 $\eta, \bar{\eta}$ 的积分顺序并不重要, 唯一的要求是 $\eta$ 和 $\bar{\eta}$ 的积分顺序互相对应 (顺序正好\textbf{相反}), 即 $d\eta_j d\eta_i \leftrightarrow d\bar{\eta}_i d\bar{\eta}_j$.
	\end{tcolorbox}
	
	\item 进一步推广,
	\begin{equation} \label{10.1.13}
		\int d\theta \int d\bar{\theta} \, e^{\bar{\theta} A \theta + \bar{\eta} \theta + \bar{\theta} \eta} = \det A \, e^{- \bar{\eta} A^{- 1} \eta}
	\end{equation}
	只需要注意到 $(\bar{\theta} + \bar{\eta} A^{- 1}) A (\theta + A^{- 1} \eta) = \bar{\theta} A \theta + \bar{\eta} \theta + \bar{\theta} \eta + \bar{\eta} A^{- 1} \eta$, 其中 $\eta \in V, \bar{\eta} \in V^*$ 都是 Grassmann number 组成的向量.
\end{itemize}

\section{Grassmann path integral}
\begin{itemize}
	\item Dirac field 的 partition function 为,
	\begin{equation}
		Z(\eta, \bar{\eta}) = \int D\Psi D\bar{\Psi} \, e^{i \int d^4 x \, (\bar{\Psi} (i \cancel{\partial} - m + i \epsilon) \Psi + \bar{\eta} \Psi + \bar{\Psi} \eta)} = e^{i E_0 T} e^{- i \int \frac{d^4 p}{(2 \pi)^4} \tilde{\bar{\eta}}(- p) \frac{1}{\cancel{p} - m + i \epsilon} \tilde{\eta}(p)}
	\end{equation}
	其中 vacuum energy 为,
	\begin{equation}
		E_0 = - 4 V \int \frac{d^3 p}{(2 \pi)^3} \frac{1}{2} \omega_p + \text{irrelevant terms}
	\end{equation}
	
	\begin{tcolorbox}[title=calculation:]
		代入 \eqref{10.1.13},
		\begin{equation}
			Z(\eta, \bar{\eta}) = \det(\underbrace{i (i \cancel{\partial} - m + i \epsilon)}_{= i A}) e^{- i^2 (- i) \bar{\eta} A^{- 1} \eta}
		\end{equation}
		其中,
		\begin{align}
			& \begin{dcases}
				\det(i (i \cancel{\partial} - m + i \epsilon)) = \det(i \underbrace{\gamma^5 (i \cancel{\partial} - m + i \epsilon) \gamma^5}_{= (- i \cancel{\partial} - m + i \epsilon)}) \\
				(i \cancel{\partial} - m + i \epsilon) (- i \cancel{\partial} - m + i \epsilon) = (\partial^2 + m^2 - i \epsilon) I_{4 \times 4}
			\end{dcases} \notag \\
			\Longrightarrow & \det(i (i \cancel{\partial} - m + i \epsilon)) = \sqrt{\det((- \partial^2 - m^2 + i \epsilon) I_{4 \times 4})} = e^{i E_0 T}
		\end{align}
		注意到 $I_{4 \times 4}$ 会带来一个 $4$ 次方的系数.
		
		\noindent\rule[0.5ex]{\linewidth}{0.5pt} % horizontal line
		
		对于指数项, 考虑,
		\begin{equation}
			(i \cancel{\partial} - m + i \epsilon) \Psi(x) = \int d^4 y \, A(x - y) \Psi(y)
		\end{equation}
		其中,
		\begin{equation}
			A(x - y) = \int \frac{d^4 p}{(2 \pi)^4} (\cancel{p} - m + i \epsilon) e^{- i p \cdot (x - y)} \Longrightarrow A^{- 1}(x - y) = S(x - y)
		\end{equation}
		其中 $S(x - y)$ 是传播子, 见 \eqref{8.6.1}, 所以指数项为,
		\begin{equation}
			e^{- i \bar{\eta} A^{- 1} \eta} = e^{- i \int d^4 x d^4 y \, \bar{\eta}(x) S(x - y) \eta(y)} = \cdots
		\end{equation}
	\end{tcolorbox}
\end{itemize}

\section{Feynman rules}
\subsection{for Yukawa interaction}
\begin{itemize}
	\item 考虑如下 Lagrangian,
	\begin{equation}
		\mathcal{L} = \bar{\Psi} (i \cancel{\partial} - m) \Psi + \frac{1}{2} ((\partial \phi)^2 - \mu^2 \phi^2) - \frac{\lambda}{4!} \phi^4 + g \bar{\Psi} \phi \Psi
	\end{equation}
	对应如下 partition function,
	\begin{equation}
		Z(\bar{\eta}, \eta, J; \lambda, g) = Z(0; 0) e^{i \int d^4 x \, (- \frac{\lambda}{4!} (\frac{\delta}{\delta i J(x)})^4 + g \frac{\delta}{\delta i \eta(x)} \frac{\delta}{\delta i J(x)} \frac{\delta}{\delta i \bar{\eta}(x)})} e^{- \frac{i}{2} J D J - i \bar{\eta} S \eta}
	\end{equation}
	
	\item 
\end{itemize}

\subsection{for QED}
\begin{itemize}
	\item 考虑如下 Lagrangian,
	\begin{equation}
		\mathcal{L} = \bar{\Psi} (i (\cancel{\partial} + i e \cancel{A}) - m) \Psi - \frac{1}{4} F^{\mu \nu} F_{\mu \nu} + \frac{1}{2} \mu^2 A^\mu A_\mu
	\end{equation}
	对应如下 partition function,
	\begin{equation}
		Z(\bar{\eta}, \eta, J; \lambda, g) = Z(0; 0) e^{i \int d^4 x \, (- e \frac{\delta}{\delta i \eta(x)} \gamma^\mu \frac{\delta}{\delta i J^\mu(x)} \frac{\delta}{\delta i \bar{\eta}(x)})} e^{- \frac{i}{2} J_\mu D^{\mu \nu} J_\nu - i \bar{\eta} S \eta}
	\end{equation}
	
	\item 
\end{itemize}
