\chapter{quantizing the Dirac field}
\section{anticommutation} \label{8.1}
\begin{itemize}
	\item 用 $\alpha, \beta$ 表示电子的量子态 (包括动量和自旋), 那么
	\begin{equation}
		\{b_\alpha, b_\beta\} = 0 \quad \{b_\alpha, b^\dag_\beta\} = \delta_{\alpha \beta}.
	\end{equation}
	
	\begin{tcolorbox}[title=comment:]
		反对称关系 $\{b_\alpha, b_\beta\} = 0$ 由实验发现, 我们希望电子有 number operator,
		\begin{equation}
			N = \sum_\alpha b^\dag_\alpha b_\alpha, \quad \text{with} \quad \begin{dcases}
				[N, b_\alpha] = - b_\alpha \\
				[N, b^\dag_\alpha] = b^\dag_\alpha
			\end{dcases},
		\end{equation}
		考虑到 $[A B, C] = A B C - C A B = A \{B, C\} - \{A, C\} B$, 所以
		\begin{equation}
			\begin{dcases}
				[N, b_\alpha] = \sum_\beta (b^\dag_\beta \{b_\beta, b_\alpha\} - \{b^\dag_\beta, b_\alpha\} b_\beta) = - \sum_\beta \{b^\dag_\beta, b_\alpha\} b_\beta \\
				[N, b^\dag_\alpha] = \sum_\beta (b^\dag_\beta \{b_\beta, b^\dag_\alpha\} - \{b^\dag_\beta, b^\dag_\alpha\} b_\beta) = \sum_\beta b^\dag_\beta \{b_\beta, b^\dag_\alpha\}
			\end{dcases},
		\end{equation}
		可见 $\{b_\alpha, b^\dag_\beta\} = \delta_{\alpha \beta}$.
	\end{tcolorbox}
\end{itemize}

\section{plane wave solutions}
\begin{itemize}
	\item Dirac 方程的平面波解具有如下形式 (其中 $p^0 = \omega_p$),
	\begin{equation}
		\Psi = u(\vec{p}) e^{- i p \cdot x} \quad \text{and} \quad \Psi = v(\vec{p}) e^{i p \cdot x},
	\end{equation}
	代入 Dirac 方程, 得到
	\begin{equation}
		(\cancel{p} - m) u(\vec{p}) = 0 \quad \text{and} \quad (- \cancel{p} - m) v(\vec{p}) = 0,
	\end{equation}
	解为
	\begin{equation}
		u(\vec{p}) = \begin{pmatrix}
			\sqrt{p \cdot \sigma} \xi \\
			\sqrt{p \cdot \bar{\sigma}} \xi
		\end{pmatrix}, \quad v = \begin{pmatrix}
			- \sqrt{p \cdot \sigma} \chi \\
			\sqrt{p \cdot \bar{\sigma}} \chi
		\end{pmatrix},
	\end{equation}
	其中 $\xi, \chi$ 为任意 $2-\dim$ 列向量, 因此 $u(\vec{p}), v(\vec{p})$ 各有两个独立解, 分别用 $u(\vec{p}, s), v(\vec{p}, s), s = \pm 1$ 表示.
	
	\begin{tcolorbox}[title=proof:]
		令 $u^T = (u_1, u_2)$ 代入,
		\begin{equation}
			\begin{pmatrix}
				- m & p \cdot \sigma \\
				p \cdot \bar{\sigma} & - m
			\end{pmatrix} \begin{pmatrix}
				u_1 \\
				u_2
			\end{pmatrix} = 0 \Longrightarrow \begin{dcases}
				p \cdot \sigma u_2 = m u_1 \\
				p \cdot \bar{\sigma} u_1 = m u_2
			\end{dcases},
		\end{equation}
		注意到
		\begin{equation}
			(p \cdot \sigma) (p \cdot \bar{\sigma}) = \omega_p^2 - p^i p^j \sigma_i \sigma_j = \omega_p^2 - |\vec{p}|^2 = m^2,
		\end{equation}
		所以, 令 $u_2 = m \xi'$, 那么
		\begin{equation}
			u = \begin{pmatrix}
				p \cdot \sigma \xi' \\
				m \xi'
			\end{pmatrix} \Longrightarrow \xi = \sqrt{p \cdot \sigma} \xi' \Longrightarrow \cdots,
		\end{equation}
		其中, $\xi$ 可以任意选取, 并且注意到了 $[(p \cdot \sigma), (p \cdot \bar{\sigma})] = 0$, 因此
		\begin{equation}
			\sqrt{p \cdot \sigma} \sqrt{p \cdot \bar{\sigma}} = \sqrt{(p \cdot \sigma) (p \cdot \bar{\sigma})} = m.
		\end{equation}
		
		\noindent\hdashrule[0.5ex]{\linewidth}{0.5pt}{1mm} % horizontal dashed line
		
		类似地, 对于 $v^T = (v_1, v_2)$, 代入,
		\begin{equation}
			\begin{pmatrix}
				m & p \cdot \sigma \\
				p \cdot \bar{\sigma} & m
			\end{pmatrix} \begin{pmatrix}
				v_1 \\
				v_2
			\end{pmatrix} = 0 \Longrightarrow \begin{dcases}
				p \cdot \sigma v_2 = - m v_1 \\
				p \cdot \bar{\sigma} v_1 = - m v_2
			\end{dcases},
		\end{equation}
		令 $v_1 = - \sqrt{p \cdot \sigma} \chi$, 那么...
		
		\noindent\rule[0.5ex]{\linewidth}{0.5pt} % horizontal line
		
		最后,
		\begin{equation}
			\begin{dcases}
				\sqrt{p \cdot \sigma} = \sqrt{\frac{m + \omega_p}{2}} I + \frac{1}{\sqrt{2 (m + \omega_p)}} \vec{p} \cdot \vec{\sigma} \\
				\sqrt{p \cdot \bar{\sigma}} = \sqrt{\frac{m + \omega_p}{2}} I - \frac{1}{\sqrt{2 (m + \omega_p)}} \vec{p} \cdot \vec{\sigma}
			\end{dcases},
		\end{equation}
		以及一些有用的公式,
		\begin{equation}
			\begin{dcases}
				\sqrt{p \cdot \sigma} \sigma^\mu \sqrt{p \cdot \sigma} = \begin{cases}
					\omega_p + \vec{p} \cdot \vec{\sigma} & \mu = 0 \\
					\omega_p \sigma^i + p^i + \frac{\vec{p} \cdot \vec{\sigma}}{2 (m + \omega_p)} 2 i \epsilon_{i j k} p^j \sigma^k & \mu = i
				\end{cases} \\
				\sqrt{p \cdot \bar{\sigma}} \sigma^\mu \sqrt{p \cdot \bar{\sigma}} = \begin{cases}
					\omega_p - \vec{p} \cdot \vec{\sigma} & \mu = 0 \\
					\omega_p \sigma^i - p^i + \frac{\vec{p} \cdot \vec{\sigma}}{2 (m + \omega_p)} 2 i \epsilon_{i j k} p^j \sigma^k & \mu = i
				\end{cases} \\
				\sqrt{p \cdot \sigma} \sigma^\mu \sqrt{p \cdot \bar{\sigma}} = \begin{cases}
					m & \mu = 0 \\
					m \sigma^i - \frac{\sqrt{p \cdot \sigma}}{\sqrt{2 (m + \omega_p)}} 2 i \epsilon_{i j k} p^j \sigma_k & \mu = i
				\end{cases} \\
				\sqrt{p \cdot \bar{\sigma}} \sigma^\mu \sqrt{p \cdot \sigma} = \begin{cases}
					m & \mu = 0 \\
					m \sigma^i + \frac{\sqrt{p \cdot \bar{\sigma}}}{\sqrt{2 (m + \omega_p)}} 2 i \epsilon_{i j k} p^j \sigma_k & \mu = i
				\end{cases}
			\end{dcases},
		\end{equation}
		另外 $(- p) \cdot \sigma = p \cdot \bar{\sigma}, (- p) \cdot \bar{\sigma} = p \cdot \sigma$, 其中 $(- p) = (\omega_p, - \vec{p})$.
	\end{tcolorbox}
	
	\item 选择归一化条件
	\begin{equation}
		\begin{dcases}
			\bar{u}(\vec{p}, s) u(\vec{p}, s') = 2 m \delta_{s s'} \\
			\bar{v}(\vec{p}, s) v(\vec{p}, s') = - 2 m \delta_{s s'}
		\end{dcases} \quad \text{and} \quad \bar{u}(\vec{p}, s) v(\vec{p}, s') = 0,
	\end{equation}
	其中 $\bar{u} = u^\dag \gamma^0, \bar{v} = v^\dag \gamma^0$, 那么
	\begin{equation}
		\begin{dcases}
			\xi^{s \dag} \xi^{s'} = \delta_{s s'} \\
			\chi^{s \dag} \chi^{s'} = \delta_{s s'}
		\end{dcases} \quad \text{and} \quad \xi^{s \dag} \chi^{s'} - \chi^{s \dag} \xi^{s'} = 0,
	\end{equation}
	可以选取
	\begin{equation}
		\xi^{+ 1} = \chi^{+ 1} = (1, 0)^T, \quad \xi^{- 1} = \chi^{- 1} = (0, 1)^T.
	\end{equation}
	\begin{itemize}
		\item 在粒子静止系下, $p_r = (m, 0, 0, 0)$,
		\begin{equation}
			\frac{u(\vec{p}_r, + 1)}{\sqrt{m}} = \begin{pmatrix}
				1 \\
				0 \\
				1 \\
				0
			\end{pmatrix}, \quad \frac{u(\vec{p}_r, - 1)}{\sqrt{m}} = \begin{pmatrix}
				0 \\
				1 \\
				0 \\
				1
			\end{pmatrix}, \quad \frac{v(\vec{p}_r, + 1)}{\sqrt{m}} = \begin{pmatrix}
				- 1 \\
				0 \\
				1 \\
				0
			\end{pmatrix}, \quad \frac{v(\vec{p}_r, - 1)}{\sqrt{m}} = \begin{pmatrix}
				0 \\
				- 1 \\
				0 \\
				1
			\end{pmatrix},
		\end{equation}
		可见 $s = \pm 1$ 分别代表 spin-up 和 spin-down.
		
		\item 另外, 我们注意到 (对 $v$ 同样适用)
		\begin{equation}
			\begin{pmatrix}
				\omega_p \\
				\vec{p}
			\end{pmatrix} = e^{\lambda J^{0 1}} \begin{pmatrix}
				m \\
				0
			\end{pmatrix} \iff u(\vec{p}, s) = e^{\frac{1}{2} \lambda \sigma^{0 1}} u(\vec{p}_r, s), \quad \text{with} \quad \frac{p_1}{m} = \sinh \lambda, p_2 = p_3 = 0.
		\end{equation}
	\end{itemize}
	
	\item 最后,
	\begin{equation}
		\begin{dcases}
			\sum_{s = \pm 1} u(\vec{p}, s) \bar{u}(\vec{p}, s) = \cancel{p} + m \\
			\sum_{s = \pm 1} v(\vec{p}, s) \bar{v}(\vec{p}, s) = \cancel{p} - m
		\end{dcases}.
	\end{equation}
	
	\begin{tcolorbox}[title=calculation:]
		首先,
		\begin{equation}
			u(\vec{p}, s) u^\dag(\vec{p}, s) = \begin{pmatrix}
				\sqrt{p \cdot \sigma} \xi^s \\
				\sqrt{p \cdot \bar{\sigma}} \xi^s
			\end{pmatrix} \begin{pmatrix}
				\xi^{s \dag} \sqrt{p \cdot \sigma} & \xi^{s \dag} \sqrt{p \cdot \bar{\sigma}}
			\end{pmatrix},
		\end{equation}
		注意到
		\begin{equation}
			\sum_{s = \pm 1} \xi^{s} \xi^{s \dag} = I_{2 \times 2},
		\end{equation}
		代入,
		\begin{equation}
			\sum_{s = \pm 1} u(\vec{p}, s) u^\dag(\vec{p}, s) = \begin{pmatrix}
				p \cdot \sigma & m \\
				m & p \cdot \bar{\sigma}
			\end{pmatrix} = (\cancel{p} + m) \gamma^0.
		\end{equation}
		
		\noindent\rule[0.5ex]{\linewidth}{0.5pt} % horizontal line
		
		类似地,
		\begin{align}
			\sum_{s = \pm 1} v(\vec{p}, s) v^\dag(\vec{p}, s) &= \sum_{s = \pm 1} \begin{pmatrix}
				\sqrt{p \cdot \sigma} \chi^s \\
				- \sqrt{p \cdot \bar{\sigma}} \chi^s
			\end{pmatrix} \begin{pmatrix}
				\chi^{s \dag} \sqrt{p \cdot \sigma} & - \chi^{s \dag} \sqrt{p \cdot \bar{\sigma}}
			\end{pmatrix} \notag \\
			&= \begin{pmatrix}
				p \cdot \sigma & - m \\
				- m & p \cdot \bar{\sigma}
			\end{pmatrix} = (\cancel{p} - m) \gamma^0.
		\end{align}
	\end{tcolorbox}
\end{itemize}

\section{the Dirac field}
\begin{itemize}
	\item $\Psi(x), \bar{\Psi}$ 有如下形式,
	\begin{equation} \label{8.3.1}
		\begin{dcases}
			\Psi(x) = \sum_{s = \pm 1} \int \frac{d^3 p}{(2 \pi)^{3 / 2} \sqrt{2 \omega_p}} (b^s_{\vec{p}} u(\vec{p}, s) e^{- i p \cdot x} + c^{s \dag}_{\vec{p}} v(\vec{p}, s) e^{i p \cdot x}) \\
			\bar{\Psi}(x) = \sum_{s = \pm 1} \int \frac{d^3 p}{(2 \pi)^{3 / 2} \sqrt{2 \omega_p}} (b^{s \dag}_{\vec{p}} \bar{u}(\vec{p}, s) e^{i p \cdot x} + c^s_{\vec{p}} \bar{v}(\vec{p}, s) e^{- i p \cdot x})
		\end{dcases}.
	\end{equation}
	
	\item 回顾 section \ref{4.4} 关于 complex scalar field 的内容, 可知 $b^\dag$ 和 $c^\dag$ 产生的粒子具有相反的电荷, 不妨令 $b^\dag$ 产生 electron (带电荷 $- e$), $c^\dag$ 产生 position (带电荷 $e$).
	
	\item section \ref{8.1} 中的讨论说明
	\begin{equation} \label{8.3.2}
		\begin{dcases}
			\{b^s_{\vec{p}}, b^{s'}_{\vec{p}'}\} = 0 \\
			\{b^s_{\vec{p}}, b^{s' \dag}_{\vec{p}'}\} = \delta^{(3)}(\vec{p} - \vec{p}') \delta_{s s'}
		\end{dcases}.
	\end{equation}
	
	\noindent\rule[0.5ex]{\linewidth}{0.5pt} % horizontal line
	
	\item $\Psi$ 的 momentum conjecture 为 ($\pi_\Psi^\mu$ 见 \eqref{7.3.4})
	\begin{equation}
		\pi_\Psi = \frac{\delta \mathcal{L}}{\delta \partial_0 \Psi} = \pi_\Psi^0 = \bar{\Psi} i \gamma^0 = i \Psi^\dag,
	\end{equation}
	存在如下 anticommutation relation,
	\begin{equation}
		\{\Psi_\alpha(t, \vec{x}), i \Psi^\dag_\beta(t, \vec{y})\} = i \delta^{(3)}(\vec{x} - \vec{y}) \delta_{\alpha \beta}.
	\end{equation}
	
	\begin{tcolorbox}[title=calculation:]
		代入 \eqref{8.3.2}, (下式中 $x = (t, \vec{x}), y = (t, \vec{y})$, 另外注意到 $u \bar{u} = u u^\dag \gamma^0$),
		\begin{align}
			\{\Psi_\alpha(t, \vec{x}), \Psi^\dag_\beta(t, \vec{y})\} =& \sum_{s = \pm} \int \frac{d^3 p_1 d^3 p_2}{(2 \pi)^3 \sqrt{4 \omega_{p_1} \omega_{p_2}}} \Big( \{b^s_{\vec{p}_1}, b^{s \dag}_{\vec{p}_2}\} u(\vec{p}_1, s) u^\dag(\vec{p}_2, s) e^{i (- p_1 \cdot x + p_2 \cdot y)} \notag \\
			& + \{c^{s \dag}_{\vec{p}_1}, c^s_{\vec{p}_2}\} v(\vec{p}_1, s) v^\dag(\vec{p}_2, s) e^{i (p_1 \cdot x - p_2 \cdot y)} \Big) \notag \\
			=& \sum_{s = \pm} \int \frac{d^3 p}{(2 \pi)^3 2 \omega_p} \Big( u(\vec{p}, s) u^\dag(\vec{p}, s) e^{i p \cdot (- x + y)} + v(\vec{p}, s) v^\dag(\vec{p}, s) e^{i p \cdot (x - y)} \Big) \notag \\
			=& \int \frac{d^3 p}{(2 \pi)^3 2 \omega_p} \Big( (\cancel{p} + m) \gamma^0 e^{i \vec{p} \cdot (\vec{x} - \vec{y})} + (\cancel{p} - m) \gamma^0 e^{- i \vec{p} \cdot (\vec{x} - \vec{y})} \Big) \notag \\
			=& \int \frac{d^3 p}{(2 \pi)^3 2 \omega_p} \Big( 2 \omega_p I \cos(\vec{p} \cdot (\vec{x} - \vec{y})) - 2 p^i \gamma^i \gamma^0 \cos(\vec{p} \cdot (\vec{x} - \vec{y})) \notag \\
			& + 2 i m \gamma^0 \sin(\vec{p} \cdot (\vec{x} - \vec{y})) \Big),
		\end{align}
		注意, 只有第一项是偶函数, 积分后不为零,
		\begin{align}
			\{\Psi_\alpha(t, \vec{x}), \Psi^\dag_\beta(t, \vec{y})\} &= \int \frac{d^3 p}{(2 \pi)^3} I \cos(\vec{p} \cdot (\vec{x} - \vec{y})) \notag \\
			&= \int \frac{d^3 p}{(2 \pi)^3} e^{i \vec{p} \cdot (\vec{x} - \vec{y})} = \delta^{(3)}(\vec{x} - \vec{y}) I.
		\end{align}
	\end{tcolorbox}
	
	\item 另外, 显然有
	\begin{equation}
		\{\Psi(x), \Psi(y)\} = \{\Psi^\dag(x), \Psi^\dag(y)\} = 0.
	\end{equation}
\end{itemize}

\section{Hamiltonian, energy-momentum tensor and angular momentum}
\subsection{Hamiltonian}
\begin{itemize}
	\item 计算 Hamiltonian,
	\begin{equation}
		H = \sum_{s = \pm 1} \int d^3 p \, \omega_p (b^{s \dag}_{\vec{p}} b^s_{\vec{p}} - c^s_{\vec{p}} c^{s \dag}_{\vec{p}}) = \sum_{s = \pm 1} \int d^3 p \, \omega_p (b^{s \dag}_{\vec{p}} b^s_{\vec{p}} + c^{s \dag}_{\vec{p}} c^s_{\vec{p}}) + E_0,
	\end{equation}
	其中 vacuum energy,
	\begin{equation} \label{8.4.2}
		E_0 = - 2 \delta^{(3)}(0) \int d^3 p \, \omega_p = - 2 V \int \frac{d^3 p}{(2 \pi)^3} \omega_p,
	\end{equation}
	的符号与标量场的正好相反.
	
	\begin{tcolorbox}[title=calculation:]
		the Hamiltonian density is
		\begin{align}
			\mathcal{H} =& i \Psi^\dag \partial_0 \Psi - \mathcal{L} = - \bar{\Psi} (i \gamma^i \partial_i - m) \Psi \notag \\
			=& \sum_{s_1, s_2 = \pm 1} \int \frac{d^3 p_1 d^3 p_2}{(2 \pi)^3 \sqrt{4 \omega_{p_1} \omega_{p_2}}} (b^{s_1 \dag}_{\vec{p}_1} \bar{u}(\vec{p}_1, s_1) e^{i p_1 \cdot x} + c^{s_1}_{\vec{p}_1} \bar{v}(\vec{p}_1, s_1) e^{- i p_1 \cdot x}) \notag \\
			& (\underbrace{(\gamma^i p_2^i + m)}_{\mapsto \omega_{p_2} \gamma^0} b^{s_2}_{\vec{p}_2} u(\vec{p}_2, s_2) e^{- i p_2 \cdot x} + \underbrace{(- \gamma^i p_2^i + m)}_{\mapsto - \omega_{p_2} \gamma^0} c^{s_2 \dag}_{\vec{p}_2} v(\vec{p}_2, s_2) e^{i p_2 \cdot x}),
		\end{align}
		代入,
		\begin{align}
			H = \int d^3 x \, \mathcal{H} =& \sum_{s_1, s_2 = \pm 1} \int \frac{d^3 p}{2 \omega_p} \Big( b^{s_1 \dag}_{\vec{p}} \bar{u}(\vec{p}, s_1) \omega_p \gamma^0 b^{s_2}_{\vec{p}} u(\vec{p}, s_2) \notag \\
			& - b^{s_1 \dag}_{\vec{p}} \bar{u}(\vec{p}, s_1) \omega_p \gamma^0 c^{s_2 \dag}_{- \vec{p}} v(- \vec{p}, s_2) e^{2 i \omega_p t} \notag \\
			& + c^{s_1}_{\vec{p}} \bar{v}(\vec{p}, s_1) \omega_p \gamma^0 b^{s_2}_{- \vec{p}} u(- \vec{p}, s_2) e^{- 2 i \omega_p t} \notag \\
			& - c^{s_1}_{\vec{p}} \bar{v}(\vec{p}, s_1) \omega_p \gamma^0 c^{s_2 \dag}_{\vec{p}} v(\vec{p}, s_2) \Big),
		\end{align}
		注意到
		\begin{equation}
			\begin{dcases}
				u^\dag(\vec{p}, s_1) u(\vec{p}, s_2) = 2 \omega_p \delta_{s_1 s_2} \\
				u^\dag(\vec{p}, s_1) v(- \vec{p}, s_2) = 0 \\
				v^\dag(\vec{p}, s_1) u(- \vec{p}, s_2) = 0 \\
				v^\dag(\vec{p}, s_1) v(\vec{p}, s_2) = 2 \omega_p \delta_{s_1 s_2}
			\end{dcases},
		\end{equation}
		代入,
		\begin{equation}
			H = \sum_{s_1, s_2 = \pm 1} \int \frac{d^3 p}{2 \omega_p} \Big( b^{s_1 \dag}_{\vec{p}} b^{s_2}_{\vec{p}} (2 \omega_p^2) \delta_{s_1 s_2} + c^{s_1}_{\vec{p}} c^{s_2 \dag}_{\vec{p}} (- 2 \omega_p^2) \delta_{s_1 s_2} \Big) = \cdots
		\end{equation}
	\end{tcolorbox}
\end{itemize}

\subsection{energy-momentum tensor}
\begin{itemize}
	\item Dirac field 的动量算符为
	\begin{equation}
		P^\mu = \int d^3 x \, T^{0 \mu} = \int d^3 p \, p^\mu (b^{s \dag}_{\vec{p}} b^s_{\vec{p}} + c^{s \dag}_{\vec{p}} c^s_{\vec{p}}),
	\end{equation}
	另外 $P^0 = H$ 还有一个 vacuum energy.
	
	\begin{tcolorbox}[title=calculation:]
		energy-momentum tensor 的 $0, \mu$ 分量为 (见 \eqref{7.4.1})
		\begin{align}
			T^{0 \mu} =& i \bar{\Psi} \gamma^0 \partial^\mu \Psi = i \Psi^\dag \partial^\mu \Psi \notag \\
			=& \sum_{s_1, s_2 = \pm 1} \int \frac{d^3 p_1 d^3 p_2}{(2 \pi)^3 \sqrt{4 \omega_{p_1} \omega_{p_2}}} (b^{s_1 \dag}_{\vec{p}_1} u^\dag(\vec{p}_1, s_1) e^{i p_1 \cdot x} + c^{s_1}_{\vec{p}_1} v^\dag(\vec{p}_1, s_1) e^{- i p_1 \cdot x}) \notag \\
			& p_2^\mu (b^{s_2}_{\vec{p}_2} u(\vec{p}_2, s_2) e^{- i p_2 \cdot x} - c^{s_2 \dag}_{\vec{p}_2} v(\vec{p}_2, s_2) e^{i p_2 \cdot x}),
		\end{align}
		代入,
		\begin{align}
			P^\mu =& \sum_{s_1, s_2 = \pm 1} \int \frac{d^3 p}{2 \omega_p} \Big( p^\mu b^{s_1 \dag}_{\vec{p}} u^\dag(\vec{p}, s_1) b^{s_2}_{\vec{p}} u(\vec{p}, s_2) - (- p^\mu) b^{s_1 \dag}_{\vec{p}} u^\dag(\vec{p}, s_1) c^{s_2 \dag}_{- \vec{p}} v(- \vec{p}, s_2) e^{2 i \omega_p t} \notag \\
			& + (- p^\mu) c^{s_1}_{\vec{p}} v^\dag(\vec{p}, s_1) b^{s_2}_{- \vec{p}} u(- \vec{p}, s_2) - p^\mu c^{s_1}_{\vec{p}} v^\dag(\vec{p}, s_1) c^{s_2 \dag}_{\vec{p}} v(\vec{p}, s_2) \Big) \notag \\
			=& \sum_{s_1, s_2 = \pm 1} \int \frac{d^3 p}{2 \omega_p} \Big( p^\mu b^{s_1 \dag}_{\vec{p}} b^{s_2}_{\vec{p}} (2 \omega_p \delta_{s_1 s_2}) - p^\mu c^{s_1}_{\vec{p}} c^{s_2 \dag}_{\vec{p}} (2 \omega_p \delta_{s_1 s_2}) \Big) \notag \\
			=& \int d^3 p \, p^\mu (b^{s \dag}_{\vec{p}} b^s_{\vec{p}} - c^s_{\vec{p}} c^{s \dag}_{\vec{p}}).
		\end{align}
	\end{tcolorbox}
\end{itemize}

\subsection{angular momentum}
\begin{itemize}
	\item Dirac field 的角动量算符为 (这部分在 Peskin 上有)
	\begin{align}
		J^{i j} &= \int d^3 \, M^{0 i j} \notag \\
		&= \epsilon^{i j k} \sum_{s_1, s_2 = \pm 1} \int d^3 p \, \frac{m}{2 \omega_p} (b^{s_1 \dag}_{\vec{p}} b^{s_2}_{\vec{p}} + c^{s_1}_{\vec{p}} c^{s_2 \dag}_{\vec{p}}) \xi^{s_1 \dag} \sigma_k \xi^{s_2} + \int d^3 x \, (x^i T^{0 j} - x^j T^{0 i}),
	\end{align}
	其中, $M^{\mu \nu \rho}$ 见 \eqref{7.4.2}.
	\begin{itemize}
		\item 把角动量算符中 spin 的部分表示为 $S^{i j}$, 那么
		\begin{equation}
			\begin{dcases}
				S^{1 2} b^{s \dag}_{\vec{p}} \ket{0} = s \frac{m}{2 \omega_p} b^{s \dag}_{\vec{p}} \ket{0} \\
				S^{1 2} c^{s \dag}_{\vec{p}} \ket{0} = - s \frac{m}{2 \omega_p} c^{s \dag}_{\vec{p}} \ket{0}
			\end{dcases}.
		\end{equation}
	\end{itemize}
	
	\begin{tcolorbox}[title=calculation:]
		角动量张量为
		\begin{align}
			M^{0 \mu \nu} =& \frac{i}{2} \underbrace{\bar{\Psi} \gamma^0}_{\Psi^\dag} \sigma^{\mu \nu} \Psi + (x^\mu T^{0 \nu} - x^\nu T^{0 \mu}) \notag \\
			=& \frac{i}{2} \sum_{s_1, s_2 = \pm 1} \int \frac{d^3 p_1 d^3 p_2}{(2 \pi)^3 \sqrt{4 \omega_{p_1} \omega_{p_2}}} (b^{s_1 \dag}_{\vec{p}_1} u^\dag(\vec{p}_1, s_1) e^{i p_1 \cdot x} + c^{s_1}_{\vec{p}_1} v^\dag(\vec{p}_1, s_1) e^{- i p_1 \cdot x}) \notag \\
			& \sigma^{\mu \nu} (b^{s_2}_{\vec{p}_2} u(\vec{p}_2, s_2) e^{- i p_2 \cdot x} + c^{s_2 \dag}_{\vec{p}_2} v(\vec{p}_2, s_2) e^{i p_2 \cdot x}) + (x^\mu T^{0 \nu} - x^\nu T^{0 \mu}),
		\end{align}
		代入,
		\begin{align}
			J^{\mu \nu} - \int d^3 x \, (x^\mu T^{0 \nu} - x^\nu T^{0 \mu}) =& \frac{i}{2} \sum_{s_1, s_2 = \pm 1} \int \frac{d^3 p}{2 \omega_p} \Big( b^{s_1 \dag}_{\vec{p}} u^\dag(\vec{p}, s_1) \sigma^{\mu \nu} b^{s_2}_{\vec{p}} u(\vec{p}, s_2) \notag \\
			& + b^{s_1 \dag}_{\vec{p}} u^\dag(\vec{p}, s_1) \sigma^{\mu \nu} c^{s_2 \dag}_{- \vec{p}} v(- \vec{p}, s_2) e^{2 i \omega_p t} \notag \\
			& + c^{s_1}_{\vec{p}} v^\dag(\vec{p}, s_1) \sigma^{\mu \nu} b^{s_2}_{- \vec{p}} u(- \vec{p}, s_2) e^{- 2 i \omega_p t} \notag \\
			& + c^{s_1}_{\vec{p}} v^\dag(\vec{p}, s_1) \sigma^{\mu \nu} c^{s_2 \dag}_{\vec{p}} v(\vec{p}, s_2) \Big),
		\end{align}
		其中
		\begin{equation}
			\begin{dcases}
				u^\dag(\vec{p}, s_1) \sigma^{i j} u(\vec{p}, s_2) = - 2 i \epsilon^{i j k} m \xi^{s_1 \dag} \sigma_k \xi^{s_2} \\
				u^\dag(\vec{p}, s_1) \sigma^{i j} v(- \vec{p}, s_2) = 0 \\
				v^\dag(\vec{p}, s_1) \sigma^{i j} u(- \vec{p}, s_2) = 0 \\
				v^\dag(\vec{p}, s_1) \sigma^{i j} v(\vec{p}, s_2) = - 2 i \epsilon^{i j k} m \chi^{s_1 \dag} \sigma_k \chi^{s_2}
			\end{dcases},
		\end{equation}
		代入 (注意到 $\xi^s = \chi^s$),
		\begin{equation}
			J^{i j} - \int d^3 x \, (x^i T^{0 j} - x^j T^{0 i}) = \epsilon^{i j k} \sum_{s_1, s_2 = \pm 1} \int \frac{d^3 p}{2 \omega_p} m (b^{s_1 \dag}_{\vec{p}} b^{s_2}_{\vec{p}} + c^{s_1}_{\vec{p}} c^{s_2 \dag}_{\vec{p}}) \xi^{s_1 \dag} \sigma_k \xi^{s_2}.
		\end{equation}
	\end{tcolorbox}
\end{itemize}

\section{electric current}
\begin{itemize}
	\item internal vector symmetry 对应的守恒流就是电流, 见 subsection \ref{subsection 7.3.1}, 有
	\begin{equation}
		Q = \int d^3 x \, J_V^0 = \sum_{s = \pm 1} \int d^3 p \, (b^{s \dag}_{\vec{p}} b^s_{\vec{p}} - c^{s \dag}_{\vec{p}} c^s_{\vec{p}}) - 2 \delta^{(3)}(0) \int d^3 p.
	\end{equation}
	
	\begin{tcolorbox}[title=calculation:]
		首先,
		\begin{align}
			\int d^3 x \, J_V^\mu = \int d^3 x \, \bar{\Psi} \gamma^\mu \Psi =& \sum_{s_1, s_2 = \pm 1} \int \frac{d^3 p}{2 \omega_p} \Big( b^{s_1 \dag}_{\vec{p}} b^{s_2}_{\vec{p}} \bar{u}(\vec{p}, s_1) \gamma^\mu u(\vec{p}, s_2) \notag \\
			& + b^{s_1 \dag}_{\vec{p}} c^{s_2 \dag}_{- \vec{p}} \bar{u}(\vec{p}, s_1) \gamma^\mu v(- \vec{p}, s_2) e^{2 i \omega_p t} \notag \\
			& + c^{s_1}_{\vec{p}} b^{s_2}_{- \vec{p}} \bar{v}(\vec{p}, s_1) \gamma^\mu u(- \vec{p}, s_2) e^{- 2 i \omega_p t} \notag \\
			& + c^{s_1}_{\vec{p}} c^{s_2 \dag}_{\vec{p}} \bar{v}(\vec{p}, s_1) \gamma^\mu v(\vec{p}, s_2) \Big),
		\end{align}
		其中
		\begin{equation}
			\begin{dcases}
				\bar{u}(\vec{p}, s_1) \gamma^\mu u(\vec{p}, s_2) = 2 p_\mu \delta_{s_1 s_2} \quad \textcolor{red}{(?)} \\
				\bar{u}(\vec{p}, s_1) \gamma^0 v(- \vec{p}, s_2) = 0 \\
				\bar{u}(\vec{p}, s_1) \gamma^i v(- \vec{p}, s_2) = \xi^{s_1 \dag} (2 m \sigma^i) \xi^{s_2} \\
				\bar{v}(\vec{p}, s_1) \gamma^\mu u(- \vec{p}, s_2) = \bar{u}(\vec{p}, s_1) \gamma^\mu v(- \vec{p}, s_2) \\
				\bar{v}(\vec{p}, s_1) \gamma^\mu v(\vec{p}, s_2) = \bar{u}(\vec{p}, s_1) \gamma^\mu u(\vec{p}, s_2)
			\end{dcases},
		\end{equation}
		代入,
		\begin{align}
			Q =& \sum_{s = \pm 1} \int d^3 p \, (b^{s \dag}_{\vec{p}} b^s_{\vec{p}} + c^s_{\vec{p}} c^{s \dag}_{\vec{p}}), \\
			J^i = \int d^3 x \, J_V^i \overset{\textcolor{red}{(?)}}{=}& \int d^3 p \, \Big( \sum_{s = \pm 1} - \frac{p^i}{\omega_p} (b^{s \dag}_{\vec{p}} b^s_{\vec{p}} + c^s_{\vec{p}} c^{s \dag}_{\vec{p}}) \notag \\
			& + \sum_{s_1, s_2 = \pm 1} (b^{s_1 \dag}_{\vec{p}} c^{s_2 \dag}_{- \vec{p}} e^{2 i \omega_p t} + c^{s_1}_{\vec{p}} b^{s_2}_{- \vec{p}} e^{- 2 i \omega_p t}) \xi^{s_1 \dag} (2 m \sigma^i) \xi^{s_2} \Big).
		\end{align}
	\end{tcolorbox}
\end{itemize}

\section{free propagator}
\begin{itemize}
	\item 参考 scalar field 中的 propagator (见 \eqref{4.1.17}), the propagator of the Dirac field is
	\begin{align}
		i S(x - y) &= \braket{0 | T \Psi(x) \bar{\Psi}(y) | 0} \notag \\
		&= \int \frac{d^3 p}{(2 \pi)^3 2 \omega_p} \Big( \theta(x^0 - y^0) (\cancel{p} + m) e^{- i p \cdot (x - y)} - \theta(y^0 - x^0) (\cancel{p} - m) e^{- i p \cdot (y - x)} \Big) \notag \\
		&= \int \frac{d^4 p}{(2 \pi)^4} \frac{i}{\cancel{p} - m + i \epsilon} e^{- i p \cdot (x - y)}, \label{8.6.1}
	\end{align}
	其中
	\begin{equation}
		(T \Psi(x) \bar{\Psi}(y))_{\alpha \beta} = \theta(x^0 - y^0) \Psi_\alpha(x) \bar{\Psi}_\beta(y) \mathcolor{red}{-} \theta(y^0 - x^0) \bar{\Psi}_{\mathcolor{red}{\beta}}(y) \Psi_{\mathcolor{red}{\alpha}}(x),
	\end{equation}
	注意到这里交换 $\Psi, \bar{\Psi}$ 是产生湮灭算符层面上的, 不是 spinor 层面上的.
	
	\begin{tcolorbox}[title=calculation:]
		分别计算 $\braket{0 | \Psi_\alpha(x) \bar{\Psi}_\beta(y) | 0}$ 和 $\braket{0 | \bar{\Psi}_\beta(y) \Psi_\alpha(x) | 0}$,
		\begin{align}
			\braket{0 | \Psi_\alpha(x) \bar{\Psi}_\beta(y) | 0} &= \sum_{s = \pm 1} \int \frac{d^3 p}{(2 \pi)^3 2 \omega_p} u_\alpha(\vec{p}, s) \bar{u}_\beta(\vec{p}, s) e^{- i p \cdot (x - y)} \notag \\
			&= \int \frac{d^3 p}{(2 \pi)^3 2 \omega_p} (\cancel{p} + m)_{\alpha \beta} e^{- i p \cdot (x - y)}, \\
			\braket{0 | \bar{\Psi}_\beta(y) \Psi_\alpha(x) | 0} &= \sum_{s = \pm 1} \int \frac{d^3 p}{(2 \pi)^3 2 \omega_p} v_\alpha(\vec{p}, s) \bar{v}_\beta(\vec{p}, s) e^{- i p \cdot (y - x)},
		\end{align}
		代入, 得到...
		
		\noindent\rule[0.5ex]{\linewidth}{0.5pt} % horizontal line
		
		把 $i S(x)$ 的第二项中的 $\vec{p}$ 变成 $- \vec{p}$,
		\begin{align}
			i S(x) &= \int \frac{d^3 p}{(2 \pi)^3 2 \omega_p} \Big( \theta(t) (\omega_p \gamma^0 - p^i \gamma^i + m) e^{- i p \cdot x} - \theta(- t) (\omega_p \gamma^0 + p^i \gamma^i - m) e^{i (\omega_p t + \vec{p} \cdot \vec{x})} \Big) \notag \\
			&= \int \frac{d^3 p}{(2 \pi)^3 2 \omega_p} \Big( \theta(t) (\omega_p \gamma^0 - p^i \gamma^i + m) e^{- i p \cdot x} + \theta(- t) (\omega_p \mapsto - \omega_p) \Big) \notag \\
			&= \int \frac{dp^0}{- 2 \pi i} \frac{1}{(p^0 - (\omega_p - i \epsilon)) (p^0 + (\omega_p - i \epsilon))} \int \frac{d^3 p}{(2 \pi)^3} (\cancel{p} + m) e^{- i p \cdot x} = \cdots,
		\end{align}
		最后,
		\begin{equation}
			\frac{\cancel{p} + m}{p^2 - m^2 + i \epsilon} (\cancel{p} - m + i \epsilon) = I.
		\end{equation}
	\end{tcolorbox}
\end{itemize}
