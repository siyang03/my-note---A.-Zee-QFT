\chapter{antiunitary operator and time reversal} \label{E}
\section{complex conjugation operator}
\begin{itemize}
	\item complex conjugation operator, $K$, is an antiunitary operator on the complex plane,
	\begin{equation}
		\begin{dcases}
			K z = z^* \\
			z K^* = z^*
		\end{dcases} \Longrightarrow K^2 = K^{* 2} = 1
	\end{equation}
	
	\item $K^* I : V^* \rightarrow V^*$ 是 dual space 上的算符.
	
	\item 对于一组 orthonormal basis, 有,
	\begin{equation}
		\braket{i | K^* I K | j} = \delta_{i j}
	\end{equation}
	并且可以证明在基矢变换后这个等式依然成立.
	
	\begin{tcolorbox}[title=proof:]
		\begin{itemize}
			\item 对基矢做 unitary transformation,
			\begin{equation}
				\ket{i'} = U \ket{i} = \sum_j \ket{j} U_{j i} \quad \text{where} \quad U_{j i} = \braket{j | U | i}
			\end{equation}
			那么,
			\begin{equation}
				\braket{i' | K^* I K | j'} = \sum_{k l} \braket{k | U^*_{k i} K^* I K U_{l j} | l} = \sum_{k l} U_{k i} U^*_{l j} \delta_{k l} = \delta_{i j}
			\end{equation}
			
			\item 对基矢做 antiunitary transformation, 只需要证明 $\ket{i'} = K \ket{i}$ 的情况, 此时,
			\begin{equation}
				\braket{i' | K^* I K | j'} = \braket{i | j} = \delta_{i j}
			\end{equation}
		\end{itemize}
	\end{tcolorbox}
\end{itemize}

\section{antiunitary operator}
\begin{itemize}
	\item 对于一个 unitary operator, $U$, $\Omega = U K$ 是一个 antiunitary operator.
	
	\item 定义其 Hermitian conjugate,
	\begin{equation}
		\Omega^\dag = K^* U^\dag \iff \braket{i | \Omega j} = \braket{j | \Omega^\dag i}^*
	\end{equation}
	那么,
	\begin{equation}
		\begin{dcases}
			\braket{\phi | \Omega \psi} = \braket{\psi | \Omega^\dag \phi}^* \\
			\braket{\Omega \phi | \Omega \psi} = \braket{\psi | \phi}
		\end{dcases}
	\end{equation}
	
	\begin{tcolorbox}[title=proof:]
		首先,
		\begin{align}
			\braket{\phi | \Omega \psi} &= \sum_{i j} \braket{i | \phi^*_i U K \psi_j | j} \notag \\
			&= \sum_{i j} \phi^*_i \psi^*_j \braket{i | U K | j} \notag \\
			&= \Big( \sum_{i j} \braket{j | K^* U^\dag | i} \phi_i \psi_j \Big)^* \notag \\
			&= \Big( \sum_{i j} \braket{j | \psi^*_j K^* U^\dag \phi_i | i} \Big)^* = \braket{\psi | K^* U^\dag | \phi}^*
		\end{align}
		其次,
		\begin{align}
			\braket{\Omega \phi | \Omega \psi} &= \braket{\phi | \Omega^\dag \Omega \psi} = \braket{\phi | K^* I K | \psi} \notag \\
			&= \sum_{i j} \braket{i | \phi^*_i K^* I K \psi_j | j} \notag \\
			&= \sum_{i j} \phi_i \psi^*_j \braket{i | K^* I K | j} = \braket{\psi | \phi}
		\end{align}
	\end{tcolorbox}
\end{itemize}

\section{time reversal in QM}
\begin{itemize}
	\item 在量子力学中,
	\begin{equation}
		\mathcal{T} : \ket{\psi} \mapsto \ket{\psi'(t')} = \int d^D x \, \ket{x} K \braket{x | \psi(t)} \quad \text{where} \quad t' = - t
	\end{equation}
	\begin{itemize}
		\item 因此, 对于动量本征态,
		\begin{equation}
			T \ket{p} = \int d^D x \, \ket{x} K e^{i \vec{p} \cdot \vec{x}} = \ket{- p}
		\end{equation}
		
		\item 对于动量算符,
		\begin{equation}
			T P T^\dag = \int d^D p \, \ket{- p} p \bra{- p} = - P
		\end{equation}
		
		\item 对于角动量算符,
		\begin{equation}
			T L T^\dag = T (X \times P) T^{- 1} = - L
		\end{equation}
	\end{itemize}
	
	\item 对于平面波,
	\begin{equation}
		\psi(t) = e^{i (\vec{k} \cdot \vec{x} - E t)} \mapsto \psi'(t') = \braket{x | K^* I K | \psi(t)} = e^{- i (\vec{k} \cdot \vec{x} - E t)}
	\end{equation}
	注意到 $t' = - t$, 代入,
	\begin{equation}
		\psi'(t) = e^{i (- \vec{k} \cdot \vec{x} - E t)}
	\end{equation}
\end{itemize}

\subsection{spin-\texorpdfstring{$\frac{1}{2}$}{1/2} non-relativistic electron}
\begin{itemize}
	\item 时间反演算符作用到 spin-up state 应该得到 spin-down state, 所以,
	\begin{equation}
		T = \sigma_2 K
	\end{equation}
	\begin{itemize}
		\item 因此,
		\begin{equation}
			T^2 = \sigma_2 K \sigma_2 K = \sigma_2^* \sigma_2 = - 1
		\end{equation}
		
		\item 具体地,
		\begin{equation}
			T \begin{pmatrix}
				1 \\
				0
			\end{pmatrix} = \begin{pmatrix}
				0 \\
				i
			\end{pmatrix} \quad T \begin{pmatrix}
				0 \\
				1
			\end{pmatrix} = \begin{pmatrix}
				- i \\
				0
			\end{pmatrix}
		\end{equation}
	\end{itemize}
	
	\item \textbf{Kramer's degeneracy:} 含有奇数个电子的时间反演不变系统, 其能级是 twofold degenerate.
	
	\begin{tcolorbox}[title=proof:]
		因为系统时间反演不变, 所以 $\psi$ 和 $T \psi$ 有相同的能级, 且 $T \psi \neq e^{i \alpha} \psi, \forall \alpha$.
		
		\noindent\rule[0.5ex]{\linewidth}{0.5pt} % horizontal line
		
		考虑 $T \psi = e^{i \alpha} \psi$, 那么,
		\begin{equation}
			T^2 \psi = T e^{i \alpha} \psi = e^{- i \alpha} e^{i \alpha} \psi = \psi
		\end{equation}
		与 $T^2 = - 1$ 矛盾.
	\end{tcolorbox}
\end{itemize}
