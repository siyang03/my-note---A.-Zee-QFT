\chapter{classical field theory and Noether's theorem}
\section{classical field theory}
\subsection{Lagrangian density and the action}
\begin{itemize}
	\item Lagrangian density, $\mathcal{L}$, 是 $\phi^a(x), \partial_\mu \phi^a(x), t$ 的函数.
	
	\item 对作用量变分得到 Euler-Lagrangian equation of motion,
	\begin{equation} \label{D.1.1}
		\frac{\delta \mathcal{L}}{\delta \phi^a} - \partial_\mu \Big( \frac{\delta \mathcal{L}}{\delta (\partial_\mu \phi^a)} \Big) = 0
	\end{equation}
	
	\begin{tcolorbox}[title=calculation:]
		对作用量进行变分,
		\begin{align}
			\delta S &= \int d^4 x \Big( \frac{\partial \mathcal{L}}{\partial \phi^a} \delta \phi^a + \frac{\partial \mathcal{L}}{\partial (\partial_\mu \phi^a)} \partial_\mu \delta \phi^a \Big) \notag \\
			&= \int d^4 x \bigg( \Big( \frac{\partial \mathcal{L}}{\partial \phi^a} - \partial_\mu \Big( \frac{\partial \mathcal{L}}{\partial (\partial_\mu \phi^a)} \Big) \Big) \delta \phi^a + \partial_\mu \Big( \frac{\partial \mathcal{L}}{\partial (\partial_\mu \phi^a)} \delta \phi^a \Big) \bigg)
		\end{align}
		由于边界变分为零...
	\end{tcolorbox}
\end{itemize}

\subsection{canonical momentum and the Hamiltonian}
\begin{itemize}
	\item \textbf{def.:} 定义一个叫 $\pi_a^\mu$ 的量,
	\begin{equation}
		\pi_a^\mu = \frac{\delta \mathcal{L}}{\delta (\partial_\mu \phi^a)}
	\end{equation}
	其中 $\pi_a \equiv \pi_a^0$ 称作 canonical momentum of the field.
	
	\item \textbf{def.:} the Hamiltonian density is,
	\begin{equation}
		\mathcal{H} = \pi_a \partial_0 \phi^a - \mathcal{L}
	\end{equation}
	
	\noindent\rule[0.5ex]{\linewidth}{0.5pt} % horizontal line
	
	\item the Hamilton's equations are,
	\begin{equation}
		\begin{dcases}
			\partial_0 \phi^a = \frac{\delta \mathcal{H}}{\delta \pi_a} \\
			- \partial_0 \pi^a = \frac{\delta \mathcal{H}}{\delta \phi^a} - \partial_i \Big( \frac{\delta \mathcal{H}}{\delta (\partial_i \phi^a)} \Big)
		\end{dcases}
	\end{equation}
	\begin{itemize}
		\item 第二个方程可以写成更紧凑的形式,
		\begin{equation}
			\partial_\mu \pi_a^\mu = \frac{\delta \mathcal{H}}{\delta \phi^a}
		\end{equation}
	\end{itemize}
\end{itemize}

\section{Noether's theorem}
\subsection{in classical particle mechanics}
\begin{itemize}
	\item 系统的 Lagrangian 为 $L(q^a, \dot{q}^a, t)$.
	
	\item 系统通过以下形式变换,
	\begin{equation}
		q^a(t) \mapsto q^a(\lambda, t) \quad \text{and} \quad q^a(t, 0) = q^a(t)
	\end{equation}
	并定义,
	\begin{equation}
		D_\lambda q^a = \frac{\partial q^a}{\partial \lambda} \Big|_{\lambda = 0}
	\end{equation}
	
	\item \textbf{Noether's theorem:} the continuous transform $\lambda$ is a \textbf{continuous symmetry} iff.,
	\begin{equation}
		D_\lambda L = \frac{d F(q^a, \dot{q}^a, t)}{dt}
	\end{equation}
	for some $F(q^a, \dot{q}^a, t)$, and the corresponding \textbf{conserved quantity} is,
	\begin{equation}
		Q = p_a D_\lambda q^a - F(q^a, \dot{q}^a, t)
	\end{equation}
	
	\begin{tcolorbox}[title=proof:]
		\begin{equation}
			D_\lambda L = \frac{\partial L}{\partial q^a} D_\lambda q^a + \frac{\partial L}{\partial \dot{q}^a} \frac{d D_\lambda q^a}{dt} = \frac{d}{dt} (p_a D_\lambda q^a)
		\end{equation}
	\end{tcolorbox}
	
	\item 几个例子如下,
	\begin{itemize}
		\item \textbf{空间平移}, $\vec{x}(t) \mapsto \vec{x}(t) + \hat{e}_i \lambda$, 相应地, $D_\lambda \vec{x} = \hat{e}_i$, 且,
		\begin{equation}
			D_\lambda L = \frac{\partial L}{\partial x^i}
		\end{equation}
		如果 $\frac{\partial L}{\partial x^i} = 0$, 那么, 有守恒量 $p_i$.
		
		\item \textbf{时间平移}, $q^a(t) \mapsto q^a(t + \lambda)$, 相应地, $D_\lambda q^a = \dot{q}^a$, 且,
		\begin{equation}
			D_\lambda L = \frac{d L}{dt} - \frac{\partial L}{\partial t}
		\end{equation}
		如果 $\frac{\partial L}{\partial t} = 0$, 那么, 有守恒量 $H = p_a \dot{q}^a - L$.
		
		\item \textbf{转动}, $\vec{x}(t) \mapsto R(\lambda, \hat{e}) \cdot \vec{x}(t)$, 相应地, $D_\lambda \vec{x} = \hat{e} \times \vec{x}$, 且,
		\begin{equation}
			D_\lambda L = \vec{x} \cdot \Big( \frac{\partial L}{\partial \vec{x}} \times \hat{e} \Big) + \hat{e} (\dot{\vec{x}} \times \vec{p})
		\end{equation}
		如果上式中两个括号内的项都为零, 那么, 有守恒量 $\hat{e} \cdot \vec{J} = \hat{e} \cdot (\vec{x} \times \vec{p})$.
	\end{itemize}
\end{itemize}

\subsection{in classical field theory}
\begin{itemize}
	\item 类似地, 系统通过以下形式变换,
	\begin{equation}
		\phi^a(x) \mapsto \phi^a(x, \lambda) \quad \text{and} \quad \phi^a(x, 0) = \phi^a(x)
	\end{equation}
	并定义,
	\begin{equation}
		D_\lambda \phi^a = \frac{\partial \phi^a}{\partial \lambda} \Big|_{\lambda = 0}
	\end{equation}
	
	\item \textbf{Noether's theorem:} the continuous transform $\lambda$ is a \textbf{continuous symmetry} iff.,
	\begin{equation} \label{D.2.11}
		D_\lambda \mathcal{L} = \partial_\mu F^\mu(\phi^a, \partial_\mu \phi^a, t)
	\end{equation}
	for some $F^\mu(\phi^a, \partial_\mu \phi^a, t)$, and the \textbf{conserved current} is,
	\begin{equation} \label{D.2.12}
		J^\mu = \pi_a^\mu D_\lambda \phi^a - F^\mu
	\end{equation}
	
	\begin{tcolorbox}[title=proof:]
		\begin{align}
			D_\lambda \mathcal{L} &= \frac{\delta \mathcal{L}}{\delta \phi^a} D_\lambda \phi^a + \frac{\delta \mathcal{L}}{\delta (\partial_\mu \phi^a)} \partial_\mu D_\lambda \phi^a \notag \\
			&= \Big( \frac{\delta \mathcal{L}}{\delta \phi^a} - \partial_\mu \Big( \frac{\delta \mathcal{L}}{\delta (\partial_\mu \phi^a)} \Big) \Big) D_\lambda \phi^a + \partial_\mu \Big( \underbrace{\frac{\delta \mathcal{L}}{\delta (\partial_\mu \phi^a)}}_{= \pi_a^\mu} D_\lambda \phi^a \Big)
		\end{align}
		代入 \eqref{D.1.1}, 得...
	\end{tcolorbox}
	
	\noindent\rule[0.5ex]{\linewidth}{0.5pt} % horizontal line
	
	\item 注意, conserved current 并不是唯一确定的, 考虑如下变换,
	\begin{equation}
		F^\mu \mapsto F'^\mu = F^\mu + \partial_\nu A^{\mu \nu} \quad \text{with} \quad A^{\mu \nu} = A^{[\mu \nu]}
	\end{equation}
	新 $F'^\mu$ 依然能满足 \eqref{D.2.11}.
	
	\item 但是, 守恒荷是唯一确定的.
	
	\begin{tcolorbox}[title=proof:]
		\begin{equation}
			Q' = \int d^3 x J^0 = \int d^3 x (\pi_a D_\lambda \phi^a - F^0) - \int d^3 x \, \partial_\mu A^{0 \mu}
		\end{equation}
		考虑到边界值为零, 且 $A^{0 0} = 0$, 所以 $Q' = Q$.
	\end{tcolorbox}
\end{itemize}

\subsection{spacetime translations and the energy-momentum tensor}
\begin{itemize}
	\item 时空平移变换为,
	\begin{equation}
		\phi^a(x) \mapsto \phi^a(x + \lambda e)
	\end{equation}
	
	\item 所以,
	\begin{equation}
		D_\lambda \phi^a = e^\mu \partial_\mu \phi^a \quad \text{and} \quad D_\lambda \mathcal{L} = e^\mu \partial_\mu \mathcal{L}
	\end{equation}
	代入 \eqref{D.2.12},
	\begin{equation}
		J^\mu = e^\nu (\underbrace{\pi_a^\mu \partial_\nu \phi^a - \delta^\mu_\nu \mathcal{L}}_{= \tensor{T}{^\mu_\nu}})
	\end{equation}
	
	\noindent\rule[0.5ex]{\linewidth}{0.5pt} % horizontal line
	
	\item 并且有,
	\begin{equation} \label{D.2.19}
		\partial_\mu T^{\mu \nu} = 0 \Longrightarrow P^\mu = \int d^3 x \, T^{0 \mu} = \text{Const.}
	\end{equation}
	来自守恒流散度为零.
\end{itemize}

\subsection{Lorentz transformations, angular momentum and something else}
\begin{itemize}
	\item Lorentz transformation 下坐标做变换 $x'^\mu = \tensor{\Lambda}{^\mu_\nu} x^\nu$, 其中 $\Lambda$ 满足,
	\begin{equation}
		\eta = \Lambda^T \eta \Lambda
	\end{equation}
	
	\item infinitesimal Lorentz transformation 是,
	\begin{equation}
		\Lambda = I + \epsilon
	\end{equation}
	其中 $\{\epsilon^{\mu \nu}\} = \epsilon \eta$ 是反对称矩阵.
	
	\begin{tcolorbox}[title=proof:]
		考虑,
		\begin{align}
			\eta &= (\Lambda \eta)^T \eta (\Lambda \eta) = (\eta + \epsilon \eta)^T \eta (\eta + \epsilon \eta) \notag \\
			&= \eta + \eta \epsilon^T + \epsilon \eta + O(\epsilon^2)
		\end{align}
	\end{tcolorbox}
	
	\noindent\hdashrule[0.5ex]{\linewidth}{0.5pt}{1mm} % horizontal dashed line
	
	\item 标量场在 Lorentz transform 下的变换为,
	\begin{equation}
		\Lambda : \phi^a(x) \mapsto \phi^a(\Lambda^{- 1} x')
	\end{equation}
	
	\noindent\rule[0.5ex]{\linewidth}{0.5pt} % horizontal line
	
	\item 有,
	\begin{equation}
		D_\lambda \phi^a = - \tensor{\epsilon}{^\mu_\nu} x^\nu \partial_\mu \phi^a \quad \text{and} \quad D_\lambda \mathcal{L} = - \tensor{\epsilon}{^\mu_\nu} x^\nu \partial_\mu \mathcal{L} = - \epsilon_{\mu \nu} \partial^\mu(x^\nu \mathcal{L})
	\end{equation}
	代入 \eqref{D.2.12},
	\begin{equation}
		J^\mu = \frac{1}{2} \epsilon_{\nu \rho} M^{\mu \nu \rho} \quad \text{where} \quad M^{\mu \nu \rho} = x^\nu T^{\mu \rho} - x^\rho T^{\mu \nu}
	\end{equation}
	且有,
	\begin{equation}
		\partial_\mu M^{\mu \nu \rho} = 0
	\end{equation}
	
	\noindent\hdashrule[0.5ex]{\linewidth}{0.5pt}{1mm} % horizontal dashed line
	
	\item 对全空间积分, 得到 6 个守恒量,
	\begin{equation}
		J^{\mu \nu} = \int d^3 x \, M^{0 \mu \nu} = \text{Const.}
	\end{equation}
	不难发现 $J^{i j}$ 对应角动量, 现在来讨论 $J^{0 i}$ 的物理意义,
	\begin{equation}
		0 = \frac{d}{dt} J^{0 i} = \frac{d}{dt} \int d^3 x (t T^{0 i} - x^i T^{0 0}) = P^i - \frac{d}{dt} \int d^3 x \, x^i T^{0 0}
	\end{equation}
	其中, 用到了 $\frac{d P^i}{dt} = 0$ (见 \eqref{D.2.19}), 可以将上式的第二项理解为质心运动的动量.
\end{itemize}
