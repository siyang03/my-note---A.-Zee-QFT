\chapter{the Dirac equation}
\begin{itemize}
	\item 整个 Part \ref{part II} 中, 我们使用 $(+, -, -, -)$ 号差, 因为 $\mathrm{Cl}_{1, 3}(\mathbb{R}) \cancel{\simeq} \mathrm{Cl}_{3, 1}(\mathbb{R})$.
	
	\item 本笔记中的算符的定义与 A. Zee 的定义不同, 存在如下对应关系,
	
	\begin{center}
		\newcolumntype{K}{>{\centering\arraybackslash}X}
		\begin{tabularx}{\linewidth}{KK}
			\toprule 
			A. Zee's def. & my def. \\
			\midrule 
			$\omega_{\mu \nu}$ & $\omega_{\mu \nu}$ \\
			$- i J^{\mu \nu}$ & $J^{\mu \nu}$ \\
			$- i \sigma^{\mu \nu}$ & $\sigma^{\mu \nu}$ \\
			\bottomrule
		\end{tabularx}
	\end{center}
\end{itemize}

\section{gamma matrices}
\begin{itemize}
	\item Pauli 矩阵如下,
	\begin{equation}
		\sigma_1 = \begin{pmatrix}
			0 & 1 \\
			1 & 0
		\end{pmatrix} \quad \sigma_2 = \begin{pmatrix}
			0 & - i \\
			i & 0
		\end{pmatrix} \quad \sigma_3 = \begin{pmatrix}
			1 & 0 \\
			0 & - 1
		\end{pmatrix}
	\end{equation}
	
	\item gamma 矩阵 (also called Dirac matrices) 如下 (其中 $i = 1, 2, 3$),
	\begin{equation}
		\gamma^0 = \begin{pmatrix}
			I & \\
			& - I
		\end{pmatrix} = I \otimes \tau_3 \quad \gamma^i = \begin{pmatrix}
			& \sigma_i \\
			- \sigma_i &
		\end{pmatrix} = i \sigma_i \otimes \tau_2 \quad \Omega = \gamma^0 \gamma^1 \gamma^2 \gamma^3 = - i \begin{pmatrix}
			& I \\
			I &
		\end{pmatrix} = - i I \otimes \tau_1
	\end{equation}
	其中 $\tau_{2, 3}$ 也是 Pauli 矩阵, 最后, 按照惯例, 定义 $\gamma^5 = i \Omega = I \otimes \tau_1$.
	\begin{itemize}
		\item 另外,
		\begin{equation}
			\begin{dcases}
				\gamma^0 \gamma^i = \sigma_i \otimes \tau_1 \\
				\gamma^i \gamma^j = - (\sigma_i \sigma_j) \otimes I
			\end{dcases} \quad \begin{dcases}
				\Omega \gamma^0 = - I \otimes \sigma_2 \\
				\Omega \gamma^i = i \sigma_i \otimes \tau_3
			\end{dcases}
		\end{equation}
	\end{itemize}
	
	\noindent\rule[0.5ex]{\linewidth}{0.5pt} % horizontal line
	
	\item gamma 矩阵满足,
	\begin{equation}
		\begin{dcases}
			(\gamma^\mu)^2 = \eta^{\mu \mu} \\
			\gamma^\mu \gamma^\nu = - \gamma^\nu \gamma^\mu & \mu \neq \nu
		\end{dcases} \Longrightarrow \{\gamma^\mu, \gamma^\nu\} = 2 \eta^{\mu \nu}
	\end{equation}
	
	\item 且存在如下关系,
	\begin{align}
		& \Omega \gamma^0 = - \gamma^1 \gamma^2 \gamma^3 \quad \Omega \gamma^1 = - \gamma^0 \gamma^2 \gamma^3 \quad \Omega \gamma^2 = \gamma^0 \gamma^1 \gamma^3 \quad \Omega \gamma^3 = - \gamma^0 \gamma^1 \gamma^2 \notag \\
		\iff & - \tensor{\epsilon}{^{\mu \nu \rho}_\sigma} \Omega \gamma^\sigma = \gamma^\mu \gamma^\nu \gamma^\rho \quad \text{when} \quad \mu \neq \nu \neq \rho
	\end{align}
	并且有 (注意到 $\Omega^2 = - 1$),
	\begin{equation}
		\{\Omega, \gamma^\mu\} = 0 \quad \{\Omega, \Omega \gamma^\mu\} = 0 \quad [\Omega, \gamma^\mu \gamma^\nu] = 0
	\end{equation}
	
	\noindent\hdashrule[0.5ex]{\linewidth}{0.5pt}{1mm} % horizontal dashed line
	
	\item 定义 $\sigma^{\mu \nu} = \frac{1}{2} [\gamma^\mu, \gamma^\nu]$ (注意, 我们的定义中没有虚数 $i$, 与 A. Zee 的定义不同),
	\begin{equation}
		\gamma^\mu \gamma^\nu = \frac{1}{2} \{\gamma^\mu, \gamma^\nu\} + \frac{1}{2} [\gamma^\mu, \gamma^\nu] = \eta^{\mu \nu} + \sigma^{\mu \nu} \Longrightarrow \begin{dcases}
			\sigma^{0 i} = \begin{pmatrix}
				& \sigma_i \\
				\sigma_i &
			\end{pmatrix} = \sigma_i \otimes \tau_1 \\
			\sigma^{i j} = - i \epsilon^{i j k} \begin{pmatrix}
				\sigma_k & \\
				& \sigma_k
			\end{pmatrix} = - i \epsilon^{i j k} \sigma_k \otimes I
		\end{dcases}
	\end{equation}
\end{itemize}

\section{Lorentz transformation and the \texorpdfstring{$(\frac{1}{2}, 0) \oplus (0, \frac{1}{2})$}{(1/2, 0)+(0, 1/2)} representation}
\begin{itemize}
	\item Lorentz 变换可以写成如下形式,
	\begin{equation}
		\Lambda = e^{\frac{1}{2} \omega_{\mu \nu} J^{\mu \nu}}
	\end{equation}
	其中 $\omega_{\mu \nu}$ 反对称, $J^{0 i}$ generate boosts and $J^{i j}$ generate rotations, (详见笔记 \href{https://github.com/siyang03/my-note---Lie-Groups-and-Lie-Algebras}{Lie Groups and Lie Algebras}).
	
	\begin{figure}[H]
		\centering
		\includegraphics[scale=1]{figures/Lorentz transformation.pdf}
		\caption{Lorentz transformation}
	\end{figure}
	
	\noindent\rule[0.5ex]{\linewidth}{0.5pt} % horizontal line
	
	\item 有 $\pi_{(\frac{1}{2}, 0) \oplus (0, \frac{1}{2})}(J^{\mu \nu}) = \frac{1}{2} \sigma^{\mu \nu}$ (up to a similarity transformation).
	
	\begin{tcolorbox}[title=calculation:]
		做如下相似变换,
		\begin{equation}
			P = \frac{\sqrt{2}}{2} \begin{pmatrix}
				I & I \\
				- I & I
			\end{pmatrix} \iff P^{- 1} = \frac{\sqrt{2}}{2} \begin{pmatrix}
				I & - I \\
				I & I
			\end{pmatrix}
		\end{equation}
		得到,
		\begin{equation}
			P^{- 1} \sigma^{0 i} P = \begin{pmatrix}
				- \sigma_i & \\
				& \sigma_i
			\end{pmatrix} \quad P^{- 1} \sigma^{i j} P = - i \epsilon^{i j k} \begin{pmatrix}
				\sigma_k & \\
				& \sigma_k
			\end{pmatrix}
		\end{equation}
		得到的结果和笔记 \href{https://github.com/siyang03/my-note---Lie-Groups-and-Lie-Algebras}{Lie Groups and Lie Algebras} 中 $(\frac{1}{2}, 0) \oplus (0, \frac{1}{2})$ 表示是完全一样的.
	\end{tcolorbox}
	
	\item Dirac spinor 是 $(\frac{1}{2}, 0) \oplus (0, \frac{1}{2})$ rep. 的 vector space 中的元素,
	\begin{equation}
		\psi = \begin{pmatrix}
			\phi \\
			\chi
		\end{pmatrix} \quad \text{with} \quad P^{- 1} \psi = \frac{\sqrt{2}}{2} \begin{pmatrix}
			\phi - \chi \\
			\phi + \chi
		\end{pmatrix}
	\end{equation}
	其中,
	\begin{equation}
		\frac{\sqrt{2}}{2} \begin{pmatrix}
			\phi \\
			- \phi
		\end{pmatrix} \in (\frac{1}{2}, 0) \quad \text{and} \quad \frac{\sqrt{2}}{2} \begin{pmatrix}
			\chi \\
			\chi
		\end{pmatrix} \in (0, \frac{1}{2})
	\end{equation}
	
	\noindent\rule[0.5ex]{\linewidth}{0.5pt} % horizontal line
	
	\item 对于 gamma 矩阵, 有,
	\begin{equation}
		\Pi(\Lambda) \gamma^\rho \Pi^{- 1}(\Lambda) = e^{\frac{1}{4} \omega_{\mu \nu} \sigma^{\mu \nu}} \gamma^\rho e^{- \frac{1}{4} \omega_{\mu \nu} \sigma^{\mu \nu}} = \tensor{(\Lambda^{- 1})}{^\rho_\sigma} \gamma^\sigma
	\end{equation}
	
	\begin{tcolorbox}[title=calculation:]
		利用 Campbell's identity,
		\begin{equation}
			e^{\frac{1}{4} \omega_{\mu \nu} \sigma^{\mu \nu}} \gamma^\rho e^{- \frac{1}{4} \omega_{\mu \nu} \sigma^{\mu \nu}} = e^{\frac{1}{4} \omega_{\mu \nu} \mathrm{ad}_{\sigma^{\mu \nu}}} \gamma^\rho
		\end{equation}
		其中 (注意 $\tensor{(J^{\mu \nu})}{^\rho_\sigma} = 2 \eta^{[\mu| \rho} \tensor{\delta}{^{|\nu]}_\sigma}$, 其中度规号差与笔记 \href{https://github.com/siyang03/my-note---Lie-Groups-and-Lie-Algebras}{Lie Groups and Lie Algebras} 中的不同),
			\begin{align}
			& \begin{dcases}
				\rho \neq \mu, \nu & \begin{aligned}
					[\sigma^{\mu \nu}, \gamma^\rho] &= \frac{1}{2} (\gamma^\mu \gamma^\nu \gamma^\rho - \gamma^\nu \gamma^\mu \gamma^\rho - \gamma^\rho \gamma^\mu \gamma^\nu + \gamma^\rho \gamma^\nu \gamma^\mu) \\
					&= - \frac{1}{2} \underbrace{(\epsilon^{\mu \nu \rho \sigma} - \epsilon^{\nu \mu \rho \sigma} - \epsilon^{\rho \mu \nu \sigma} + \epsilon^{\rho \nu \mu \sigma})}_{= 0} \Omega \gamma_\sigma = 0
				\end{aligned} \\
				\rho = \mu \ \text{or} \ \nu \ \text{and} \ \mu \neq \nu & [\sigma^{\mu \nu}, \gamma^\rho] = 2 (\eta^{\mu \rho} \gamma^\mu - \eta^{\nu \rho} \gamma^\nu)
			\end{dcases} \notag \\
			\Longrightarrow & [\sigma^{\mu \nu}, \gamma^\rho] = 2 (\eta^{\nu \rho} \gamma^\mu - \eta^{\mu \rho} \gamma^\nu) = - 2 \tensor{(J^{\mu \nu})}{^\rho_\sigma} \gamma^\sigma
		\end{align}
		代入, 得到,
		\begin{equation}
			e^{\frac{1}{4} \omega_{\mu \nu} \mathrm{ad}_{\sigma^{\mu \nu}}} \gamma^\rho = \sum_{n = 0}^\infty \frac{1}{n!} \tensor{\Big( \Big( - \frac{1}{2} \omega_{\mu \nu} J^{\mu \nu} \Big)^n \Big)}{^\rho_\sigma} \gamma^\sigma = \tensor{(\Lambda^{- 1})}{^\rho_\sigma} \gamma^\sigma
		\end{equation}
		
		\noindent\hdashrule[0.5ex]{\linewidth}{0.5pt}{1mm} % horizontal dashed line
		
		可以用 "无穷小" Lorentz 变换验证以上计算,
		\begin{align}
			\Pi(1 + \delta \tensor{\omega}{^\mu_\nu}) \gamma^\rho \Pi^{- 1}(1 + \delta \tensor{\omega}{^\mu_\nu}) &= \gamma^\rho + \frac{1}{4} \delta \omega_{\mu \nu} [\sigma^{\mu \nu}, \gamma^\rho] \notag \\
			&= (1 - \delta \tensor{\omega}{^\rho_\sigma}) \gamma^\sigma
		\end{align}
	\end{tcolorbox}
	
	\item 对于 Dirac spinor,
	\begin{equation}
		\Pi(\Lambda) \psi(x) = \psi'(\Lambda x)
	\end{equation}
	注意 $\partial'_\mu = \tensor{\Lambda}{_\mu^\nu} \partial_\nu$, 所以,
	\begin{equation}
		(i \gamma^\mu \partial_\mu - m) \psi(x) = 0 \iff (i \gamma^\mu \partial'_\mu - m) \psi'(\Lambda x) = 0
	\end{equation}
	
	\begin{tcolorbox}[title=calculation:]
		首先,
		\begin{equation}
			\Lambda^T \eta \Lambda = \eta \Longrightarrow \tensor{(\Lambda^{- 1})}{^\mu_\nu} = \tensor{(\eta \Lambda^T \eta)}{^\mu_\nu} = \tensor{\Lambda}{_\nu^\mu}
		\end{equation}
		考虑,
		\begin{equation}
			\Pi^{- 1}(\Lambda) \gamma^\mu \Pi(\Lambda) = \tensor{\Lambda}{^\mu_\nu} \gamma^\nu \Longrightarrow \gamma^\mu \Pi(\Lambda) = \tensor{\Lambda}{^\mu_\nu} \Pi(\Lambda) \gamma^\nu
		\end{equation}
		代入,
		\begin{align}
			(i \gamma^\mu \partial'_\mu - m) \psi'(\Lambda x) &= (i \gamma^\mu \tensor{\Lambda}{_\mu^\nu} \partial_\nu - m) \Pi(\Lambda) \psi(x) \notag \\
			&= \Pi(\Lambda) (i \gamma^\rho \underbrace{\tensor{\Lambda}{^\mu_\rho} \tensor{\Lambda}{_\mu^\nu}}_{= \delta^\nu_\rho} \partial_\nu - m) \psi(x) \notag \\
			&= \Pi(\Lambda) (i \gamma^\mu \partial_\mu - m) \psi(x) = 0
		\end{align}
	\end{tcolorbox}
	\begin{itemize}
		\item 关键部分在于,
		\begin{equation}
			\gamma^\mu \psi'(\Lambda x) = \gamma^\mu \Pi(\Lambda) \psi(x) = \tensor{\Lambda}{^\mu_\nu} \gamma^\nu \psi(x)
		\end{equation}
	\end{itemize}
\end{itemize}

\section{Dirac equation}
\begin{itemize}
	\item A. Zee: our discussion provides a unified view of the equations of motion in relativistic physics: they just project out the unphysical components.
	
	\item the Dirac equation is,
	\begin{equation}
		(i \gamma^\mu \partial_\mu - m) \psi = 0 \iff (\gamma^\mu p_\mu - m) \tilde{\psi} = 0
	\end{equation}
	首先可以看出 $\psi$ 满足 Klein-Gordan equation,
	\begin{align}
		& (i \gamma^\mu \partial_\mu - m) (i \gamma^\nu \partial_\nu - m) \psi = \Big( - \frac{1}{2} \{\gamma^\mu, \gamma^\nu\} \partial_\mu \partial_\nu - 2 i m \gamma^\mu \partial_\mu + m^2 \Big) \psi = 0 \notag \\
		\Longrightarrow & (- \partial^2 - m^2) \psi = 0
	\end{align}
	\begin{itemize}
		\item 在粒子静止系下 $p_\mu = (m, 0, 0, 0)$, Dirac 方程给出,
		\begin{equation}
			(\gamma^0 - 1) \tilde{\psi} = 0 \Longrightarrow \begin{pmatrix}
				0 & \\
				& I
			\end{pmatrix} \tilde{\psi} = 0
		\end{equation}
		因此, $\tilde{\psi}$ 的后两个分量为零 $\Longrightarrow \psi$ 只有两个自由度.
	\end{itemize}
\end{itemize}
