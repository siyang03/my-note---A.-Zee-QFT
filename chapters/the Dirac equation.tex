\chapter{the Dirac equation}
\section{Dirac equation}
\begin{itemize}
	\item A. Zee: our discussion provides a unified view of the equations of motion in relativistic physics: they just project out the unphysical components.
	
	\item the Dirac equation is,
	\begin{equation}
		(i \gamma^\mu \partial_\mu - m) \Psi = 0 \iff (\gamma^\mu p_\mu - m) \tilde{\Psi} = 0
	\end{equation}
	首先可以看出 $\Psi$ 满足 Klein-Gordan equation,
	\begin{align}
		& (i \gamma^\mu \partial_\mu - m) (i \gamma^\nu \partial_\nu - m) \Psi = \Big( - \frac{1}{2} \{\gamma^\mu, \gamma^\nu\} \partial_\mu \partial_\nu - 2 i m \gamma^\mu \partial_\mu + m^2 \Big) \Psi = 0 \notag \\
		\Longrightarrow & (- \partial^2 - m^2) \Psi = 0
	\end{align}
	\begin{itemize}
		\item 在粒子静止系下 $p_\mu = (m, 0, 0, 0)$, Dirac 方程给出,
		\begin{equation}
			(\gamma^0 - 1) \tilde{\Psi} = 0 \Longrightarrow \begin{pmatrix}
				0 & \\
				& I
			\end{pmatrix} \tilde{\Psi} = 0
		\end{equation}
		因此, $\tilde{\Psi}$ 的后两个分量为零 $\Longrightarrow \Psi$ 只有两个自由度.
	\end{itemize}
	
	\item Dirac 方程的 Lorentz covariance 见 \eqref{6.2.12}.
\end{itemize}

\section{Dirac Lagrangian}
\begin{itemize}
	\item 根据 \eqref{6.2.21} 以及之前标量场的计算经验, 可知,
	\begin{equation} \label{7.2.1}
		\mathcal{L} = \bar{\Psi} (i \gamma^\mu \partial_\mu - m) \Psi = (- i \partial_\mu \bar{\Psi} \gamma^\mu - m \bar{\Psi}) \Psi + \text{total diff.}
	\end{equation}
	其中, 与复标量场论中类似, 可以把 $\Psi, \Psi^\dag$ 或 $\Psi, \bar{\Psi}$ 视为独立变量.
\end{itemize}

\section{chirality or handedness}
\begin{itemize}
	\item 本 section 使用 Weyl basis.
	
	\item parity transformation 会把 left spinor 变成 right spinor and vice versa,
	\begin{equation}
		\gamma^0 \Psi_L = \begin{pmatrix}
			0 \\
			\psi_L
		\end{pmatrix} \quad \gamma^0 \Psi_R = \begin{pmatrix}
			\psi_R \\
			0
		\end{pmatrix}
	\end{equation}
	
	\item 把 Lagrangian 中的 $\Psi$ 拆开,
	\begin{align}
		\mathcal{L} &= \bar{\Psi}_L (i \cancel{\partial}) \Psi_L + \bar{\Psi}_R (i \cancel{\partial}) \Psi_R - m (\bar{\Psi}_L \Psi_R + \bar{\Psi}_R \Psi_L) \notag \\
		&= \psi_L^\dag i \bar{\sigma}^\mu \partial_\mu \psi_L + \psi_R^\dag i \sigma^\mu \partial_\mu \psi_R - m (\psi_L^\dag \psi_R + \psi_R^\dag \psi_L)
	\end{align}
	其中注意到了 $\gamma^0 \gamma^\mu$ 的非对角分块为零.
\end{itemize}

\subsection{internal vector symmetry}
\begin{itemize}
	\item 做变换 $\Psi \mapsto e^{i \theta} \Psi$, Lagrangian 保持不变, 利用 Noether's theorem 得到守恒流 (见 section \ref{D.2}),
	\begin{equation}
		J_V^\mu = \bar{\Psi} \gamma^\mu \Psi
	\end{equation}
	其中, 按照惯例省略了虚数 $i$.
	
	\begin{tcolorbox}[title=calculation:]
		计算广义动量,
		\begin{equation}
			\begin{dcases}
				\pi_\Psi^\mu = \frac{\delta \mathcal{L}}{\delta \partial_\mu \Psi} = \bar{\Psi} i \gamma^\mu \\
				\pi_{\bar{\Psi}}^\mu = 0
			\end{dcases} \quad \text{or} \quad \begin{dcases}
				\pi_\Psi^\mu = 0 \\
				\pi_{\bar{\Psi}}^\mu = \frac{\delta \mathcal{L}}{\delta \partial_\mu \bar{\Psi}} = - i \gamma^\mu \Psi
			\end{dcases}
		\end{equation}
		
		\noindent\rule[0.5ex]{\linewidth}{0.5pt} % horizontal line
		
		这里看起来有点奇怪, 需要再说明一下. 对于 \eqref{7.2.1} 第一个等号后边,
		\begin{equation}
			\begin{dcases}
				\pi_\Psi^\mu = \frac{\delta \mathcal{L}}{\delta \partial_\mu \Psi} = \bar{\Psi} i \gamma^\mu & \frac{\delta \mathcal{L}}{\delta \Psi} = - m \bar{\Psi} \\
				\pi_{\bar{\Psi}}^\mu = 0 & \frac{\delta \mathcal{L}}{\delta \bar{\Psi}} = (i \gamma^\mu \partial_\mu - m) \Psi
			\end{dcases} \Longrightarrow \begin{dcases}
				- (\partial_\mu \bar{\Psi}) i \gamma^\mu - m \bar{\Psi} = 0 \\
				(i \gamma^\mu \partial_\mu - m) \Psi = 0
			\end{dcases}
		\end{equation}
		对于 \eqref{7.2.1} 第二个等号后边, 忽略掉全微分项,
		\begin{equation}
			\begin{dcases}
				\pi_\Psi^\mu = 0 & \frac{\delta \mathcal{L}}{\delta \Psi} = - i \partial_\mu \bar{\Psi} \gamma^\mu - m \bar{\Psi} \\
				\pi_{\bar{\Psi}}^\mu = \frac{\delta \mathcal{L}}{\delta \partial_\mu \bar{\Psi}} = - i \gamma^\mu \Psi & \frac{\delta \mathcal{L}}{\delta \bar{\Psi}} = - m \Psi
			\end{dcases} \Longrightarrow \begin{dcases}
				- i \partial_\mu \bar{\Psi} \gamma^\mu - m \bar{\Psi} = 0 \\
				(i \gamma^\mu \partial_\mu - m) \Psi = 0
			\end{dcases}
		\end{equation}
	\end{tcolorbox}
\end{itemize}

\subsection{axial symmetry}
\begin{itemize}
	\item 做变换,
	\begin{equation}
		\Psi \mapsto e^{i \theta \gamma^5} \Psi = \begin{pmatrix}
			e^{- i \theta} \Psi_L \\
			e^{i \theta} \Psi_R
		\end{pmatrix}
	\end{equation}
	在 $m = 0$ 时 Lagrangian 保持不变, 对应的守恒流为,
	\begin{equation}
		J_A^\mu = \bar{\Psi} \gamma^\mu \gamma^5 \Psi
	\end{equation}
	根据 \eqref{6.2.23}, 是一个 pseudovector.
\end{itemize}

\section{energy-momentum tensor and angular momentum}
\begin{itemize}
	\item 
\end{itemize}

\section{interaction in QED}
\begin{itemize}
	\item 
\end{itemize}
