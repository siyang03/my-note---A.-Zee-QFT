\chapter{Maxwell's equations}
\section{the \texorpdfstring{$(1, 0) \oplus (0, 1)$}{(1, 0)+(0, 1)} representation of the Lorentz algebra}
\begin{itemize}
	\item 反对称张量 $F_{[\mu \nu]}$ 是 $(1, 0) \oplus (0, 1)$ rep. 中的向量, 详见笔记 \href{https://github.com/siyang03/my-note---Lie-Groups-and-Lie-Algebras}{Lie Groups and Lie Algebras}.
\end{itemize}

\section{Maxwell's equations}
\begin{itemize}
	\item 电磁场的 Lagrangian 为 (现实中 $\mu = 0$)
	\begin{equation} \label{11.2.1}
		\mathcal{L} = - \frac{1}{4} F^{\mu \nu} F_{\mu \nu} + \frac{1}{2} \mu^2 A^\mu A_\mu, \quad \text{with} \quad F_{\mu \nu} = 2 \partial_{[\mu} A_{\nu]},
	\end{equation}
	其中 $A_\mu = (\phi, - \vec{A})$, 对作用量变分得到运动方程 (Proca equation),
	\begin{equation} \label{11.2.2}
		\partial^\nu F_{\nu \mu} + \mu^2 A_\mu = 0.
	\end{equation}
	
	\begin{tcolorbox}[title=calculation:]
		场方程还可以写成
		\begin{equation}
			\Big( - \partial^\mu \partial^\nu + \eta^{\mu \nu} (\partial^2 + \mu^2) \Big) A_\nu = 0 \iff (k^2 - \mu^2) \tilde{A}_\mu(k) = k_\mu k^\nu \tilde{A}_\nu(k)
		\end{equation}
		和
		\begin{equation}
			\begin{dcases}
				- \nabla^2 A_0 - \nabla \cdot \frac{\partial \vec{A}}{\partial t} + \mu^2 A_0 = 0 \\
				\frac{\partial}{\partial t} \Big( - \nabla A_0 - \frac{\partial \vec{A}}{\partial t} \Big) + \Big( \underbrace{\nabla^2 \vec{A} - \nabla (\nabla \cdot \vec{A})}_{= - \nabla \times (\nabla \times \vec{A})} \Big) - \mu^2 \vec{A} = 0
			\end{dcases}.
		\end{equation}
	\end{tcolorbox}
	
	\begin{itemize}
		\item 如果引入 Lorentz gauge condition (如 \eqref{2.1.12} 所示, 在 $\mu \neq 0$ 时必然成立),
		\begin{equation}
			\text{field eq. \eqref{11.2.2}} \overset{\mu \neq 0}{\iff} \begin{dcases}
				(\partial^2 + \mu^2) A_\mu = 0 \\
				\partial^\mu A_\mu = 0
			\end{dcases}.
		\end{equation}
	\end{itemize}
	
	\item 此外, $F_{\mu \nu}$ 满足 Bianchi identity,
	\begin{equation} \label{11.2.4}
		\nabla_\rho F_{\mu \nu} + \nabla_\nu F_{\rho \mu} + \nabla_\mu F_{\nu \rho} = 0.
	\end{equation}
	
	\begin{tcolorbox}[title=calculation:]
		代入定义式,
		\begin{align}
			\nabla_\rho \nabla_{[\mu} A_{\nu]} + \cdots =& + \mathcolor{red}{\rho \mu \nu} - \mathcolor{blue}{\rho \nu \mu} \notag \\
			& + \mathcolor{blue}{\nu \rho \mu} - \mathcolor{orange}{\nu \mu \rho} \notag \\
			& + \mathcolor{orange}{\mu \nu \rho} - \mathcolor{red}{\mu \rho \nu} \notag \\
			=& (\underbrace{\tensor{R}{_{\rho \mu \nu}^\sigma} + \tensor{R}{_{\nu \rho \mu}^\sigma} + \tensor{R}{_{\mu \nu \rho}^\sigma}}_{= 0}) A_\sigma.
		\end{align}
	\end{tcolorbox}
	
	\noindent\rule[0.5ex]{\linewidth}{0.5pt} % horizontal line
	
	\item 令
	\begin{equation}
		F_{\mu \nu} = \begin{pmatrix}
			0 & E_1 & E_2 & E_3 \\
			- E_1 & 0 & - B_3 & B_2 \\
			- E_2 & B_3 & 0 & - B_1 \\
			- E_3 & - B_2 & B_1 & 0
		\end{pmatrix} \iff \begin{dcases}
			F_{0 i} = \vec{E} = - \nabla \phi - \frac{\partial \vec{A}}{\partial t} \\
			- \frac{1}{2} \epsilon^{i j k} F_{j k} = \vec{B} = \nabla \times \vec{A}
		\end{dcases},
	\end{equation}
	代入场方程 \eqref{11.2.2},
	\begin{equation}
		\begin{dcases}
			\nabla \cdot \vec{E} + \mu^2 \phi = 0 \\
			- \frac{\partial \vec{E}}{\partial t} + \nabla \times \vec{B} + \mu^2 \vec{A} = 0
		\end{dcases},
	\end{equation}
	代入 Bianchi identity \eqref{11.2.4},
	\begin{equation}
		\begin{dcases}
			\nabla \cdot \vec{B} = 0 & \rho, \mu, \nu = 1, 2, 3 \\
			\nabla \times \vec{E} - \frac{\partial \vec{B}}{\partial t} = 0 & \rho, \mu, \nu = 0, i, j
		\end{dcases}.
	\end{equation}
	
	\item 最后, 电磁场的能动量张量见 subsection \ref{subsection D.4.1}.
\end{itemize}

\subsection{gauge symmetry (gauge redundancy)}
\begin{itemize}
	\item $A_\mu$ 有 4 个分量, 但光子只有 2 个自由度 (偏振态).
	
	\item 首先, 考虑 $\tilde{A}_\mu$ 的场方程, 对于指标 $\mu = 0$ 有 (计算过程见 subsection \ref{subsection A.2.1})
	\begin{equation} \label{11.2.11}
		\tilde{A}_0 = \frac{k^0}{\omega_k^2} \vec{k} \cdot \vec{\tilde{A}} \Longrightarrow A_0(t, \vec{x}) = \int d^3 y \, \frac{e^{- \mu |\vec{x} - \vec{y}|}}{4 \pi |\vec{x} - \vec{y}|} \nabla_y \cdot \frac{\partial \vec{A}(t, \vec{y})}{\partial t},
	\end{equation}
	这是一个约束条件, 将此式代入剩余的场方程, 得到
	\begin{equation} \label{11.2.12}
		(k^2 - \mu^2) \vec{\tilde{A}} = (k^2 - \mu^2) \frac{\vec{k} (\vec{k} \cdot \vec{\tilde{A}})}{\omega_k^2},
	\end{equation}
	因此:
	\begin{itemize}
		\item 当 $\mu = 0$ 时,
		\begin{equation}
			\begin{dcases}
				\text{on shell:} & \vec{\tilde{A}} \ \text{取值任意} \\
				\text{off shell:} & \tilde{A}_0 = \frac{k^0}{|\vec{k}|} |\vec{\tilde{A}}|, \vec{\tilde{A}} = |\vec{\tilde{A}}| \hat{e}_k
			\end{dcases} \Longrightarrow \tilde{A}_\mu(k) = \tilde{\mathcal{A}}_\mu(k) 2 \pi \delta(k^2) - i k_\mu \tilde{\lambda}(k),
		\end{equation}
		其中 $\vec{\tilde{\mathcal{A}}}(k), \tilde{\lambda}(k) = \frac{|\vec{\tilde{A}}|}{|\vec{k}|}$ 是任意函数, 且 $\mathrm{sign}(k^0) |\vec{k}| \tilde{\mathcal{A}}_0 - \vec{k} \cdot \vec{\tilde{\mathcal{A}}} = 0$. 因此
		\begin{equation} \label{11.2.14}
			A_\mu(x) = \partial_\mu \lambda(x) + \int \frac{d^3 k}{(2 \pi)^3 2 |\vec{k}|} \Big( \tilde{\mathcal{A}}_\mu(|\vec{k}|, \vec{k}) e^{- i (|\vec{k}| x^0 - \vec{k} \cdot \vec{x})} + \tilde{\mathcal{A}}^*_\mu(|\vec{k}|, \vec{k}) e^{i (|\vec{k}| x^0 - \vec{k} \cdot \vec{x})} \Big),
		\end{equation}
		第二项是 on shell 平面波解的叠加, 并且振幅满足 $k^\mu \tilde{\mathcal{A}}_\mu(|\vec{k}|, \vec{k}) = 0$.
		
		\item 当 $\mu \neq 0$ 时,
		\begin{equation}
			\begin{dcases}
				\text{on shell:} & \vec{\tilde{A}} \ \text{取值任意} \\
				\text{off shell:} & \vec{k} \cdot \vec{\tilde{A}} = 0 \Longrightarrow \vec{\tilde{A}} = \tilde{A}_0 = 0
			\end{dcases} \Longrightarrow \tilde{A}_\mu(k) = \tilde{\mathcal{A}}_\mu(k) 2 \pi \delta(k^2 - \mu^2).
		\end{equation}
		其中 $\vec{\tilde{\mathcal{A}}}(k)$ 是任意函数, 且 $\mathrm{sign}(k^0) \omega_k \tilde{\mathcal{A}}_0 - \vec{k} \cdot \vec{\tilde{\mathcal{A}}} = 0$. 因此
		\begin{equation}
			A_\mu(x) = \int \frac{d^3 k}{(2 \pi)^3 2 \omega_k} \Big( \tilde{\mathcal{A}}_\mu(\omega_k, \vec{k}) e^{- i (\omega_k x^0 - \vec{k} \cdot \vec{x})} + \tilde{\mathcal{A}}^*_\mu(\omega_k, \vec{k}) e^{i (\omega_k x^0 - \vec{k} \cdot \vec{x})} \Big).
		\end{equation}
	\end{itemize}
	
	\begin{tcolorbox}[title=calculation:]
		对 \eqref{11.2.12} 两边同时内积 $\vec{k}$, 有
		\begin{equation}
			(k^2 - \mu^2) (\vec{k} \cdot \vec{\tilde{A}}) = 0 \quad \text{or} \quad \omega_k = |\vec{k}|,
		\end{equation}
		因此 $\mu = 0$ 时 \eqref{11.2.12} 自然成立, 而 massive 情况下 $\tilde{A}_0 = \vec{k} \cdot \vec{\tilde{A}} = 0$ 除非 on shell.
	\end{tcolorbox}
	
	\item 除此之外还需要一个约束条件 (接下来默认 $\mu = 0$).
	
	\item 考虑 \eqref{11.2.14}, 注意到给定初始条件 $A_\mu(t_0, \vec{x}), \partial_\nu A_\mu(t_0, \vec{x})$ 无法唯一确定参数 $\lambda(x), \tilde{\mathcal{A}}_\mu(|\vec{k}|, \vec{k})$.
	
	\begin{tcolorbox}[title=calculation:]
		对 $A_\mu$ 做平面波分解,
		\begin{equation}
			\begin{dcases}
				\int d^3 x \, e^{- i \vec{p} \cdot \vec{x}} A_0(x) = \frac{1}{2 |\vec{p}|} \Big( e^{- i |\vec{p}| x^0} \tilde{\mathcal{A}}_0(|\vec{p}|, \vec{p}) + e^{i |\vec{p}| x^0} \tilde{\mathcal{A}}_0(- |\vec{p}|, \vec{p}) \Big) + \int d^3 x \, e^{- i \vec{p} \cdot \vec{x}} \frac{\partial \lambda}{\partial t} \\
				\int d^3 x \, e^{- i \vec{p} \cdot \vec{x}} \vec{A}(x) = \frac{1}{2 |\vec{p}|} \Big( e^{- i |\vec{p}| x^0} \vec{\tilde{\mathcal{A}}}(|\vec{p}|, \vec{p}) + e^{i |\vec{p}| x^0} \vec{\tilde{\mathcal{A}}}(- |\vec{p}|, \vec{p}) \Big) + i \vec{p} \int d^3 x \, e^{- i \vec{p} \cdot \vec{x}} \lambda
			\end{dcases},
		\end{equation}
		对 $\partial_\nu A_\mu$ 做平面波分解,
		\begin{equation}
			\begin{dcases}
				\frac{\partial A_0(x)}{\partial t} \mapsto \frac{- i}{2} \Big( e^{- i |\vec{p}| x^0} \tilde{\mathcal{A}}_0(|\vec{p}|, \vec{p}) - e^{i |\vec{p}| x^0} \tilde{\mathcal{A}}_0(- |\vec{p}|, \vec{p}) \Big) + \int d^3 x \, e^{- i \vec{p} \cdot \vec{x}} \frac{\partial^2 \lambda}{\partial t^2} \\
				\frac{\partial \vec{A}(x)}{\partial t} \mapsto \frac{- i}{2} \Big( e^{- i |\vec{p}| x^0} \vec{\tilde{\mathcal{A}}}(|\vec{p}|, \vec{p}) - e^{i |\vec{p}| x^0} \vec{\tilde{\mathcal{A}}}(- |\vec{p}|, \vec{p}) \Big) - i \vec{p} \int d^3 x \, e^{- i \vec{p} \cdot \vec{x}} \frac{\partial \lambda}{\partial t} \\
				\nabla A_\mu \mapsto i \vec{p} \int d^3 x \, e^{- i \vec{p} \cdot \vec{x}} A_\mu(x) \quad \text{(这是冗余的)}
			\end{dcases},
		\end{equation}
		可见, 由于 $\lambda$ 的存在, 我们无法唯一确定参数 $\tilde{\mathcal{A}}_\mu$.
	\end{tcolorbox}
	
	也就是说, 给定初始条件, 我们可以求解 $A_\mu(x)$ up to a function $\partial_\mu \lambda$.
	
	\item gauge redundancy: 将 $A_\mu$ 和 $A_\mu + \partial_\mu \lambda$ 认为是同一个物理态.
	
	\item Lorentz gauge 是
	\begin{equation}
		\partial_\mu A^\mu = 0,
	\end{equation}
	注意到方程 $\partial^2 \lambda = - f$ 总是有解, 它的 Green's function 是
	\begin{equation}
		G^{(\pm)}(x) = \frac{1}{4 \pi |\vec{x}|} \delta(x^0 \mp |\vec{x}|).
	\end{equation}
	\begin{itemize}
		\item Lorentz gauge 下, $A_\mu$ 的通解是
		\begin{equation}
			A_\mu(x) = \int \frac{d^3 k}{(2 \pi)^3 2 |\vec{k}|} \Big( \tilde{\mathcal{A}}_\mu(|\vec{k}|, \vec{k}) e^{- i (|\vec{k}| x^0 - \vec{k} \cdot \vec{x})} + \tilde{\mathcal{A}}^*_\mu(|\vec{k}|, \vec{k}) e^{i (|\vec{k}| x^0 - \vec{k} \cdot \vec{x})} \Big).
		\end{equation}
	\end{itemize}
	
	\item Coulomb gauge 是
	\begin{equation}
		\nabla \cdot \vec{A} = 0,
	\end{equation}
	注意到方程 $\nabla^2 \lambda = - f$ 总是有解. 且根据 \eqref{11.2.11} 可知 $A_0(x) = 0$.
	\begin{itemize}
		\item Coulomb gauge 下, $A_\mu$ 的通解是
		\begin{equation}
			\vec{A}(x) = \int \frac{d^3 k}{(2 \pi)^3 2 |\vec{k}|} \Big( \vec{\tilde{\mathcal{A}}}_\perp(|\vec{k}|, \vec{k}) e^{- i (|\vec{k}| x^0 - \vec{k} \cdot \vec{x})} + \vec{\tilde{\mathcal{A}}}^*_\perp(|\vec{k}|, \vec{k}) e^{i (|\vec{k}| x^0 - \vec{k} \cdot \vec{x})} \Big),
		\end{equation}
		其中 $\vec{\tilde{\mathcal{A}}}_\perp = \vec{\tilde{\mathcal{A}}} - \hat{e}_k \hat{e}_k \cdot \vec{\tilde{\mathcal{A}}}$.
	\end{itemize}
\end{itemize}
