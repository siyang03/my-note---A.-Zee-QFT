\chapter{the quantization of the electromagnetic field}
\section{massive}
\begin{itemize}
	\item 场的广义动量为
	\begin{equation}
		\pi^0 = 0, \quad \vec{\pi} = \vec{E} = i \int \frac{d^3 k}{(2 \pi)^3 2 \omega_k} \omega_k \Big( 1 - \frac{\vec{k} \vec{k}}{\omega_k^2} \Big) \cdot \Big( \vec{\tilde{\mathcal{A}}}(k) e^{- i (\omega_k x^0 - \vec{k} \cdot \vec{x})} - \vec{\tilde{\mathcal{A}}}^*(k) e^{i (\omega_k x^0 - \vec{k} \cdot \vec{x})} \Big).
	\end{equation}
	
	\item 场算符是
	\begin{equation}
		A_\mu(x) = \int \frac{d^3 k}{(2 \pi)^{3 / 2} \sqrt{2 \omega_k}} \sum_{i = 1}^3 \Big( \epsilon^{(i)}_\mu(\vec{k}) a^{(i)}_{\vec{k}} e^{- i (\omega_k x^0 - \vec{k} \cdot \vec{x})} + \epsilon^{(i) *}_\mu(\vec{k}) a^{(i) \dag}_{\vec{k}} e^{i (\omega_k x^0 - \vec{k} \cdot \vec{x})} \Big),
	\end{equation}
	其中 $k^\mu \epsilon^{(i)}_\mu(\vec{k}) = 0$, 并满足归一化条件,
	\begin{equation}
		\begin{dcases}
			\epsilon^{(i) *}_\mu(\vec{k}) \epsilon^{(j) \mu}(\vec{k}) = - \delta_{i j} \\
			\sum_{i = 1}^3 \epsilon^{(i) *}_\mu(\vec{k}) \epsilon^{(i)}_\nu(\vec{k}) = - \eta_{\mu \nu} + \frac{k_\mu k_\nu}{\mu^2}
		\end{dcases}.
	\end{equation}
	\begin{itemize}
		\item 对于 $k^\mu = (\omega_k, 0, 0, k)^T$, $\epsilon^{(i)}_\mu(\vec{k})$ 的具体形式为
		\begin{equation}
			\epsilon^{(1)}_\mu(\vec{k}) = (0, 1, 0, 0), \quad \epsilon^{(2)}_\mu(\vec{k}) = (0, 0, 1, 0), \quad \epsilon^{(3)}_\mu(\vec{k}) = (- \frac{k}{\mu}, 0, 0, \frac{\omega_k}{\mu}).
		\end{equation}
		
		\item 产生湮灭算符满足
		\begin{equation}
			[a^{(i)}_{\vec{k}_1}, a^{(j) \dag}_{\vec{k}_2}] = \delta_{i j} \delta^{(3)}(\vec{k}_1 - \vec{k}_2).
		\end{equation}
		正则对易关系为
		\begin{equation}
			\begin{dcases}
				[\vec{\pi}(t, \vec{x}), A_0(t, \vec{y})] = 0 \\
				[\pi_i(t, \vec{x}), A_j(t, \vec{y})] = i \int \frac{d^3 k}{(2 \pi)^3} \Big( - \eta_{i j} + \frac{k_i k_j}{\mu^2} \Big) e^{i \vec{k}(\vec{x} - \vec{y})} = i \Big( \delta_{i j} - \frac{\partial_i \partial_j}{\mu^2} \Big) \delta^{(3)}(\vec{x} - \vec{y})
			\end{dcases}.
		\end{equation}
	\end{itemize}
	
	\begin{tcolorbox}[title=calculation:]
		对易关系 (下式中所有动量都 on shell)
		\begin{align}
			& [\pi^i(t, \vec{x}), A_\mu(t, \vec{y})] \notag \\
			=& i \int \frac{d^3 k_1}{(2 \pi)^3 2 \omega_{k_1}} \omega_{k_1} \Big( \eta^{i j} + \frac{k_1^i k_1^j}{\omega_{k_1}^2} \Big) \sum_{k = 1}^3 \Big( \epsilon^{(k)}_j(\vec{k}_1) \epsilon^{(k) *}_\mu(\vec{k}_1) e^{- i k_1 \cdot (x - y)} + \epsilon^{(k) *}_j(\vec{k}_1) \epsilon^{(k)}_\mu(\vec{k}_1) e^{i k_1 \cdot (x - y)} \Big) \notag \\
			= & i \int \frac{d^3 k_1}{(2 \pi)^3 2 \omega_{k_1}} \omega_{k_1} \Big( - \delta^i_\mu + \frac{k_1^i k_{1 \mu}}{\mu^2} \Big) (e^{- i k_1 \cdot (x - y)} + e^{i k_1 \cdot (x - y)}),
		\end{align}
		如果指标 $\mu = 0$, 被积函数是奇函数, 结果为零, 所以...
	\end{tcolorbox}
	
	\item 传播子为
	\begin{equation}
		i D_{\mu \nu}(x - y) \equiv \braket{0 | T(A_\mu(x) A_\nu(y)) | 0} = \int \frac{d^4 k}{(2 \pi)^4} \frac{i}{k^2 - \mu^2 + i \epsilon} \Big( - \eta_{\mu \nu} + \frac{k_\mu k_\nu}{\mu^2} \Big) e^{- i k \cdot (x - y)}.
	\end{equation}
	
	\begin{tcolorbox}[title=calculation:]
		考虑
		\begin{align}
			\braket{0 | A_\mu(x) A_\nu(y) | 0} &= \int \frac{d^3 k_1}{(2 \pi)^3 2 \omega_{k_1}} \sum_{i = 1}^3 \epsilon^{(i)}_\mu(\vec{k}_1) \epsilon^{(i) *}_\nu(\vec{k}_1) e^{- i k_1 \cdot (x - y)} \notag \\
			&= \int \frac{d^3 k}{(2 \pi)^3 2 \omega_k} \Big( - \eta_{\mu \nu} + \frac{k_\mu k_\nu}{\mu^2} \Big) e^{- i k \cdot (x - y)},
		\end{align}
		因此,
		\begin{equation}
			i D_{\mu \nu}(x) = \int \frac{d^3 k}{(2 \pi)^3 2 \omega_k} \Big( - \eta_{\mu \nu} + \frac{k_\mu k_\nu}{\mu^2} \Big) \Big( \theta(t) e^{- i (\omega_k t - \vec{k} \cdot \vec{x})} + \theta(- t) e^{i (\omega_k t - \vec{k} \cdot \vec{x})} \Big),
		\end{equation}
		对于第二项,
		\begin{align}
			& I_{\mu \nu}(x) = \int \frac{d^3 k}{(2 \pi)^3 2 \omega_k} \Big( - \eta_{\mu \nu} + \frac{k_\mu k_\nu}{\mu^2} \Big) \theta(- t) e^{i (\omega_k t - \vec{k} \cdot \vec{x})} \notag \\
			\Longrightarrow & \begin{dcases}
				I_{0 0}(x) = \int \frac{d^3 k}{(2 \pi)^3 2 \omega_k} \Big( - \eta_{0 0} + \frac{\omega_k \omega_k}{\mu^2} \Big) \theta(- t) e^{i (\omega_k t + \vec{k} \cdot \vec{x})} \\
				I_{0 i}(x) = \int \frac{d^3 k}{(2 \pi)^3 2 \omega_k} \Big( - \eta_{0 i} + \frac{\omega_k (\mathcolor{red}{-} k_i)}{\mu^2} \Big) \theta(- t) e^{i (\omega_k t + \vec{k} \cdot \vec{x})} \\
				I_{i j}(x) = \int \frac{d^3 k}{(2 \pi)^3 2 \omega_k} \Big( - \eta_{i j} + \frac{(- k_i) (- k_j)}{\mu^2} \Big) \theta(- t) e^{i (\omega_k t + \vec{k} \cdot \vec{x})}
			\end{dcases} \notag \\
			\Longrightarrow & I_{\mu \nu}(x) = \int \frac{d^3 k}{(2 \pi)^3 2 \omega_k} \Big( - \eta_{\mu \nu} + \frac{k_\mu k_\nu}{\mu^2} \Big) \theta(- t) e^{- i k \cdot x} \quad \text{with} \quad k_0 = - \omega_k,
		\end{align}
		因此...
	\end{tcolorbox}
\end{itemize}

\section{massless}
\begin{itemize}
	\item 本节分别在 Lorentz gauge 和 Coulomb gauge 下对电磁场量子化.
	
	\item 电磁场的广义动量为
	\begin{equation}
		\pi^{\mu, \nu} = \frac{\delta \mathcal{L}}{\delta \partial_\mu A_\nu} = - \partial^\mu A^\nu + \partial^\nu A^\mu = F^{\nu \mu},
	\end{equation}
	简写 $\pi^\mu \equiv \pi^{0 \mu}$, 有
	\begin{equation}
		\pi^0 = 0, \quad \vec{\pi} = \vec{E}.
	\end{equation}
	
	\item $\vec{E}$ 是可观测量, 因此其形式不受 gauges 影响,
	\begin{equation}
		\vec{\pi} = i \int \frac{d^3 k}{(2 \pi)^3 2 |\vec{k}|} |\vec{k}| \Big( \vec{\tilde{\mathcal{A}}}_\perp(k) e^{- i (|\vec{k}| x^0 - \vec{k} \cdot \vec{x})} - \vec{\tilde{\mathcal{A}}}^*_\perp(k) e^{i (|\vec{k}| x^0 - \vec{k} \cdot \vec{x})} \Big).
	\end{equation}
\end{itemize}

\subsection{in Coulomb gauge}
\begin{itemize}
	\item Coulomb gauge 似乎更常见.
	
	\item 场算符是
	\begin{equation}
		\vec{A}(x) = \int \frac{d^3 k}{(2 \pi)^{3 / 2} \sqrt{2 |\vec{k}|}} \sum_{i = 1}^2 \Big( \vec{\epsilon}^{\, (i)}(\vec{k}) a^{(i)}_{\vec{k}} e^{- i (\omega_k x^0 - \vec{k} \cdot \vec{x})} + \vec{\epsilon}^{\, (i) *}(\vec{k}) a^{(i) \dag}_{\vec{k}} e^{i (\omega_k x^0 - \vec{k} \cdot \vec{x})} \Big),
	\end{equation}
	其中 $\vec{k} \cdot \vec{\epsilon}^{\, (i)}(\vec{k}) = 0$, 并满足归一化条件
	\begin{equation}
		\begin{dcases}
			\vec{\epsilon}^{\, (i) *}(\vec{k}) \cdot \vec{\epsilon}^{\, (j)}(\vec{k}) = \delta_{i j} \\
			\sum_{i = 1}^2 \vec{\epsilon}^{\, (i) *}(\vec{k}) \vec{\epsilon}^{\, (i)}(\vec{k}) = 1 - \frac{\vec{k} \vec{k}}{|\vec{k}|^2}
		\end{dcases}.
	\end{equation}
	\begin{itemize}
		\item 对于 $\vec{k} = (0, 0, k)^T$, $\vec{\epsilon}^{\, (i)}(\vec{k})$ 的具体形式为 (分别对应线偏振和圆偏振)
		\begin{equation}
			\begin{dcases}
				\vec{\epsilon}^{\, (1)}(\vec{k}) = (1, 0, 0)^T \\
				\vec{\epsilon}^{\, (2)}(\vec{k}) = (0, 1, 0)^T
			\end{dcases} \quad \text{or} \quad \begin{dcases}
				\vec{\epsilon}^{\, (1)}(\vec{k}) = \frac{1}{\sqrt{2}} (1, - i, 0)^T \\
				\vec{\epsilon}^{\, (2)}(\vec{k}) = \frac{1}{\sqrt{2}} (1, + i, 0)^T
			\end{dcases}.
		\end{equation}
		
		\item 产生湮灭算符满足 $[a^{(i)}_{\vec{k}_1}, a^{(j) \dag}_{\vec{k}_2}] = \delta_{i j} \delta^{(3)}(\vec{k}_1 - \vec{k}_2)$. 正则对易关系为 (差一个负号 \textcolor{red}{(?)} $\rightarrow$ 来自度规)
		\begin{equation}
			[\pi_i(t, \vec{x}), A_j(t, \vec{y})] = i \Big( \delta_{i j} - \frac{\partial_i \partial_j}{\nabla^2} \Big) \delta^{(3)}(\vec{x} - \vec{y}).
		\end{equation}
		
		\begin{tcolorbox}[title=calculation:]
			\begin{align}
				& [\pi_i(t, \vec{x}), A_j(t, \vec{y})] \notag \\
				=& i \int \frac{d^3 k_1}{(2 \pi)^3 2} \sum_{k = 1}^2 \Big( \epsilon^{(k)}_i(\vec{k}_1) \epsilon^{(k) *}_j(\vec{k}_1) e^{- i k_1 \cdot (x - y)} + \epsilon^{(k) *}_i(\vec{k}_1) \epsilon^{(k)}_j(\vec{k}_1) e^{i k_1 \cdot (x - y)} \Big) \notag \\
				=& i \int \frac{d^3 k}{(2 \pi)^3} \Big( \delta_{i j} - \frac{k_i k_j}{|\vec{k}|^2} \Big) e^{i \vec{k} \cdot (\vec{x} - \vec{y})}.
			\end{align}
		\end{tcolorbox}
	\end{itemize}
	
	\item 传播子为
	\begin{equation}
		i D_{i j}(x - y) = \int \frac{d^4 k}{(2 \pi)^4} \frac{i}{k^2 + i \epsilon} \Big( \delta_{i j} - \frac{k_i k_j}{|\vec{k}|^2} \Big) e^{- i k \cdot (x - y)}.
	\end{equation}
	
	\begin{tcolorbox}[title=calculation:]
		考虑
		\begin{align}
			\braket{0 | A_i(x) A_j(y) | 0} &= \int \frac{d^3 k}{(2 \pi)^3 2 |\vec{k}|} \sum_{k = 1}^2 \epsilon^{(k)}_i(\vec{k}) \epsilon^{(k) *}_j(\vec{k}) e^{- i k \cdot (x - y)} \notag \\
			&= \int \frac{d^3 k}{(2 \pi)^3 2 |\vec{k}|} \Big( \delta_{i j} - \frac{k_i k_j}{|\vec{k}|^2} \Big) e^{- i k \cdot (x - y)},
		\end{align}
		因此
		\begin{align}
			i D_{i j}(x - y) = \int \frac{d^4 k}{(2 \pi)^4} \frac{i}{k^2 + i \epsilon} \Big( \delta_{i j} - \frac{k_i k_j}{|\vec{k}|^2} \Big) e^{- i k \cdot (x - y)}.
		\end{align}
	\end{tcolorbox}
\end{itemize}

\subsection{in Lorentz gauge}
\begin{itemize}
	\item \textcolor{red}{完全没懂.}
	
	\item 在 Lorentz gauge 下, 算符 $\partial_\mu A^\mu \neq 0$, 并修改 Lagrangian 为
	\begin{equation}
		\mathcal{L} = - \frac{1}{4} F_{\mu \nu} F^{\mu \nu} - \frac{1}{2 \alpha} (\partial_\mu A^\mu)^2,
	\end{equation}
	那么
	\begin{equation} \label{12.2.13}
		\begin{dcases}
			(1 - \frac{1}{\alpha}) \partial_\mu \partial_\nu A^\nu - \partial_\nu \partial^\nu A_\mu = 0 \\
			\pi^{\mu, \nu} = \frac{\delta \mathcal{L}}{\delta \partial_\mu A_\nu} = F^{\nu \mu} - \frac{1}{\alpha} (\partial_\rho A^\rho) \eta^{\mu \nu}
		\end{dcases}.
	\end{equation}
	因此 $\pi^0 = - \frac{1}{\alpha} \partial_\mu A^\mu, \vec{\pi} = \vec{E}$.
	\begin{itemize}
		\item $\alpha = 1$ 称作 Feynman gauge, $\alpha = 0$ 称作 Landau gauge, $\alpha \rightarrow \infty$ 称作 unitary gauge.
	\end{itemize}
	
	\item 场方程 \eqref{12.2.13} 的 Green's function 为
	\begin{equation}
		\tilde{G}^{(\pm) \mu \nu}(k) = - \frac{1}{k^2 \pm i \epsilon} \Big( \eta^{\mu \nu} - (1 - \alpha) \frac{k^\mu k^\nu}{k^2} \Big).
	\end{equation}
	
	\item 取 $\alpha = 1$, 那么场算符为
	\begin{equation}
		A_\mu(x) = \int \frac{d^3 k}{(2 \pi)^{3 / 2} \sqrt{2 |\vec{k}|}} \sum_{\nu = 0}^3 \Big( \epsilon^{(\nu)}_\mu(\vec{k}) a^{(\nu)}_{\vec{k}} e^{- i k \cdot x} + \epsilon^{(\nu) *}_\mu(\vec{k}) a^{(\nu) \dag}_{\vec{k}} e^{i k \cdot x} \Big),
	\end{equation}
	其中 $\epsilon^{(\nu)}_\mu(\vec{k})$ 满足
	\begin{equation}
		\begin{dcases}
			\epsilon^{(\mu)}(\vec{k}) \cdot \epsilon^{(\nu)}(\vec{k}) = \eta^{\mu \nu} \\
			\sum_{\rho, \sigma = 0}^3 \epsilon^{(\rho)}_\mu(\vec{k}) \epsilon^{(\sigma)}_\nu(\vec{k}) \eta_{\rho \sigma} = \eta_{\mu \nu} \\
			k \cdot \epsilon^{(\mu = 1, 2)}(\vec{k}) = 0, \quad k \cdot \epsilon^{(\mu = 0, 3)}(\vec{k}) = |\vec{k}|
		\end{dcases}.
	\end{equation}
	\begin{itemize}
		\item 对于 $k^\mu = (k, 0, 0, k)^T$, $\epsilon^{(\nu)}_\mu(\vec{k})$ 的具体形式为
		\begin{equation}
			\epsilon^{(0)}_\mu(\vec{k}) = \begin{pmatrix}
				1 \\
				0 \\
				0 \\
				0
			\end{pmatrix}, \quad \epsilon^{(1)}_\mu(\vec{k}) = \begin{pmatrix}
				0 \\
				1 \\
				0 \\
				0
			\end{pmatrix}, \quad \epsilon^{(2)}_\mu(\vec{k}) = \begin{pmatrix}
				0 \\
				0 \\
				1 \\
				0
			\end{pmatrix}, \quad \epsilon^{(3)}_\mu(\vec{k}) = \begin{pmatrix}
				0 \\
				0 \\
				0 \\
				1
			\end{pmatrix}.
		\end{equation}
		
		\item 正则对易关系为
		\begin{equation}
			[\pi_\mu(t, \vec{x}), A_\nu(t, \vec{y})] = 
		\end{equation}
		
		\begin{tcolorbox}[title=calculation:]
			首先计算 $\pi^\mu(x)$,
			\begin{align}
				\pi^0(x) &= - \partial_\mu A^\mu = i \int \frac{d^3 k}{(2 \pi)^{3 / 2} \sqrt{2 |\vec{k}|}} \sum_{\mu = 0, 3} \Big( k \cdot \epsilon^{(\mu)}(\vec{k}) a^{(\mu)}_{\vec{k}} e^{- i k \cdot x} - k \cdot \epsilon^{(\mu) *}(\vec{k}) a^{(\mu) \dag}_{\vec{k}} e^{i k \cdot x} \Big) \notag \\
				&= i \int \frac{d^3 k}{(2 \pi)^{3 / 2} \sqrt{2 |\vec{k}|}} |\vec{k}| \sum_{\mu = 0, 3} \Big( a^{(\mu)}_{\vec{k}} e^{- i k \cdot x} - a^{(\mu) \dag}_{\vec{k}} e^{i k \cdot x} \Big)
			\end{align}
			注意到 $\sum_{\mu = 0}^3 \epsilon^{(\mu) 0}(\vec{k}) a^{(\mu)}_{\vec{k}} = \sum_{\mu = 0, 3} a^{(\mu)}_{\vec{k}}$ \textcolor{red}{(?)}.
			
			David Tong 的 (6.40) 式不满足 $\pi^0 = - \partial_\mu A^\mu$ \textcolor{red}{(?)}.
		\end{tcolorbox}
	\end{itemize}
\end{itemize}
