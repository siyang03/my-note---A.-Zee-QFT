\chapter{the Dirac equation}
\section{Dirac equation}
\begin{itemize}
	\item A. Zee: our discussion provides a unified view of the equations of motion in relativistic physics: they just project out the unphysical components.
	
	\item the Dirac equation is
	\begin{equation}
		(i \gamma^\mu \partial_\mu - m) \Psi = 0 \iff (\gamma^\mu p_\mu - m) \tilde{\Psi} = 0 \Longrightarrow \begin{dcases}
			i \sigma^\mu \partial_\mu \psi_R - m \psi_L = 0 \\
			i \bar{\sigma}^\mu \partial_\mu \psi_L - m \psi_R = 0
		\end{dcases}.
	\end{equation}
	首先可以看出 $\Psi$ 满足 Klein-Gordan equation,
	\begin{align}
		& (i \gamma^\mu \partial_\mu - m) (i \gamma^\nu \partial_\nu - m) \Psi = \Big( - \frac{1}{2} \{\gamma^\mu, \gamma^\nu\} \partial_\mu \partial_\nu - 2 i m \gamma^\mu \partial_\mu + m^2 \Big) \Psi = 0 \notag \\
		\Longrightarrow & (- \partial^2 - m^2) \Psi = 0.
	\end{align}
	\begin{itemize}
		\item 在粒子静止系下 $p_\mu = (m, 0, 0, 0)$, Dirac 方程给出 (这里采用 Dirac basis)
		\begin{equation}
			(\gamma^0 - 1) \tilde{\Psi}_\text{Dirac} = 0 \Longrightarrow \begin{pmatrix}
				0 & \\
				& I
			\end{pmatrix} \tilde{\Psi}_\text{Dirac} = 0.
		\end{equation}
		因此, $\tilde{\Psi}$ 的后两个分量为零 $\Longrightarrow \Psi$ 只有两个自由度.
	\end{itemize}
	
	\item Dirac 方程的 Lorentz covariance 见 \eqref{6.2.10}.
\end{itemize}

\section{Dirac Lagrangian}
\begin{itemize}
	\item 根据 \eqref{6.2.19} 以及之前标量场的计算经验, 可知
	\begin{equation} \label{7.2.1}
		\mathcal{L} = \bar{\Psi} (i \gamma^\mu \partial_\mu - m) \Psi = (- i \partial_\mu \bar{\Psi} \gamma^\mu - m \bar{\Psi}) \Psi + \text{total diff.},
	\end{equation}
	其中, 与复标量场论中类似, 可以把 $\Psi, \Psi^\dag$ 或 $\Psi, \bar{\Psi}$ 视为独立变量.
\end{itemize}

\section{chirality or handedness}
\begin{itemize}
	\item parity transformation 会把 left spinor 变成 right spinor and vice versa,
	\begin{equation}
		\gamma^0 \Psi_L = \begin{pmatrix}
			0 \\
			\psi_L
		\end{pmatrix}, \quad \gamma^0 \Psi_R = \begin{pmatrix}
			\psi_R \\
			0
		\end{pmatrix}.
	\end{equation}
	
	\item 把 Lagrangian 中的 $\Psi$ 拆开,
	\begin{align}
		\mathcal{L} &= \bar{\Psi}_L (i \cancel{\partial}) \Psi_L + \bar{\Psi}_R (i \cancel{\partial}) \Psi_R - m (\bar{\Psi}_L \Psi_R + \bar{\Psi}_R \Psi_L) \notag \\
		&= \psi_L^\dag i \bar{\sigma}^\mu \partial_\mu \psi_L + \psi_R^\dag i \sigma^\mu \partial_\mu \psi_R - m (\psi_L^\dag \psi_R + \psi_R^\dag \psi_L),
	\end{align}
	其中注意到了 $\gamma^0 \gamma^\mu$ 的非对角分块为零.
\end{itemize}

\subsection{internal vector symmetry} \label{subsection 7.3.1}
\begin{itemize}
	\item 做变换 $\Psi \mapsto e^{i \theta} \Psi$, Lagrangian 保持不变, 利用 Noether's theorem 得到守恒流 (见 section \ref{D.2}),
	\begin{equation} \label{7.3.3}
		J_V^\mu = \bar{\Psi} \gamma^\mu \Psi,
	\end{equation}
	其中, 按照惯例省略了虚数 $i$.
	
	\begin{tcolorbox}[title=calculation:]
		计算广义动量,
		\begin{equation} \label{7.3.4}
			\begin{dcases}
				\pi_\Psi^\mu = \frac{\delta \mathcal{L}}{\delta \partial_\mu \Psi} = \bar{\Psi} i \gamma^\mu \\
				\pi_{\bar{\Psi}}^\mu = 0
			\end{dcases} \quad \text{or} \quad \begin{dcases}
				\pi_\Psi^\mu = 0 \\
				\pi_{\bar{\Psi}}^\mu = \frac{\delta \mathcal{L}}{\delta \partial_\mu \bar{\Psi}} = - i \gamma^\mu \Psi
			\end{dcases}.
		\end{equation}
		
		\noindent\rule[0.5ex]{\linewidth}{0.5pt} % horizontal line
		
		这里看起来有点奇怪 (canonical transformation), 需要再说明一下. 对于 \eqref{7.2.1} 第一个等号后边,
		\begin{equation}
			\begin{dcases}
				\pi_\Psi^\mu = \frac{\delta \mathcal{L}}{\delta \partial_\mu \Psi} = \bar{\Psi} i \gamma^\mu & \frac{\delta \mathcal{L}}{\delta \Psi} = - m \bar{\Psi} \\
				\pi_{\bar{\Psi}}^\mu = 0 & \frac{\delta \mathcal{L}}{\delta \bar{\Psi}} = (i \gamma^\mu \partial_\mu - m) \Psi
			\end{dcases} \Longrightarrow \begin{dcases}
				- (\partial_\mu \bar{\Psi}) i \gamma^\mu - m \bar{\Psi} = 0 \\
				(i \gamma^\mu \partial_\mu - m) \Psi = 0
			\end{dcases},
		\end{equation}
		对于 \eqref{7.2.1} 第二个等号后边, 忽略掉全微分项,
		\begin{equation}
			\begin{dcases}
				\pi_\Psi^\mu = 0 & \frac{\delta \mathcal{L}}{\delta \Psi} = - i \partial_\mu \bar{\Psi} \gamma^\mu - m \bar{\Psi} \\
				\pi_{\bar{\Psi}}^\mu = \frac{\delta \mathcal{L}}{\delta \partial_\mu \bar{\Psi}} = - i \gamma^\mu \Psi & \frac{\delta \mathcal{L}}{\delta \bar{\Psi}} = - m \Psi
			\end{dcases} \Longrightarrow \begin{dcases}
				- i \partial_\mu \bar{\Psi} \gamma^\mu - m \bar{\Psi} = 0 \\
				(i \gamma^\mu \partial_\mu - m) \Psi = 0
			\end{dcases}.
		\end{equation}
	\end{tcolorbox}
\end{itemize}

\subsection{axial symmetry}
\begin{itemize}
	\item 做变换
	\begin{equation}
		\Psi \mapsto e^{i \theta \gamma^5} \Psi = \begin{pmatrix}
			e^{- i \theta} \Psi_L \\
			e^{i \theta} \Psi_R
		\end{pmatrix},
	\end{equation}
	在 $m = 0$ 时 Lagrangian 保持不变, 对应的守恒流为
	\begin{equation}
		J_A^\mu = \bar{\Psi} \gamma^\mu \gamma^5 \Psi,
	\end{equation}
	根据 \eqref{6.2.21}, 是一个 pseudovector.
\end{itemize}

\section{energy-momentum tensor and angular momentum}
\begin{itemize}
	\item Dirac 场的 energy-momentum tensor 为
	\begin{equation} \label{7.4.1}
		T_{\mu \nu} = i \bar{\Psi} \gamma_\mu \partial_\nu \Psi - \eta_{\mu \nu} \mathcal{L},
	\end{equation}
	其中, 对于满足运动方程的 Dirac 场, $\mathcal{L} = 0$.
	
	\item  Dirac 场的 angular momentum 为
	\begin{equation} \label{7.4.2}
		M^{\mu \nu \rho} = \frac{i}{2} \bar{\Psi} \gamma^\mu \sigma^{\nu \rho} \Psi(x) + (x^\nu T^{\mu \rho} - x^\rho T^{\mu \nu}).
	\end{equation}
	
	\begin{tcolorbox}[title=calculation:]
		做变换 $x \mapsto e^{\frac{1}{2} \lambda \omega_{\mu \nu} J^{\mu \nu}} x$, 那么
		\begin{align}
			& \Psi(x) \mapsto \Psi'(x') = e^{\frac{1}{4} \lambda \omega_{\mu \nu} \sigma^{\mu \nu}} \Psi(x) \notag \\
			\Longrightarrow & D_\lambda \Psi'(\mathcolor{red}{x}) = \frac{1}{4} \omega_{\mu \nu} \sigma^{\mu \nu} \Psi(x) - \frac{1}{2} \omega_{\mu \nu} \tensor{(J^{\mu \nu})}{^\rho_\sigma} x^\sigma \partial_\rho \Psi(x),
		\end{align}
		所以
		\begin{equation}
			J^\mu = \frac{i}{4} \omega_{\nu \rho} \bar{\Psi} \gamma^\mu \sigma^{\nu \rho} \Psi(x) + \cdots \Longrightarrow M^{\mu \nu \rho} = \frac{i}{2} \bar{\Psi} \gamma^\mu \sigma^{\nu \rho} \Psi(x) + (x^\nu T^{\mu \rho} - x^\rho T^{\mu \nu}).
		\end{equation}
	\end{tcolorbox}
\end{itemize}

\section{charge conjugation, parity and time reversal}
\begin{itemize}
	\item 沿用 A. Zee 的 notation, 变换映射分别用 $\mathcal{C}, \mathcal{P}, \mathcal{T}$ 表示, 相应的矩阵用 $C, P, T$ 表示.
\end{itemize}

\subsection{charge conjugation and antimatter}
\begin{itemize}
	\item 定义矩阵 $C$,
	\begin{equation}
		C = - \gamma^0 \gamma^2 \Longrightarrow C \gamma^0 = - i \begin{pmatrix}
			& & & 1 \\
			& & - 1 & \\
			& - 1 & & \\
			1 & & &
		\end{pmatrix} = \gamma^2 \Longrightarrow \begin{dcases}
			(\gamma^2)^{- 1} \gamma^\mu \gamma^2 = - \gamma^{\mu *} \\
			C^{- 1} \gamma^\mu C = - (\gamma^\mu)^T
		\end{dcases},
	\end{equation}
	因此 $- \gamma^{\mu *}$ 同样满足 Clifford algebra.
	\begin{itemize}
		\item 另外, 有 $(\gamma^2)^{- 1} = \gamma^{2 *} = - \gamma^2$ 和 $C^{- 1} = C$.
	\end{itemize}
	
	\begin{tcolorbox}[title=calculation:]
		\begin{equation}
			\gamma^0 C^{- 1} \gamma^0 C \gamma^0 = - \gamma^{\mu *} \Longrightarrow C^{- 1} \gamma^0 C = - \gamma^0 \gamma^{\mu *} \gamma^0 = - (\underbrace{\gamma^0 \gamma^\mu \gamma^0}_{= \gamma^{\mu \dag}})^*,
		\end{equation}
		其中用到了 $\gamma^0 \gamma^\mu \gamma^0 = \gamma^{\mu \dag}$, 见 \eqref{6.2.15}.
	\end{tcolorbox}
	
	\item $\Psi_c = \gamma^2 \Psi^*$ 满足如下方程 (对比 \eqref{7.6.1}),
	\begin{equation}
		(- i \gamma^{\mu *} (\partial_\mu - i e A_\mu) - m) \Psi^* = 0 \Longrightarrow (\gamma^2)^{- 1} (i \gamma^\mu (\partial_\mu - i e A_\mu) - m) \Psi_c = 0,
	\end{equation}
	可见 $\Psi_c$ 满足变换 $- e \mapsto + e$ 后的 Dirac 方程, $\Psi_c$ is the field of positron.
	
	\noindent\rule[0.5ex]{\linewidth}{0.5pt} % horizontal line
	
	\item 对于 Lorentz 变换, $e^{\frac{1}{2} \lambda \omega_{\mu \nu} J^{\mu \nu}}, \lambda \in [0, 1]$, 有
	\begin{equation}
		\begin{dcases}
			\Psi \mapsto \Psi'(x') = e^{\frac{1}{4} \omega_{\mu \nu} \sigma^{\mu \nu}} \Psi \\
			\Psi_c \mapsto \gamma^2 \underbrace{(\gamma^2)^{- 1}  e^{\frac{1}{4} \omega_{\mu \nu} \sigma^{\mu \nu}} \gamma^2 \Psi^*}_{= (\Psi'(x'))^*} = e^{\frac{1}{4} \omega_{\mu \nu} \sigma^{\mu \nu}} \Psi_c
		\end{dcases},
	\end{equation}
	可见 $\Psi_c$ 与 $\Psi$ 的变换形式相同.
\end{itemize}

\subsection{parity}
\begin{itemize}
	\item 对于 parity, 有 $x \rightarrow x' = (x^0, - \vec{x})$, 在 Dirac eq. 中
	\begin{equation}
		\gamma^0 \gamma^\mu = \tensor{P}{^\mu_\nu} \gamma^\nu \gamma^0 \Longrightarrow (i \gamma^\mu \partial'_\mu - m) \gamma^0 \Psi(x) = 0,
	\end{equation}
	因此
	\begin{equation}
		\mathcal{P} : \Psi(x) \mapsto \Psi'(x') = \gamma^0 \Psi(x).
	\end{equation}
\end{itemize}

\subsection{time reversal}
\begin{itemize}
	\item 时间反演算符为
	\begin{equation}
		T = (i \sigma_2 \otimes I) K = \gamma^1 \gamma^3 K,
	\end{equation}
	其中 $K$ 是 complex conjugation operator (见 appendix \ref{E}). 另外, 有 $T^2 = - 1$, 符合预期.
	
	\begin{tcolorbox}[title=proof:]
		时间反演之后, $\Psi'(t') = T \Psi(t)$ 满足如下方程,
		\begin{equation}
			i \frac{\partial}{\partial t'} \Psi'(t') = H \Psi'(x'),
		\end{equation}
		其中
		\begin{equation}
			H = - i \gamma^0 \gamma^i \frac{\partial}{\partial x^i} + \gamma^0 m,
		\end{equation}
		且 Hamiltonian 满足时间反演不变, $H'(t') \equiv T H(t) T^\dag = H(t)$, 即 (其中 $T = U K$)
		\begin{equation}
			\begin{dcases}
				T (i \gamma^0 \gamma^i) T^\dag = i \gamma^0 \gamma^i \\
				T \gamma^0 T^\dag = \gamma^0
			\end{dcases} \Longrightarrow \begin{dcases}
				U (- i \gamma^0 \gamma^{i *}) U^\dag = U (- i \gamma^0 \gamma^2 \gamma^i \gamma^2) U^\dag = i \gamma^0 \gamma^i \\
				[U, \gamma^0] = 0
			\end{dcases},
		\end{equation}
		满足以上要求的 $U$ 具有以下形式,
		\begin{equation}
			U = \begin{pmatrix}
				a \sigma_2 & b \sigma_2 \\
				b \sigma_2 & a \sigma_2
			\end{pmatrix}, \quad \text{with} \quad \begin{dcases}
				|a|^2 + |b|^2 = 1 \\
				a^* b + b^* a = 0
			\end{dcases},
		\end{equation}
		不妨令 $a = i, b = 0$.
	\end{tcolorbox}
\end{itemize}

\subsection{CPT theorem}
\begin{itemize}
	\item 在 CPT 变换下
	\begin{equation}
		\mathcal{CPT} : \Psi(x) \mapsto \gamma^1 \gamma^3 K (\gamma^0 \gamma^2 \Psi^*) = \Omega \Psi = - i \gamma^5 \Psi.
	\end{equation}
	
	\item 任何 Lorentz covariant theory 都满足 CPT 不变性.
\end{itemize}

\section{interaction in QED}
\begin{itemize}
	\item 注意, 我们采用通常 (比如 Peskin) 的符号 $\mathcolor{red}{e = - |e|}$, 与 A. Zee 的符号 $e > 0$ 不同.
	
	\item QED 的 Lagrangian 为
	\begin{equation} \label{7.6.1}
		\mathcal{L}_\text{QED} = \bar{\Psi} (i \gamma^\mu D_\mu - m) \Psi - \frac{1}{4} F^{\mu \nu} F_{\mu \nu} + \frac{1}{2} \mu^2 A^\mu A_\mu,
	\end{equation}
	其中
	\begin{equation}
		D_\mu = \partial_\mu + i e A_\mu,
	\end{equation}
	可见电子和电磁场耦合项为 $- e A_\mu J_V^\mu$, 其中 $J_V^\mu$ 是 internal vector symmetry 的守恒流, 见 \eqref{7.3.3}.
	
	\item QED 里的 Dirac 方程为
	\begin{equation}
		(i \gamma^\mu (\partial_\mu + i e A_\mu) - m) \Psi = 0 \quad \text{and} \quad - i (\partial_\mu - i e A_\mu) \bar{\Psi} \gamma^\mu - m \bar{\Psi} = 0.
	\end{equation}
\end{itemize}

\section{Majorana neutrino}
\begin{itemize}
	\item 因为在 Lorentz 变换下, $\Psi, \Psi_c$ 行为相同, 因此 Majorana 方程同样满足 Lorentz covariance,
	\begin{equation}
		i \cancel{\partial} \Psi - m \Psi_c = 0 \quad \text{and} \quad i \cancel{\partial} \Psi_c - m \Psi = 0,
	\end{equation}
	因此
	\begin{equation}
		(- \partial^2 - m^2) \Psi = 0,
	\end{equation}
	满足 Klein-Gordon 方程.
	
	\begin{tcolorbox}[title=calculation:]
		\begin{equation}
			- \gamma^\mu \gamma^\nu \partial_\mu \partial_\nu \Psi = m (i \cancel{\partial}) \Psi_c = m^2 \Psi.
		\end{equation}
	\end{tcolorbox}
	
	\item Majorana 方程对应的 Lagrangian 为
	\begin{equation}
		\mathcal{L} = \bar{\Psi} i \cancel{\partial} \Psi - \frac{1}{2} m (\Psi^T C \Psi + \bar{\Psi} C \bar{\Psi}^T),
	\end{equation}
	相应的广义动量为
	\begin{equation}
		\begin{dcases}
			\pi_\Psi^\mu = \bar{\Psi} i \gamma^\mu & \frac{\delta \mathcal{L}}{\delta \Psi} = - m \Psi^T C \\
			\pi_{\bar{\Psi}}^\mu = 0 & \frac{\delta \mathcal{L}}{\delta \bar{\Psi}} = i \cancel{\partial} \Psi - m C \bar{\Psi}^T = i \cancel{\partial} \Psi - m \Psi_c
		\end{dcases}.
	\end{equation}
	\begin{itemize}
		\item 注意, $\Psi$ 应该被当作 Grassmann numbers, 因此, 对于反对称矩阵 $C$, 有 $\Psi^T C \Psi, \bar{\Psi} C \bar{\Psi}^T \neq 0$.
	\end{itemize}
	
	\begin{tcolorbox}[title=calculation:]
		对 $\Psi$ 变分得到
		\begin{align}
			0 &= \frac{\delta \mathcal{L}}{\delta \Psi} - \partial_\mu \pi_\Psi^\mu \notag \\
			&= - m \Psi^T C - i \partial_\mu \bar{\Psi} \gamma^\mu \notag \\
			&= (- m \Psi^T - i \partial_\mu \bar{\Psi} \gamma^\mu C) C \notag \\
			&= (- m \Psi + i C (\gamma^\mu)^T \gamma^0 \partial_\mu \Psi^*)^T C,
		\end{align}
		其中
		\begin{equation}
			C (\gamma^\mu)^T \gamma^0 = C (- C^{- 1} \gamma^\mu C) \gamma^0 = - \gamma^\mu C \gamma^0 = - \gamma^\mu \gamma^2,
		\end{equation}
		代入, 得到 \textcolor{red}{(?)},
		\begin{equation}
			- i \cancel{\partial} \Psi_c - m \Psi = 0.
		\end{equation}
	\end{tcolorbox}
	
	\item Majorana eq. v.s. Dirac eq.:
	\begin{itemize}
		\item Majorana eq. 只适用于 electrically neutral fields \textcolor{red}{(?)}.
		
		\item Majorana eq. preserves handedness \textcolor{red}{(?)}.
	\end{itemize}
\end{itemize}
