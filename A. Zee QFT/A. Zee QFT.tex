\documentclass[10pt, a4paper]{report}
\usepackage[T1]{fontenc}
\usepackage[left=2cm, right=2cm, top=2cm, bottom=2cm]{geometry}
\usepackage{graphicx}
\usepackage{mathtools}
\usepackage{amssymb}
\usepackage{mathrsfs} % to use \mathscr{...}
\usepackage{xcolor} % \textcolor{red}{...} & \mathcolor{red}{...}
\usepackage{nameref}
\usepackage{hyperref} %Automatically links \label and \ref commands; Always load last
\hypersetup{
	linktocpage,
	colorlinks=true, % false: boxed links; true: colored links
	linkcolor=red, % color of internal links
	citecolor=blue, % color of links to bibliography
	filecolor=magenta, % color of file links
	urlcolor=purple, % color of external links
} % hidelinks
%----------------------------------------------------------------------
\usepackage{fancyhdr}

\pagestyle{fancy}
\fancyhf{}  % Clear all headers and footers

% Redefine chapter mark to include chapter number and title
\renewcommand{\chaptermark}[1]{%
	\markboth{Chapter \thechapter\ #1}{}%
}

\fancyhead[L]{\nouppercase{\leftmark}}  % Header always on the left
\fancyfoot[C]{\thepage}                 % Centered page number
%----------------------------------------------------------------------
% for XeTeX
\usepackage{xeCJK}
\setCJKmainfont{GenRyuMin JP} % 源流明体
\setCJKmonofont{GenRyuMin JP} % font used in \texttt{} (just to prevent a warning)
% for LuaTeX
%\usepackage{luatexja-fontspec}
%\setmainjfont{GenRyuMin2 JP} % 源流明体
%----------------------------------------------------------------------
% to handle a large project
\usepackage{import}
%----------------------------------------------------------------------
\usepackage{indentfirst}
\setlength{\parindent}{0em} % 首行缩进
%----------------------------------------------------------------------
% horizontal line
%\noindent\rule[0.5ex]{\linewidth}{0.5pt} % horizontal line

\usepackage{dashrule}
%\noindent\hdashrule[0.5ex]{\linewidth}{0.5pt}{1mm} % horizontal dashed line
%----------------------------------------------------------------------
\usepackage[breakable]{tcolorbox} % box with color; highlight text \colorbox{yellow}{...}
\tcbset{breakable, colback=white, colframe=green!30!black, coltext=black!60!white, fonttitle=\bfseries}

\usepackage{tabularx} % LaTeX table
\usepackage{booktabs}

\usepackage{float} % appropriate way to handle figures
% example:
%\begin{figure}[H]
%	\centering
%	\includegraphics[scale=1]{figures/file-name}
%	\caption{...}
%\end{figure}
\usepackage{subcaption}
% following David Tong's convention, one should always add a caption to figures, but not to tables.
% use \vcenter{\hbox{\includegraphics[...]{...}}} to insert figures (e.g. Feynman diagrams) in equation environment.
%----------------------------------------------------------------------
\numberwithin{equation}{section}
\allowdisplaybreaks % allow page breaks inside math environments globally

\usepackage{braket}
\usepackage{cancel}

\usepackage{leftindex}
\usepackage{tensor} % how to handle indeces: https://tex.stackexchange.com/questions/11542/left-and-right-subscript-superscript

\usepackage{simpler-wick} % to use Wick contraction
%\usepackage[compat=1.1.0]{tikz-feynman}

% use \, in math environment
% except for texts embeded in math environment, use \ instead
%----------------------------------------------------------------------
\usepackage{orcidlink}
\title{\Huge \textbf{Quantum Field Theory}}
\author{万思扬}
\date{August 16, 2024}
%----------------------------------------------------------------------
\begin{document}
	\maketitle
	%\import{chapters/}{title page}
	
	\pdfbookmark{\contentsname}{toc} % add pdf bookmark to the ToC
	\tableofcontents
	
	\pagebreak
	
	\import{chapters/}{convention, notation, and units}
	
	\part{motivation and foundation}
	
	\import{chapters/}{free field theory}
	
	\import{chapters/}{Coulomb and Newton - repulsive and attraction}
	
	\import{chapters/}{Feynman diagrams}
	
	\import{chapters/}{canonical quantization}
	
	\import{chapters/}{disturbing the vacuum - Casimir effect}
	
	\part{Dirac and spinor} \label{part II}
	
	\import{chapters/}{the Dirac spinor}
	
	\import{chapters/}{the Dirac equation}
	
	\import{chapters/}{quantizing the Dirac field}
	
	\import{chapters/}{spin-statistics connection}
	
	\import{chapters/}{Grassmann path integrals and Feynman diagrams for Fermions}
	
	\part{quantum electrodynamics}
	
	\import{chapters/}{Maxwell's equations}
	
	\import{chapters/}{the quantization of the electromagnetic field}
	
	\appendix
	
	% Add the appendix to the table of contents and bookmarks
	\part*{Appendices}
	\addcontentsline{toc}{part}{Appendices}
	% --- Appendices get "Appendix <letter> <title>" ---
	\renewcommand{\chaptermark}[1]{%
		\markboth{Appendix \thechapter\ #1}{}%
	}
	
	\import{chapters/}{Dirac delta function}
	
	\import{chapters/}{Gaussian integrals and Gaussian-Berezin integrals}
	
	\import{chapters/}{perturbation theory in QM}
	
	\import{chapters/}{classical field theory and Noether's theorem}
	
	\import{chapters/}{antiunitary operator and time reversal}
\end{document}
